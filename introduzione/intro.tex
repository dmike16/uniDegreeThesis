\chapter*{Introduzione}
\phantomsection
\addcontentsline{toc}{chapter}{Introduzione}

Un campo della matematica che in questi anni si sta sviluppando
notevolmente, è quello che si occupa del trattamento delle immagini
sia 2D che 3D. E se da un lato sono migliorati i strumeti di
acquisizione delle stesse (come  scanner 2D/3D), dall'altro si sta
sviluppando sempre più l'idea, già da anni consolidata, di utilizare
equazioni alle derivate parziali (PDE) nel trattamento delle
immagini. Questa si basa sul concetto di \emph{scale-space} o
\emph{rappresentazione multiscala}, in cui l'immagine
iniziale $u_0(x)$ viene racchiusa in una famiglia $u(x,t)$, con
$t\in\mathbb{R}^+$ parametrizzante la scala, attraverso l'azione di un
\emph{kernel} (filtro) $K(\cdot,t)$ su di essa. Uno dei filtri più
usati  e studiati è quello \emph{gaussiano}, in questo caso  $u(x,t)$
è definita dalla convoluzione del filtro con $u_0(x)$ e può essere
vista come la soluzione dell'equazione del calore con dato iniziale $u_0(x)$
\[
u_t=\Delta u.
\]
Una \emph{scale-space} cosi costruita  è stato dimostrato avere, tra
le proprità principali, la linearità e la \emph{cusualità}, cioè
non viene aggiunta  ``informazione'' quando ci spostiamo da una scala
$t_1$ ad una $t_2>t_1$, che matematicamente può essere tradotto con il
principio del massimo ed  è anche legata alla proprietà di
\emph{semi-gruppo}, vale a dire che $u(x,t_2)$ può essere ottenuta da
$u(x,t_1)$ senza conoscere $u_0(x)$, con $t_1<t_2$.   Nonostante queste
buone proprietà, tuttavia, il flusso calorico classico presenta alcuni
problemi, in particolare non è intrinseco alla giometria
dell'immagine. Questo problema è stato risolto
(\cite[][]{sapiro:tann}) sostituendo il flusso calorico classico con
un flusso  geometrico, che in generale dipende da caretteristiche
geometriche dell'oggetto in esame come ad esempio la curavtura. Uno
\emph{scale-space} che si evolve secondo questi flussi, essendo
ottenuto come soluzione di un PDE, continua a soddisfare le proprietà
di causualità (principio del massimo) e di semi gruppo ed in più è
intrinseco alle proprietà dell'oggetto in esame. 
Questo concetto è alla base di molte applicazioni nella computer
grafica e nell'analisi dell'immagini, in modo particolare può essere
usato per processi di \emph{denoising} di un immagine. E' noto che
quando acquisiamo o inviamo un immagine questa si può deteriorare, cioè 
viene aggiunto del \emph{rumore}, e quindi abbiamo bisogno di
filtrarla per eliminare le discontinuità. Il modello di filtraggio per
immagini 3D che presentiamo in questo lavoro, si basa su uno dei flussi più
studiati nella letteratura, il flusso per curvatura media che rientra
nel gruppo dei flussi geometrici precedenti. Questo, tuttavia presenta
dei limiti, in quanto è stato dimostrato che superfici convesse tendono a
collassare in un punto, mentre superfici non convesse posso subire dei
cambiamenti topologici generando delle singolarità; un degli esempi
più famosi nella letteratura è quello del manubrio (\emph{dumbbell}). 
Grazie al lavoro di Sapiro in \cite[][]{gui:sapiro}, dove il flusso
per curvatura media è stato riscritto in modo tale da conservare il
volume e alla rappresentaizone \emph{level-set} delle superfici
introdotta da Oscher e Sethian, abbiamo riscritto l'equazione per
curvatura media che preserva il volume ottenendo un PDE parabolico non
lineare del secondo ordine, singolare nei punti dove il gradiente della
soluzione si annulla. Quindi è stato necessario immergere il tutto
nella teoria delle soluzioni viscose; anche se, eccecion fatta per
alcuni casi particolari, ancora non sono stati raggiunti risultati
definitivi sull'esistenza e unicità delle soluzioni nel caso della
nostra equazione. Discretizzando tale equazione con uno schema
semi-lagrangiano e un schema \emph{ad hoc} nel caso di gradiente
nullo; abbiamo ralizzato un algoritmo di filtraggio (scritto
in\emph{C-Language}) che preserva il volume; testandolo sia su superfici
geometriche che su immagini reali. Abbiamo messo in evidenza come il metodo
\emph{volume preserving}, nel caso di superfici geometriche, riesce a
superare alcuni limiti dello schema classico, tra cui il collasso in
un punto di superfici convesse e ad ritardare i cambiamenti topologici
che emergono per quelle non convesse, come nel caso del manubrio,
mentre nel caso dell'immagine reale, su cui l'abbiamo testato, non
sone emerse le differenze a causa  dell'impossibilità di effettuare
troppe iterazioni.

La tesi è strutturata come segue: nel Capitolo §\ref{cap:cap1} abbiamo
richiamato alcune nozioni sulle superfici in forma level-set,
presentato l'equazione di evoluzione per curvatura media, calcolato il
termine che ci conserva il volume e riscritto il tutto in formulazione
level-set. Nel capitolo §\ref{cp:cp2}, abbiamo esposto i concetti
realitivi a soluzioni di viscosità, riportando alcuni teoremi di
esistenza e unicità in casi particolari, necessari per fornire un
robusto background teorico alla nostra equazione. Lo schema del moto
per curvatura media classico (MCM) e la sua versione che preserva il
volume (VPMCM) sono stati ottenuti nel capitolo §\ref{cp:cp2-00},
riportando anche il conto per la consistanza di MCM e di una versione
semplificata di PVMCM. Infine nel capitolo §\ref{cap:cp4} abbiamo
sia mostato il buon comportamento dello schema VPMCM nel caso di superfici
geometriche, sia in presenza che in assenza di rumore, e testato su di
un'immagine reale riportando le dovute considerazioni. 
   
