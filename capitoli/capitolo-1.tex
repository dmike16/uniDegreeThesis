\chapter{Evoluzione Geometrica Per Curvatura Media}
'WIP'
%%%%%%%%%%%%%%%%%%%%%%%%%%%%%%%%%%%%%%%%%%%%%%%%%%%%
%
% Section 1.1
%
%%%%%%%%%%%%%%%%%%%%%%%%%%%%%%%%%%%%%%%%%%%%%%%%%%%%
\section{Qualcosa sulle Superfici in $\mathbb{R}^3$}

Consideriamo un sottoinsieme $S\subseteq\mathbb{R}^3$, diremo che è una superficie regolare se per ogni punto $p\in S$ esiste un intorno $V$ di $p$ in $\mathbb{R}^3$, un aperto $U\subset\mathbb{R}^2$ e un'applicazione differenziabile $\mathcal{S}:U\longrightarrow V\cap S$ tale che:
\begin{enumerate}
  \item $\mathcal{S}$ è un omeomorfismo,
  \item il differenziale $d\mathcal{S}$ ha rango massimo in ogni punto di $U$.
\end{enumerate}
L'applicazione $\mathcal{S}$ si dice parametrizzazione regolare in un intorno del punto $p$. Mentre l'insieme $V\cap S=\mathcal{S}(U)$ si dice porzione di superficie parametrica regolare. La condizione di regolarità, ci esclude superfici che intersecano se stesse.

Siano $u$ e $v$ coordinate su $U\subset\mathbb{R}^2$ , $S = \mathcal{S}(U)$ e $p=\mathcal{S}(u_0,v_0)$; si definisce spazio tangente ad $S$ nel punto $p$, e si indica con $T_pS$, l'immgine di $\mathbb{R}^2$ tramite l'applicazione $d\mathcal{S}_{(u_0,v_0)}$:
\[
(a,b)\in\mathbb{R}^2\quad d\mathcal{S}_{(u_0,v_0)}(a,b)= 
\frac{d}{dt}\mathcal{S}(u_0+at,v_0+bt)|_{t=0}=
\]
\[
=a\mathcal{S}_u(u_0,v_0)+b\mathcal{S}_v(u_0,v_0),
\]
con $\mathcal{S}_u(u_0,v_0)$ e $\mathcal{S}_v(u_0,v_0)$ colonne della matrice:
\[
d\mathcal{S}=
\begin{bmatrix}
  x_u & x_v \\
  y_u & y_v \\
  z_u & z_v 
\end{bmatrix}
\]
avendo tale matrice rango massimo, per definizione di superficie regolare, i vettori $\mathcal{S}_u$ e $\mathcal{S}_v$ sono dunque una base per $T_pS$. Poichè il loro prodotto vettoriale è non nullo, in quanto linearmente indipendenti, e ortogonale al piano da essi individuato, definiamo il versore normale alla superficie nel punto $p$ come (vedere figura \ref{fig:cp-111}):
\begin{equation}
  \label{eq:cp-111}
  \vec{n} = \frac{\mathcal{S}_u\wedge\mathcal{S}_v}{||\mathcal{S}_u\wedge\mathcal{S}_v||}=\frac{1}{\sqrt{\det(g)}}(\mathcal{S}_u\wedge\mathcal{S}_v),
\end{equation}
\begin{figure}[!hp]
  \tdplotsetmaincoords{60}{40}
  \begin{tikzpicture}[tdplot_main_coords,gray,thick]

    \coordinate (O) at (0,0,0);

    \tdplotsetcoord{P}{1.5}{90}{240};
    \tdplotsetcoord{P1}{2.5}{90}{105};
    \tdplotsetcoord{P2}{2.7}{90}{-30};
    \tdplotsetcoord{P3}{3.4}{90}{45}

    
    \draw[->,blue] (0,0,0)node[anchor=east]{$p$} -- 
    (2,0,0) node[anchor=south]{$\mathcal{S}_u$};
    \draw[->,blue] (0,0,0) -- (0,2,0) node[anchor=north]{$\mathcal{S}_v$};
    \draw[->,blue] (0,0,0) -- (0,0,2.5) node[anchor=north east]{$\vec{n}$};
    
    \draw (P) -- (P1) -- (P3) -- (P2) -- (P);
    
    \tdplotsetcoord{W}{6}{90}{281}
    \tdplotsetcoord{W1}{6.5}{90}{91}
    \tdplotsetcoord{W2}{6}{90}{318}
    \tdplotsetcoord{W3}{5.5}{90}{25}
    
    \draw (W) .. controls (-2,0,0) and (-2,1,0) .. (W1);
    \draw (W2) .. controls (3,0,0) and (3,0.5,0) .. (W3);
    \draw (W) to[out=60,in=60] (W2);
    \draw (W) to[out=240,in=240](W2);
    \draw (W1) to[out=190,in=190] (W3);
    \draw (W1) to[out=10,in=10] (W3);

  \end{tikzpicture}

  \caption{Piano tangente e versore normale alla superficie $S$ nel punto $p$.}
  \label{fig:cp-111}
\end{figure}
con $g$ \emph{prima forma fondamentale della parametrizzazione} $\mathcal{S}$, definita da:
\begin{definizione}
L'applicazione differenziabile $g:U\longrightarrow gl(2,\mathbb{R})$ è definita da $g(u)=(g_{ij}(u))$, con:
\begin{equation}
\label{eq:cp-112}
g_{ij}(u) = <\mathcal{S}_i,\mathcal{S}_j>\quad i,j=u,v.
\end{equation}
\end{definizione}
\begin{osservazione}
La prima forma fondamentale è simmetrica ,definita positiva e rappresenta il prodotto scalare in $T_pS$ nella base $\mathcal{S}_u$,$\mathcal{S}_v$. 
Si noti anche che :
\[
\det(g) = ||\mathcal{S}_u||^2||\mathcal{S}_v||^2-<\mathcal{S}_u,\mathcal{S}_v>^2 = ||\mathcal{S}_u\wedge\mathcal{S}_v||^2> 0.
\]
\end{osservazione}

\begin{osservazione}
Se $\tilde{u}$,$\tilde{v}$ sono un nuovo sisteme di coordinate su $U$, vale:
\[
\mathcal{S}_{\tilde{u}}\wedge\mathcal{S}_{\tilde{v}}=\det(\frac{\partial u,v}{\partial \tilde{u},\tilde{v}})\mathcal{S}_u\wedge\mathcal{S}_v
\]
quindi $\vec{n}$ resta invariato o cambia segno a seconda del segno del determinante della trasformazione parametrica.
\end{osservazione}
Il versore normale \eqref{eq:cp-111} individua un'applicazione differenziabile, che ad ogni punto della supeficie $S$ associa un punto della sfera unitaria $S^2$ in $\mathbb{R}^3$. Tale mappa, $N: S\longrightarrow S^2$, viene detta \emph{applicazione di Gauss}. Fissiamo, ora, un punto $p\in S$ , consideriamo un'applicazione  $F: T_pS\longrightarrow T_pS$ e vediamo come definirla. A tal proposito consideriamo una curva  $\alpha : (-\epsilon,\epsilon)\longrightarrow S$ sulla superficie $S$, tale che $\alpha(0)=p$. Per ogni $t\in (-\epsilon,\epsilon)$ vale che :
\[
<N(\alpha(t)),N(\alpha(t))> = 1,
\] 
deriviamo rispetto a $t$ in $t=0$:
\[
0 = <\frac{dN\alpha}{dt}(0),N(\alpha(0))>+<N(\alpha(0)),\frac{dN\alpha}{dt}(0)>.
\]
Poniamo per definizione $\tilde{F}(\alpha) = -\frac{dN\alpha}{dt}(0)$. Dalla relazione precedente segue che $<\tilde{F}(\alpha),N(p)> = 0$ e quindi $\tilde{F}(\alpha)\in T_pS$. Senza entrare nel dettaglio, si dimostra che $\tilde{F}(\alpha)$ dipende solo dalla mappa di Gauss e dal vettore $\alpha'(0)\in T_pS$. Definiamo dunque $F(\alpha'(0))=\tilde{F}(\alpha)$, in tal modo la nostra $F$ risulta essere ben definita, lineare e inolte per ogni vettore $\theta\in T_pS$ esiste una curva $\alpha$ tale che $\alpha'(0)=\theta$.

Introduciamo la \emph{seconda forma fondamentale}:
\begin{definizione}
Sia $b_{ij}=<\mathcal{S}_{ij},\vec{n}>$, per $i,j=u,v$. L'applicazione $b:\longrightarrow gl(2,\mathbb{R})$,tale che $b(u)=(b_{ij}(u))$ è detta seconda forma fondamentale della parametrizzazione $\mathcal{S}$; con
\[
\mathcal{S}_{ij}=\frac{\partial^2\mathcal{S}}{\partial i\partial j}\quad i,j = u,v,
\]
essendo $\mathcal{S}_{ij}=\mathcal{S}_{ji}$, per il Teorema dell'invertibilità dell'ordine di derivazione, essa è dunque simmetrica.
\end{definizione}
Facendo riferimento all'applicazione $F$ appena introdotta, possiamo riscrivere la seconda forma fondamentale:
\begin{equation}
\label{eq:cp-113}
b_{ij}=<F(\mathcal{S}_i),\mathcal{S}_j>,
\end{equation}
da ciò deduciamo che l'endomorfismo $F$ è autoaggiunto rispetto al prodotto scalare indotto su $\mathbb{R}^3$. Quindi, in una base ortonormale di $T_pS$, l'operatore $F$ è rappresentato da una matrice simmetrica ed è dunque diagonalizzabile con matrici ortogonali; e gli autovalori $k_1$, $k_2$ di $F$ si dicono \emph{curvaure principali} di $S$ in $p$. Rappresentado $F$ nella base $\mathcal{S}_u$, $\mathcal{S}_v$, abbiamo:
\[
F(\mathcal{S}_i) = f_i^1\mathcal{S}_u + f_i^2\mathcal{S}_v,\quad i=u,v,
\]
con $f_i^j$ i coefficenti della matrice dell'applicazione $F$. Sostituendo quest ultima equazione in \eqref{eq:cp-113} si dimostra che
\[
\det(F) = \frac{\det(b)}{\det(g)}=k_1k_2.
\]
Inoltre siccome la matrice $g$ è invertibile, i polinomi $\det(bg^{-1}-tI)$ e $\det(b-tg)$ differiscono solo per una costante moltiplicativa; quindi $k_1$, $k_2$ coincidono con le radici del polinomio $\det(b-tg)$.
In conclusione,si definiscono \emph{curvatura media} e \emph{curvatura gaussiana} rispettivamente,
\[
H=\frac{1}{2}(k_1+k_2)=\frac{1}{2}\trace(F),\quad K=k_1k_2=\det(F).
\]
\begin{osservazione}
La definizione precedente è usata in giometria. Esiste un'altra definizione di curvatura media per una superficie $S\subseteq\mathbb{R}^3$:
\begin{equation}
  \label{eq:cp-114}
  \tilde{H}(x):= -\Div(\vec{n}(x)), \quad x\in S\subseteq\mathbb{R}^3
\end{equation}
dove $\vec{n}(x)$ è il vettore unitàrio normale.
Si può provare che $\tilde{H}(x)=2H(x)$, quindi esse differiscono soltanto per una costante moltiplicativa. Senza entrare nel dettaglio, si definisce divergenza di superfice di un campo vettoriale $X$ ,$C^1$, sulla superficie $S$ in $\mathbb{R}^3$:
\[
(\Div_{S}X)(x) = \sum_{j=1}^{2}{\left<(D_{\tau^j} X)(x),\tau^j\right>},
\]
con $\tau^j$ base ortonormale di $TS$ e $D_\tau X$ derivata del campo nella direzione $\tau\in TS$, o meglio se prendiamo una curva $\alpha(t)$ su $S$ tale che $\alpha(0) = x$ e $\frac{d\alpha}{dt}(0)=\tau$ allora
\[
(D_\tau X)(x)=\frac{d}{dt}\left(X(\alpha(t))\right)\left|_{t=0}\right. =(\tau\cdot D)X=\sum_{j=1}^3\tau_j\frac{\partial}{\partial x_j}X.
\]
Prendiamo come campo $X$ il versone normale $\vec{n}(x)$ , ricordando la definizione della seconda forma fondamentale e della curvatura media otteniamo:
\[
H(x)= \sum_{j=1}^{2}{\left<F(\tau_j),\tau_j\right>} = -\sum_{j=1}^{2}{\left<(D_\tau \vec{n})(x),\tau^j\right>}=-(\Div_{S}\vec{n})(x).
\]
Ora riprendendo la definizione di divergenza di superficie ed operando alcune manipolazioni algebriche, senza entrare troppo nel dettaglio (per tutti i passaggi e le ipotesi da fare vedere **RIF**),possimao scrivere:
\[
\begin{aligned}
(\Div_{S}\vec{n})(x) &= \trace\left(\sum_{j=1}^2{(\tau_j\otimes\tau_j)}D\vec{n}(x)\right)=\\
&\trace\big((I-\vec{n}(x)\otimes\vec{n}(x))D\vec{n}(x)\big)=\Div\left(\vec{n}(x)\right).
\end{aligned}
\]
dove $p\otimes p=p^tp$ con $p$ vettore colonna in $\mathbb{R}^3$ e $(I-\vec{n}(x)\otimes\vec{n}(x))$ matrice di proiezione che associa a un punto $\zeta \in T_x\mathbb{R}^3=\mathbb{R}^3$ un punto su $T_xS$ con $x\in S$.
\end{osservazione}
%%%%%%%%%%%%%%%%%%%%%%%%%%%%%%%%%%%%%%%%%%%%%%%%%%%%
%
% Section 1.2
%
%%%%%%%%%%%%%%%%%%%%%%%%%%%%%%%%%%%%%%%%%%%%%%%%%%%%
\section{Moto per curvatura media \emph{Volume Preserving}}

Consideriamo superfici che si evolvono nel tempo. \\
Sia $\mathcal{S}(\tilde{p},t):U\times[0,T)\longrightarrow\mathbb{R}^3$, con $U\subseteq\mathbb{R}^2$, una famiglia di superfici regolari, dove $t$ parametrizza la famiglia e $\tilde{p}$ la superficie. Assumiamo che essa obedisce alla seguente PDE:
\begin{equation}
  \label{eq:cp121}
  \frac{\partial\mathcal{S}(\tilde{p},t)}{\partial t} = \alpha(\tilde{p},t)\vec{t}(\tilde{p},t) + \beta(\tilde{p},t)\vec{n}(\tilde{p},t)
\end{equation}
con $\mathcal{S}_0(\tilde{p})$ condizione iniziale. Indichiamo con $\vec{t}$ e $\vec{n}$ ripsettivamente, il versore tangente e il versore normale interno alla superficie; mentre $\alpha$ e $\beta$ sono le componenti tangenziale e normale della velocità con cui si deforma la superfcie. L'equazione scritta sopra risulta essere molto generale, infatti notiamo che per una qualsiasi velocità di deformazione $\vec{V}$ si ha $\alpha=<\vec{V},\vec{t}>$ e $\beta=<\vec{V},\vec{n}>$. Essendo interessati alla sola deformazione geometrica ed non a come cambia la parametrizzazione, Il PDE \eqref{eq:cp121} può essere semplificata applicando il seguente lemma:
\begin{lemma}
\label{lemm:cp-121}
Se $\beta$ non dipende dalla parametrizzazione, ma è una caratteristica geometrica intrinseca della superficie, tipo la curvatura, allora l'imagine di $\mathcal{S}(\tilde{p},t)$ che soddisfa \eqref{eq:cp121} è identica all'imagine della famiglia $\mathcal{S}(p,t)$ che soddisfa
\begin{equation}
  \label{eq:cp122}
  \frac{\partial\mathcal{S}(p,t)}{\partial t} =\beta(p,t)\vec{n}(p,t).
\end{equation}
\end{lemma}

Per la dimostrazione del lemma si veda **RIF**, soffermiamoci sulle conseguenze.
In poche parole, se la componente normale della velocità dipende soltato da caratteristiche geometriche della curva, l'evoluzione della famiglia è influenzata solo da quest'ultima componente; mentre la componente tangenziale influisce esclusivamente sulla parametrizzazione.
Detto ciò, un generico flusso geometrico di superfici ha la forma:
\[
\frac{\partial\mathcal{S}}{\partial t} = \beta(H,K)\vec{n},
\] 
cioè la velocità è una funzione delle due curvature principali (media e gaussiana). Il flusso che vogliamo studiare è quello per curvatura media, quindi l'equazione precedente diventa:
\begin{equation}
  \label{eq:cp-123}
  \frac{\partial\mathcal{S}}{\partial t} = H\vec{n}
\end{equation}
con $\mathcal{S}(p,0)=S_0$ dato iniziale.
\begin{osservazione}
Questo flusso è quello che in letteratura viene anche chiamato \emph{gradient discendent flow} , in quanto la superficie si evolve cercando di minimizzare l'area. 
\end{osservazione}
\begin{osservazione}
Come provato negli articoli **ref** , questo flusso deforma superfici convesse in un punto, ma superfici non covesse posso ,per esempio, dividersi prima di diventare convesse generando delle singolarità. Facciamo alcuni semplici esempi.
\end{osservazione}

\begin{esempio}[The Sphere]
Vogliamo calcolarci la soluzione esatta nel caso della sfera che si evolve secondo la \eqref{eq:cp-123}. Sia $\mathcal{S}(p,t) = S^2(R(t))$, cioè una famiglia di sfere in $\mathbb{R}^3$ di raggio $R(t)$ e centro l'origine, e $\mathcal{S}(p,0) = S^2(R_0)$ il nostro dato iniziale. Calcoliamoci la curvatura media, utilizzando l'espressione \eqref{eq:cp-114} e ricordando che la normale interna alla sfera è $\vec{n}(x(t))=-\frac{x(t)}{|x(t)|}$ con $x(t)\in S^2(R(t))$:
\[
H(x(t)) =-\Div(\vec{n}(x(t))=\Div\left(\frac{x(t)}{|x(t)|}\right)=\frac{2}{R(t)}.   
\]
Allora la nostra equazione \eqref{eq:cp-123} diventa:
\[
\overset{.}{R}(t)=-\frac{2}{R(t)}, \text{ con } R(0) = R_0,
\]
risolvendo otteniamo:
\[
R(t)=\sqrt{R_0^2-4t}.
\]
Notiamo che:
\[
R(\hat{t})=0 \Longleftrightarrow \hat{t}=\frac{R_0^2}{4}.
\]
Quindi la sfera si ridurrà ad un punto in un tempo finito $\hat{t}$, che è chiamato tempo di collasso (figura \ref{fig:cp-121}).

\begin{figure}[!hp]
  \begin{center}
  \tdplotsetmaincoords{60}{40}
  \begin{tikzpicture}[tdplot_main_coords,gray,thick]
   
    \coordinate (O) at (0,0,0);
       
    \tdplotsetcoord{P}{2.0}{90}{45};
    \tdplotsetcoord{P2}{5.5}{90}{45};
    \tdplotsetcoord{P3}{5.5}{90}{40};
    
    \draw [->,blue!52](P) to[out=50,in=130](P2);
    \node[shape=circle,draw=blue!50,fill=blue!12,inner sep=0pt,minimum size=1mm]
    (origin) at (O) {};
    \node [blue,left] at (origin.south) {$O$};
    \node[shape=circle,draw=gray,fill=gray,inner sep=0pt,minimum size=2mm]
    (end) at (P3) {};
    \node [blue,below] at (end.south) {$O$};

    \draw  [tdplot_main_coords,dashed] (O) ellipse (42pt and 20pt);
    \tdplotsetthetaplanecoords{40};
    \draw [thick,tdplot_rotated_coords](1.5,0,0) arc (0:360:1.5);
    
        
  \end{tikzpicture}
  \end{center}
  \caption{Evoluzione della sfera di raggio $R$, per curvatura media}
  \label{fig:cp-121}
\end{figure}
\end{esempio}

\begin{esempio}[The Dumbbell]
Il manubrio(dumbbell) può essere considerato come due spfere di equal raggio, connesse da un cilindro. Assumendo che le due sfere sono molto grandi ed il cilindro lungo e fino; l'evoluzione per curvatura media mi fa collassare velocemente il cilindro in una linea e le due sfere si evolvono più lentamente sempre in sfere più piccole. Dopo un certo tempo il manubrio si divide in due sfere disconnesse, in questa situazione si è verificato un cambiamento della topologia della superficie e quindi non si può più parlare di vettore normale e di curvatura. A questo punto le due sfere si evolvono separatamente fin a collassare in due punti (ved figura \ref{fig:cp-122}). 

\begin{figure}[!hptb]
\begin{center}
 \tdplotsetmaincoords{60}{40}
 \begin{tikzpicture}[tdplot_main_coords,gray,thick]
   
   \coordinate (O) at (0,0,0);
      
   \tdplotsetcoord{P}{3.5}{90}{5};
   \tdplotsetcoord{P2}{5.5}{90}{-30};
   \tdplotsetcoord{P3}{7.5}{90}{40};
   
   \draw [->,blue!52](P) to[out=190,in=150](P2);
   \node[shape=circle,draw=blue!50,fill=blue!12,inner sep=0pt,minimum size=1mm]
   (origin) at (O) {};
   \node [blue,below] at (origin.south) {$O$};
   \node[shape=circle,draw=blue!50,fill=blue!12,inner sep=0pt,minimum size=1mm]
   (end) at (P3) {};
   \node [blue,below] at (end.south) {$P$};

   \draw  [tdplot_main_coords,dashed] (O) ellipse (42pt and 20pt);
   \tdplotsetthetaplanecoords{40};
   \draw [thick,tdplot_rotated_coords](1.5,0,0) arc (0:360:1.5);

   \draw  [tdplot_main_coords,dashed] (P3) ellipse (42pt and 20pt);
   \tdplotsetthetaplanecoords{40};
   \draw [tdplot_rotated_coords] (P3) circle (1.5);
  
   \draw (1.8,0,0.2) -- (5.4,3.0,0.2);
   \draw (1.9,0,1.2) -- (5.5,2.9,1.2);
        
 \end{tikzpicture}


 \tdplotsetmaincoords{60}{40}
 \begin{tikzpicture}[tdplot_main_coords,gray,thick]
   
   \coordinate (O) at (0,0,0);
      
   \tdplotsetcoord{P}{2.0}{90}{10};
   \tdplotsetcoord{P2}{3.5}{90}{-30};
   \tdplotsetcoord{P3}{7.5}{90}{40};
   
   \draw [->,blue!52](P) to[out=190,in=150](P2);
   \node[shape=circle,draw=blue!50,fill=blue!12,inner sep=0pt,minimum size=1mm]
   (origin) at (O) {};
   \node [blue,below] at (origin.south) {$O$};
   \node[shape=circle,draw=blue!50,fill=blue!12,inner sep=0pt,minimum size=1mm]
   (end) at (P3) {};
   \node [blue,below] at (end.south) {$P$};

   \draw  [tdplot_main_coords,dashed] (O) ellipse (22pt and 5pt);
   \tdplotsetthetaplanecoords{40};
   \draw [tdplot_rotated_coords](O) circle (0.8);

   \draw  [tdplot_main_coords,dashed] (P3) ellipse (22pt and 5pt);
   \tdplotsetthetaplanecoords{40};
   \draw [tdplot_rotated_coords] (P3) circle (0.8);
  
   \draw (1.0,0.0,0.4) -- (6.2,3.0,1.0);
        
 \end{tikzpicture}

 \tdplotsetmaincoords{60}{40}
 \begin{tikzpicture}[tdplot_main_coords,gray,thick]
   
   \coordinate (O) at (0,0,0);
      
   \tdplotsetcoord{P}{2.0}{90}{45};
   \tdplotsetcoord{P2}{4.5}{90}{45};
   \tdplotsetcoord{P3}{3.0}{90}{40};
   \tdplotsetcoord{P4}{6.5}{90}{45};
   \tdplotsetcoord{P5}{6.7}{90}{40};
   \tdplotsetcoord{P6}{8.0}{90}{40};

   \draw [->,blue!52](P2) to[out=50,in=130](P4);
   \node[shape=circle,draw=blue!50,fill=blue!12,inner sep=0pt,minimum size=1mm]
   (origin) at (O) {};
   \node [blue,below] at (origin.south) {$O$};
   \node[shape=circle,draw=blue!50,fill=blue!12,inner sep=0pt,minimum size=1mm]
   (end) at (P3) {};
   \node [blue,below] at (end.south) {$P$};
   \node[shape=circle,draw=gray,fill=gray,inner sep=0pt,minimum size=2mm]
   (end) at (P5) {};
   \node [blue,below] at (end.south) {$0$};
   \node[shape=circle,draw=gray,fill=gray,inner sep=0pt,minimum size=2mm]
   (end) at (P6) {};
   \node [blue,below] at (end.south) {$P$};

   \draw  [tdplot_main_coords,dashed] (O) ellipse (13pt and 5pt);
   \tdplotsetthetaplanecoords{40};
   \draw [tdplot_rotated_coords](O) circle (0.5);

   \draw  [tdplot_main_coords,dashed] (P3) ellipse (13pt and 5pt);
   \tdplotsetthetaplanecoords{40};
   \draw [tdplot_rotated_coords] (P3) circle (0.5);
      
 \end{tikzpicture}
\end{center}
  \caption{Evoluzione del dumbbell  per curvatura media}
  \label{fig:cp-122}
\end{figure}
\end{esempio}
Questi due esempi ci confermano quanto detto sopra, cioè superfici convesse collassano in un punto e superfici non convesse posso generare delle discontinuità, vedimao come aggirare questi due problemi. 

Partiamo dal primo. Quello che vogliamo fara è normalizzare l'equazione \eqref{eq:cp-123} così che il volume viene preservato. Tale processo di normalizzazione, viene raggiunto cambiando la scala temporale, passando da $t$ a $\tau$, in modo tale che la nosta superficie diventa:
\[
\mathcal{\tilde{S}}(\tau):=\psi(t)\mathcal{S}(t),
\]
con $\psi(t)$ rappresentante il fattore di normalizzazione (\emph{time-scaling}) dato da:
\[
\psi^n(t) = \frac{\partial \tau}{\partial t},
\]
con $n$ tale che $\psi^{-n+1}(t)H=\tilde{H}$. Uno dei vantaggi di questa normalizzazione è che le due superfici hanno la stesse proprietà geometriche e quindi il loro comportamento è identico. Il nuovo tempo $\tau$ deve essere scelto in modo da ottenere $V_{\tau}\equiv 0$. Ricordando che la nostra superficie $\mathcal{S}(t)$ si evolve secondo l'equazione \eqref{eq:cp-123}, allora otteniamo la seguente evoluzione per $\mathcal{\tilde{S}}(\tau)$ : 
\[
\begin{aligned}
  &\frac{\partial\mathcal{\tilde{S}}(\tau)}{\partial\tau} =  \frac{\partial\mathcal{\tilde{S}}(t(\tau))}{\partial\tau} = \frac{\partial t}{\partial\tau}\frac{\partial\mathcal{\tilde{S}}(t(\tau))}{\partial t} =\\
  &=\psi^{-n}\left(\psi_t\mathcal{S}(t)+\psi(t)\mathcal{S}_t(t)\right)=\psi^{-n}\psi_t\mathcal{S}(t)+\psi^{-n+1}H\vec{n}=\\
  &=\psi^{-n}\psi_t\mathcal{S}(t)+\tilde{H}\vec{n}=\psi^{-n-1}\psi_t\mathcal{\tilde{S}}(\tau)+\tilde{H}\vec{n}.
\end{aligned}
\]

Poichè la geometria dell'evolulzione è influenzata solo dalla componente normale della velocità, utilizzando il lemma \ref{lemm:cp-121} otteniamo:
\[
\frac{\partial\mathcal{\tilde{S}}(\tau)}{\partial\tau}=\psi^{-n-1}\psi_t\left<\mathcal{\tilde{S}}(\tau),\vec{n}\right>+\tilde{H}\vec{n}.
\]
Definiamo la funzione supporto $\rho$ come:
\[
\rho:=-\left<\mathcal{S},\vec{n}\right>,
\]
allora
\[
\frac{\partial\mathcal{\tilde{S}}(\tau)}{\partial\tau}=-\psi^{-n-1}\psi_t\tilde{\rho}+\tilde{H}\vec{n}.
\]
Rricordando che, se una superficie evolve secondo un'equazione del tipo
\[
\frac{\partial \mathcal{S}}{\partial t}=\beta\vec{n},
\]
allora il tasso di cambiamento del volume è dato da
\begin{equation}
\label{eq:cp-124}
\frac{d V}{d t}=-\iint\beta d\mu
\end{equation}
con $d\mu$ elemento infinitesimo di superficie, vedere osservazione \ref{oss:cp-122}.
 Applicando la \eqref{eq:cp-124} giungiamo a:
\[
\tilde{V}_{\tau}=\iint(\psi^{-n-1}\psi_t\tilde{\rho}-\tilde{H})d\tilde{\mu},
\]
se noi richiediamo che $\tilde{V}_{\tau}\equiv 0$, allora:
\[
\psi^{-n-1}\psi_t\iint\tilde{\rho}d\mu = \iint\tilde{H}d\tilde{\mu}.
\]
 Tenedo presente che il volume di una superficie è pari a :
\begin{equation}
  \label{eq:cp-125}
V=\frac{1}{3}\iint-\left<\mathcal{S},\vec{n}\right>d\mu,
\end{equation}
in quanto $\vec{n}$ indica la normale interna (ved oss \ref{oss:cp-121}), e che rimane costante nel tempo (cosi da noi richiesto) possiamo scrivere:
\[
V(0)=\tilde{V}(\tau)=\frac{1}{3}\iint\tilde{\rho}d\tilde{\mu};
\]
mettendo insieme questa con l'espressione precedente otteniamo:
\[
\psi^{-n-1}\psi_t=\frac{\iint\tilde{H}d\tilde{\mu}}{3V_0},
\]
e finalmente l'equazione \emph{volume preserving} cercata:
\begin{equation}
\label{eq:cp-126}
\frac{\partial\mathcal{\tilde{S}}(\tau)}{\partial\tau}=\left(\tilde{H}-\frac{\tilde{\rho}\iint\tilde{H}d\tilde{\mu}}{3V_0}\right).
\end{equation}
\begin{osservazione}[Calcolo del Volume]
\label{oss:cp-121}
Vediamo da dove deriva l'espressione \eqref{eq:cp-125}. Prendiamo un insieme $\Omega\subset\mathbb{R}^3$ limitato, la cui frontiera è costituita da una superficie regolare e orientabile, ed un campo vettoriale $F$ regolare in esso, allora vale il \emph{Teorema della Divergenza} cioè:
\[
\int_{\Omega}\Div{F(x)}dV =\int_{\partial\Omega}\left<F(x),\vec{n}(x)\right>d\mu
\]
dove $\vec{n}(x)$,in tal caso, indica la normale esterna e $dV$,$d\mu$ rispettivamente elemento di volume ed elemento di superficie. Prendiamo come campo vettoriale $F(x)=\frac{1}{3}x$, quindi il volume di $\Omega$ è pari a:
\[
vol(\Omega)=\int_{\Omega}1dV=\int_{\Omega}\Div{(F(x))}dV=\frac{1}{3}\int_{\partial\Omega}\left<x,\vec{n}\right>d\mu,
\]
essendo l'ultimo integrale fatto su $\partial\Omega$, il vettore $\vec{x}$ rappresenta un punto della supeficie, quindi equivale a \eqref{eq:cp-125}.
\end{osservazione}

\begin{osservazione}[Tasso di variazione del Volume]
\label{oss:cp-122}
Consideriamo un insieme $\Omega(t)\subset\mathbb{R}^3$ \emph{time-dependent}, la cui frontiera è costituita da una superficie regolare e orientabile, ed un campo vettoriale $F(t,\vec{x})$ regolare a valori reali. Calcoliamoci il tasso di cambiamento dell'integrale:
\[
\begin{aligned}
\frac{d}{dt}&\int_{\Omega(t)}F(t,x)dV(t) = \int_{\Omega(t)}\frac{d}{dt}\left(F(t,\vec{x})dx(t)dy(t)dz(t)\right)=\\
 =&\int_{\Omega(t)}\left(\frac{d}{dt}F + F\frac{d}{dt}(dx_1)dx_2dx_3 + Fdx_1\frac{d}{dt}(dx_2)dx_3 + Fdx_1dx_2\frac{d}{dt}(dx_3)\right)=\\
=&\int_{\Omega(t)}\left(\frac{d}{dt}F + F\frac{\partial v_1}{\partial x_1}+ F\frac{\partial v_2}{\partial x_2} + F\frac{\partial v_3}{\partial x_3}\right)dx_1dx_2dx_3\\
=& \int_{\Omega(t)}\left(\frac{d}{dt}F + F\Div{\vec{v}}\right)dV = \int_{\Omega(t)}\left(\frac{\partial F}{\partial t} + \Div{(F\vec{v})}\right)dV.
\end{aligned}
\]
Applichiamo il \emph{Teorema della divergenza} all'ultima espressione
\[
\frac{d}{dt}\int_{\Omega(t)}F(t,x)dV(t)=\int_{\Omega(t)}\frac{\partial F}{\partial t}dV + \int_{\partial \Omega(t)}F\left<\vec{v},\vec{n}\right>d\mu.
\]
Per ottenere il volume nel primo integrale dobbiamo scegliere $F(t,x)\equiv 1$ e quindi otteniamo la \eqref{eq:cp-124}, eccenzion fatta per il segno in quanto in questo caso $\vec{n}$ rappresenta la normale esterna.
\end{osservazione}
%%%%%%%%%%%%%%%%%%%%%%%%%%%%%%%%%
%
%
% Section 1.3 
%
%
%%%%%%%%%%%%%%%%%%%%%%%%%%%%%%%%%%
\section{Approccio \emph{level-set}}  

Nella sezione precedente abbiamo visto come risolvere il problema del collasso, cioè in un tempo finito il volume diventa nullo. Tuttavia quando si implementa numericamente tale flusso (scopo finale di questo lavoro !!), ci si imbatte in altri problemi :
\begin{itemize}

  \item Il nostro algoritmo deve essere robusto in quanto deve approssimare un flusso continuo. Una semplice approssimazione Lagrangiana basata sulle particelle che si muovono lungo la superficie richiede un passo temporale impraticabile, poichè le particelle possono trovarsi molto vicine o molto lontane durante la deformazione.

   \item Lo svilupparsi di singolarità, è un altro problema che dobbiamo tener presente. Quindi abbiamo bisogno di uno schema che trovi ,in tal caso, la corretta soluzione debole.
   
   \item Cambiamenti topologici possono verificarsi (fusione o divisione) e tener traccia di questi con tracciamenti di particelle è un compito alquano arduo.

\end{itemize}

Tutti questi problemi ha portato allo svilupparsi ti tecniche \emph{level-set}. Introduciamo alcuni concetti base e riformuliamo la nostra equazione con un \emph{approccio level-set}.%
\footnote{Per maggiori chiamrimenti o approfondimente consultare \emph{Surface Evolution Equations-- a level set method --,Giga}.}    %




