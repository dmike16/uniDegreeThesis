\chapter{Evoluzione Geometrica Per Curvatura Media}
'WIP'
%%%%%%%%%%%%%%%%%%%%%%%%%%%%%%%%%%%%%%%%%%%%%%%%%%%%
%
% Section 1.1
%
%%%%%%%%%%%%%%%%%%%%%%%%%%%%%%%%%%%%%%%%%%%%%%%%%%%%
\section{Qualcosa sulle Superfici in $\mathbb{R}^3$}

Consideriamo un sottoinsieme $S\subseteq\mathbb{R}^3$, diremo che è una superficie regolare se per ogni punto $p\in S$ esiste un intorno $V$ di $p$ in $\mathbb{R}^3$, un aperto $U\subset\mathbb{R}^2$ e un'applicazione differenziabile $\mathcal{S}:U\longrightarrow V\cap S$ tale che:
\begin{enumerate}
  \item $\mathcal{S}$ è un omeomorfismo,
  \item il differenziale $d\mathcal{S}$ ha rango massimo in ogni punto di $U$.
\end{enumerate}
L'applicazione $\mathcal{S}$ si dice parametrizzazione regolare in un intorno del punto $p$. Mentre l'insieme $V\cap S=\mathcal{S}(U)$ si dice porzione di superficie parametrica regolare. La condizione di regolarità, ci esclude superfici che intersecano se stesse.

Siano $u$ e $v$ coordinate su $U\subset\mathbb{R}^2$ , $S = \mathcal{S}(U)$ e $p=\mathcal{S}(u_0,v_0)$; si definisce spazio tangente ad $S$ nel punto $p$, e si indica con $T_pS$, l'immgine di $\mathbb{R}^2$ tramite l'applicazione $d\mathcal{S}_{(u_0,v_0)}$:
\[
(a,b)\in\mathbb{R}^2\quad d\mathcal{S}_{(u_0,v_0)}(a,b)= 
\frac{d}{dt}\mathcal{S}(u_0+at,v_0+bt)|_{t=0}=
\]
\[
=a\mathcal{S}_u(u_0,v_0)+b\mathcal{S}_v(u_0,v_0),
\]
con $\mathcal{S}_u(u_0,v_0)$ e $\mathcal{S}_v(u_0,v_0)$ colonne della matrice:
\[
d\mathcal{S}=
\begin{bmatrix}
  x_u & x_v \\
  y_u & y_v \\
  z_u & z_v 
\end{bmatrix}
\]
avendo tale matrice rango massimo, per definizione di superficie regolare, i vettori $\mathcal{S}_u$ e $\mathcal{S}_v$ sono dunque una base per $T_pS$. Poichè il loro prodotto vettoriale è non nullo, in quanto linearmente indipendenti, e ortogonale al piano da essi individuato, definiamo il versore normale alla superficie nel punto $p$ come (vedere figura \ref{fig:cp-11}):
\begin{equation}
  \label{eq:cp-10}
  \vec{n} = \frac{\mathcal{S}_u\wedge\mathcal{S}_v}{||\mathcal{S}_u\wedge\mathcal{S}_v||}=\frac{1}{\sqrt{\det(g)}}(\mathcal{S}_u\wedge\mathcal{S}_v),
\end{equation}
\begin{figure}[!hp]
  \tdplotsetmaincoords{60}{40}
  \begin{tikzpicture}[tdplot_main_coords]

    \coordinate (O) at (0,0,0);

    \tdplotsetcoord{P}{1.5}{90}{240};
    \tdplotsetcoord{P1}{2.5}{90}{105};
    \tdplotsetcoord{P2}{2.7}{90}{-30};
    \tdplotsetcoord{P3}{3.4}{90}{45}

    
    \draw[thick,->,blue] (0,0,0)node[anchor=east]{$p$} -- 
    (2,0,0) node[anchor=south]{$\mathcal{S}_u$};
    \draw[thick,->,blue] (0,0,0) -- (0,2,0) node[anchor=north]{$\mathcal{S}_v$};
    \draw[thick,->,blue] (0,0,0) -- (0,0,2.5) node[anchor=north east]{$\vec{n}$};
    
    \draw (P) -- (P1) -- (P3) -- (P2) -- (P);
    
    \tdplotsetcoord{W}{6}{90}{281}
    \tdplotsetcoord{W1}{6.5}{90}{91}
    \tdplotsetcoord{W2}{6}{90}{318}
    \tdplotsetcoord{W3}{5.5}{90}{25}
    
    \draw (W) .. controls (-2,0,0) and (-2,1,0) .. (W1);
    \draw (W2) .. controls (3,0,0) and (3,0.5,0) .. (W3);
    \draw (W) to[out=60,in=60] (W2);
    \draw (W) to[out=240,in=240](W2);
    \draw (W1) to[out=190,in=190] (W3);
    \draw (W1) to[out=10,in=10] (W3);

  \end{tikzpicture}

  \caption{Piano tangente e versore normale alla superficie $S$ nel punto $p$.}
  \label{fig:cp-11}
\end{figure}
con $g$ \emph{prima forma fondamentale della parametrizzazione} $\mathcal{S}$, definita da:
\begin{definizione}
L'applicazione differenziabile $g:U\longrightarrow gl(2,\mathbb{R})$ è definita da $g(u)=(g_{ij}(u))$, con:
\begin{equation}
\label{eq:cp-11}
g_{ij}(u) = <\mathcal{S}_i,\mathcal{S}_j>\quad i,j=u,v.
\end{equation}
\end{definizione}
\begin{osservazione}
La prima forma fondamentale è simmetrica ,definita positiva e rappresenta il prodotto scalare in $T_pS$ nella base $\mathcal{S}_u$,$\mathcal{S}_v$. 
Si noti anche che :
\[
\det(g) = ||\mathcal{S}_u||^2||\mathcal{S}_v||^2-<\mathcal{S}_u,\mathcal{S}_v>^2 = ||\mathcal{S}_u\wedge\mathcal{S}_v||^2> 0.
\]
\end{osservazione}

\begin{osservazione}
Se $\tilde{u}$,$\tilde{v}$ sono un nuovo sisteme di coordinate su $U$, vale:
\[
\mathcal{S}_{\tilde{u}}\wedge\mathcal{S}_{\tilde{v}}=\det(\frac{\partial u,v}{\partial \tilde{u},\tilde{v}})\mathcal{S}_u\wedge\mathcal{S}_v
\]
quindi $\vec{n}$ resta invariato o cambia segno a seconda del segno del determinante della trasformazione parametrica.
\end{osservazione}
Il versore normale \eqref{eq:cp-10} individua un'applicazione differenziabile, che ad ogni punto della supeficie $S$ associa un punto della sfera unitaria $S^2$ in $\mathbb{R}^3$. Tale mappa, $N: S\longrightarrow S^2$, viene detta \emph{applicazione di Gauss}. Fissiamo, ora, un punto $p\in S$ , consideriamo un'applicazione  $F: T_pS\longrightarrow T_pS$ e vediamo come definirla. A tal proposito consideriamo una curva  $\alpha : (-\epsilon,\epsilon)\longrightarrow S$ sulla superficie $S$, tale che $\alpha(0)=p$. Per ogni $t\in (-\epsilon,\epsilon)$ vale che :
\[
<N(\alpha(t)),N(\alpha(t))> = 1,
\] 
deriviamo rispetto a $t$ in $t=0$:
\[
0 = <\frac{dN\alpha}{dt}(0),N(\alpha(0))>+<N(\alpha(0)),\frac{dN\alpha}{dt}(0)>.
\]
Poniamo per definizione $\tilde{F}(\alpha) = -\frac{dN\alpha}{dt}(0)$. Dalla relazione precedente segue che $<\tilde{F}(\alpha),N(p)> = 0$ e quindi $\tilde{F}(\alpha)\in T_pS$. Senza entrare nel dettaglio, si dimostra che $\tilde{F}(\alpha)$ dipende solo dalla mappa di Gauss e dal vettore $\alpha'(0)\in T_pS$. Definiamo dunque $F(\alpha'(0))=\tilde{F}(\alpha)$, in tal modo la nostra $F$ risulta essere ben definita, lineare e inolte per ogni vettore $\theta\in T_pS$ esiste una curva $\alpha$ tale che $\alpha'(0)=\theta$.

Introduciamo la \emph{seconda forma fondamentale}:
\begin{definizione}
Sia $b_{ij}=<\mathcal{S}_{ij},\vec{n}>$, per $i,j=u,v$. L'applicazione $b:\longrightarrow gl(2,\mathbb{R})$,tale che $b(u)=(b_{ij}(u))$ è detta seconda forma fondamentale della parametrizzazione $\mathcal{S}$; con
\[
\mathcal{S}_{ij}=\frac{\partial^2\mathcal{S}}{\partial i\partial j}\quad i,j = u,v,
\]
essendo $\mathcal{S}_{ij}=\mathcal{S}_{ji}$, per il Teorema dell'invertibilità dell'ordine di derivazione, essa è dunque simmetrica.
\end{definizione}
Facendo riferimento all'applicazione $F$ appena introdotta, possiamo riscrivere la seconda forma fondamentale:
\begin{equation}
\label{eq:cp-12}
b_{ij}=<F(\mathcal{S}_i),\mathcal{S}_j>,
\end{equation}
da ciò deduciamo che l'endomorfismo $F$ è autoaggiunto rispetto al prodotto scalare indotto su $\mathbb{R}^3$. Quindi, in una base ortonormale di $T_pS$, l'operatore $F$ è rappresentato da una matrice simmetrica ed è dunque diagonalizzabile con matrici ortogonali; e gli autovalori $k_1$, $k_2$ di $F$ si dicono \emph{curvaure principali} di $S$ in $p$. Rappresentado $F$ nella base $\mathcal{S}_u$, $\mathcal{S}_v$, abbiamo:
\[
F(\mathcal{S}_i) = f_i^1\mathcal{S}_u + f_i^2\mathcal{S}_v,\quad i=u,v,
\]
con $f_i^j$ i coefficenti della matrice dell'applicazione $F$. Sostituendo quest ultima equazione in \eqref{eq:cp-12} si dimostra che
\[
\det(F) = \frac{\det(b)}{\det(g)}=k_1k_2.
\]
Inoltre siccome la matrice $g$ è invertibile, i polinomi $\det(bg^{-1}-tI)$ e $\det(b-tg)$ differiscono solo per una costante moltiplicativa; quindi $k_1$, $k_2$ coincidono con le radici del polinomio $\det(b-tg)$.
In conclusione,si definiscono \emph{curvatura media} e \emph{curvatura gaussiana} rispettivamente,
\[
H=\frac{1}{2}(k_1+k_2)=\frac{1}{2}\trace(F),\quad K=k_1k_2=\det(F).
\]
%%%%%%%%%%%%%%%%%%%%%%%%%%%%%%%%%%%%%%%%%%%%%%%%%%%%
%
% Section 1.2
%
%%%%%%%%%%%%%%%%%%%%%%%%%%%%%%%%%%%%%%%%%%%%%%%%%%%%
\section{Moto per curvatura media \emph{Volum Preserving}}
