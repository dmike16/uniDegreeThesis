\chapter{Evoluzione Geometrica Per Curvatura Media}
'WIP'
%%%%%%%%%%%%%%%%%%%%%%%%%%%%%%%%%%%%%%%%%%%%%%%%%%%%
%
% Section 1.1
%
%%%%%%%%%%%%%%%%%%%%%%%%%%%%%%%%%%%%%%%%%%%%%%%%%%%%
\section{Qualcosa sulle Superfici}

Consideriamo un sottoinsieme $S\subseteq\mathbb{R}^3$, diremo che è una superficie regolare se per ogni punto $p\in S$ esiste un intorno $V$ di $p$ in $\mathbb{R}^3$, un aperto $U\subset\mathbb{R}^2$ e un'applicazione differenziabile $\mathcal{S}:U\longrightarrow V\cap S$,$(u,v)\longrightarrow\mathcal{S}=(x(u,v),y(u,v),z(u,v))$, tale che:
\begin{enumerate}
  \item $\mathcal{S}$ è un omeomorfismo,
  \item il differenziale $d\mathcal{S}$ ha rango massimo in ogni punto di $U$.
\end{enumerate}
L'applicazione $\mathcal{S}$ si dice parametrizzazione regolare in un intorno del punto $p$. Mentre l'insieme $V\cap S=\mathcal{S}(U)$ si dice porzione di superficie parametrica regolare. La condizione di regolarità, ci esclude superfici che intersecano se stesse.

Siano $u$ e $v$ coordinate su $U\subset\mathbb{R}^2$ , $S = \mathcal{S}(U)$ e $p=\mathcal{S}(u_0,v_0)$; si definisce spazio tangente ad $S$ nel punto $p$, e si indica con $T_pS$, l'immagine di $\mathbb{R}^2$ tramite l'applicazione $d\mathcal{S}_{(u_0,v_0)}$:
\[
(a,b)\in\mathbb{R}^2\quad d\mathcal{S}_{(u_0,v_0)}(a,b)= 
\frac{d}{dt}\mathcal{S}(u_0+at,v_0+bt)|_{t=0}=
\]
\[
=a\mathcal{S}_u(u_0,v_0)+b\mathcal{S}_v(u_0,v_0),
\]
con $\mathcal{S}_u(u_0,v_0)$ e $\mathcal{S}_v(u_0,v_0)$ colonne della matrice:
\[
d\mathcal{S}=
\begin{bmatrix}
  x_u & x_v \\
  y_u & y_v \\
  z_u & z_v 
\end{bmatrix}
\]
avendo tale matrice rango massimo, per definizione di superficie regolare, i vettori $\mathcal{S}_u$ e $\mathcal{S}_v$ sono dunque una base per $T_pS$. Poichè il loro prodotto vettoriale è non nullo, in quanto linearmente indipendenti, e ortogonale al piano da essi individuato, definiamo il versore normale alla superficie nel punto $p$ come (vedere figura \ref{fig:cp-111}):
\begin{equation}
  \label{eq:cp-111}
  \vec{n} = \frac{\mathcal{S}_u\wedge\mathcal{S}_v}{||\mathcal{S}_u\wedge\mathcal{S}_v||}=\frac{1}{\sqrt{\det(g)}}(\mathcal{S}_u\wedge\mathcal{S}_v),
\end{equation}
\begin{figure}[!hp]
  \tdplotsetmaincoords{60}{40}
  \begin{tikzpicture}[tdplot_main_coords,gray,thick]

    \coordinate (O) at (0,0,0);

    \tdplotsetcoord{P}{1.5}{90}{240};
    \tdplotsetcoord{P1}{2.5}{90}{105};
    \tdplotsetcoord{P2}{2.7}{90}{-30};
    \tdplotsetcoord{P3}{3.4}{90}{45}

    
    \draw[->,blue] (0,0,0)node[anchor=east]{$p$} -- 
    (2,0,0) node[anchor=south]{$\mathcal{S}_u$};
    \draw[->,blue] (0,0,0) -- (0,2,0) node[anchor=north]{$\mathcal{S}_v$};
    \draw[->,blue] (0,0,0) -- (0,0,2.5) node[anchor=north east]{$\vec{n}$};
    
    \draw (P) -- (P1) -- (P3) -- (P2) -- (P);
    
    \tdplotsetcoord{W}{6}{90}{281}
    \tdplotsetcoord{W1}{6.5}{90}{91}
    \tdplotsetcoord{W2}{6}{90}{318}
    \tdplotsetcoord{W3}{5.5}{90}{25}
    
    \draw (W) .. controls (-2,0,0) and (-2,1,0) .. (W1);
    \draw (W2) .. controls (3,0,0) and (3,0.5,0) .. (W3);
    \draw (W) to[out=60,in=60] (W2);
    \draw (W) to[out=240,in=240](W2);
    \draw (W1) to[out=190,in=190] (W3);
    \draw (W1) to[out=10,in=10] (W3);

  \end{tikzpicture}

  \caption{Piano tangente e versore normale alla superficie $S$ nel punto $p$.}
  \label{fig:cp-111}
\end{figure}
con $g$ \emph{prima forma fondamentale della parametrizzazione} $\mathcal{S}$, definita da:
\begin{definizione}
L'applicazione differenziabile $g:U\longrightarrow gl(2,\mathbb{R})$, dove $gl(2,\mathbb{R})$ rappresenta il gruppo di tutte le matrici invertibili $2\times2$ con coefficenti reali,  è definita da $g(u)=(g_{ij}(u))$, con:
\begin{equation}
\label{eq:cp-112}
g_{ij}(u) = <\mathcal{S}_i,\mathcal{S}_j>\quad i,j=u,v.
\end{equation}
\end{definizione}
\begin{osservazione}
La prima forma fondamentale è simmetrica ,definita positiva e rappresenta il prodotto scalare in $T_pS$ nella base $\mathcal{S}_u$,$\mathcal{S}_v$. 
Si noti anche che :
\[
\det(g) = ||\mathcal{S}_u||^2||\mathcal{S}_v||^2-<\mathcal{S}_u,\mathcal{S}_v>^2 = ||\mathcal{S}_u\wedge\mathcal{S}_v||^2> 0.
\]
\end{osservazione}

\begin{osservazione}
Se $\tilde{u}$,$\tilde{v}$ sono un nuovo sistema di coordinate su $U$ con $u\equiv u(\tilde{u},\tilde{v})$ e $v\equiv v(\tilde{u},\tilde{v})$, vale:
\[
\mathcal{S}_{\tilde{u}}\wedge\mathcal{S}_{\tilde{v}}=\det(\frac{\partial u,v}{\partial \tilde{u},\tilde{v}})\mathcal{S}_u\wedge\mathcal{S}_v\,,\quad \det(\frac{\partial u,v}{\partial \tilde{u},\tilde{v}})=
\begin{vmatrix}
  \partial_{\tilde{u}}u & \partial_{\tilde{v}}u \\
  \partial_{\tilde{u}}v & \partial_{\tilde{v}}v
\end{vmatrix}
\]
quindi $\vec{n}$ resta invariato o cambia segno a seconda del segno del determinante dello jacobiano della trasformazione per il cambio di variabili.
\end{osservazione}
Il versore normale \eqref{eq:cp-111} individua un'applicazione differenziabile, che ad ogni punto della supeficie $S$ associa un punto della sfera unitaria $S^2$ in $\mathbb{R}^3$. Tale mappa, $N: S\longrightarrow S^2$, viene detta \emph{applicazione di Gauss}. Fissiamo, ora, un punto $p\in S$ , consideriamo un'applicazione  $F: T_pS\longrightarrow T_pS$ cha andremo a definire. A tal proposito consideriamo una curva  $\alpha : (-\epsilon,\epsilon)\longrightarrow S$ sulla superficie $S$, tale che $\alpha(0)=p$. Per ogni $t\in (-\epsilon,\epsilon)$ vale che :
\[
<N(\alpha(t)),N(\alpha(t))> = 1,
\] 
deriviamo rispetto a $t$ in $t=0$:
\[
0 = {<\frac{dN(\alpha(t))}{dt},N(\alpha(t))>}_{\big{|}_{t=0}}+{<N(\alpha(t)),\frac{dN(\alpha(t))}{dt}>}_{\big{|}_{t=0}}.
\]
Poniamo per definizione $\tilde{F}(\alpha) = -\frac{dN\alpha}{dt}(0)$. Dalla relazione precedente segue che $<\tilde{F}(\alpha),N(p)> = 0$ e quindi $\tilde{F}(\alpha)\in T_pS$. Senza entrare nel dettaglio, si dimostra che $\tilde{F}(\alpha)$ dipende solo dalla mappa di Gauss e dal vettore $\alpha'(0)\in T_pS$. Definiamo dunque $F(\alpha'(0))=\tilde{F}(\alpha)$, in tal modo la nostra $F$ risulta essere ben definita, lineare e inolte per ogni vettore $\theta\in T_pS$ esiste una curva $\alpha$ tale che $\alpha'(0)=\theta$.

Introduciamo la \emph{seconda forma fondamentale}:
\begin{definizione}
Sia $b_{ij}=<\mathcal{S}_{ij},\vec{n}>$, per $i,j=u,v$. L'applicazione $b:U\longrightarrow gl(2,\mathbb{R})$,tale che $b(u)=(b_{ij}(u))$ è detta seconda forma fondamentale della parametrizzazione $\mathcal{S}$; con
\[
\mathcal{S}_{ij}=\frac{\partial^2\mathcal{S}}{\partial i\partial j}\quad i,j = u,v,
\]
\end{definizione}
\begin{osservazione}
Essendo $\mathcal{S}_{ij}=\mathcal{S}_{ji}$, per il Teorema dell'invertibilità dell'ordine di derivazione, $b(u)$ è dunque simmetrica.
\end{osservazione}
Facendo riferimento all'applicazione $F$ appena introdotta, possiamo riscrivere la seconda forma fondamentale:
\begin{equation}
\label{eq:cp-113}
b_{ij}=<F(\mathcal{S}_i),\mathcal{S}_j>,
\end{equation}
da ciò deduciamo che l'endomorfismo $F$ è autoaggiunto rispetto al prodotto scalare indotto su $\mathbb{R}^3$. Quindi, in una base ortonormale di $T_pS$, l'operatore $F$ è rappresentato da una matrice simmetrica ed è dunque diagonalizzabile con matrici ortogonali; e gli autovalori $k_1$, $k_2$ di $F$ si dicono \emph{curvaure principali} di $S$ in $p$. Rappresentado $F$ nella base $\mathcal{S}_u$, $\mathcal{S}_v$, abbiamo:
\[
F(\mathcal{S}_i) = f_i^1\mathcal{S}_u + f_i^2\mathcal{S}_v,\quad i=u,v,
\]
con $f_i^j$ i coefficenti della matrice dell'applicazione $F$. Sostituendo quest ultima equazione in \eqref{eq:cp-113} si dimostra che
\[
\det(F) = \frac{\det(b)}{\det(g)}=k_1k_2.
\]
Inoltre siccome la matrice $g$ è invertibile, i polinomi $\det(bg^{-1}-tI)$ e $\det(b-tg)$ differiscono solo per una costante moltiplicativa; quindi $k_1$, $k_2$ coincidono con le radici del polinomio $\det(b-tg)$. \\
In conclusione, si definiscono \emph{curvatura media}\footnote{In alcuni testi è definita coma la media delle curvature principali, quindi differisce dalla nostra defnizione solo per un fattore moltiplicativo. \cite[vedi][cp 1.3]{giga:main}.} \,%
 e \emph{curvatura gaussiana} rispettivamente,
\begin{equation}
\label{eq:cp-111-add}
H=k_1+k_2=\trace(F),\quad K=k_1k_2=\det(F).
\end{equation}
\begin{osservazione}
La definizione precedente di curvatura media è equivalente alla seguente:
\begin{equation}
  \label{eq:cp-114}
  \tilde{H}(x):= -\Div(\vec{n}(x)), \quad x\in S\subseteq\mathbb{R}^3
\end{equation}
dove $\vec{n}(x)$ è il vettore unitàrio normale. Senza entrare nel dettaglio, si definisce divergenza di superfice di un campo vettoriale $X$ ,$C^1$, sulla superficie $S$ in $\mathbb{R}^3$:
\[
(\Div_{S}X)(x) = \sum_{j=1}^{2}{\left<(D_{\tau^j} X)(x),\tau^j\right>},
\]
con $\tau^j$ base ortonormale di $TS$ e $D_\tau X$ derivata del campo nella direzione $\tau\in TS$, o meglio se prendiamo una curva $\alpha(t)$ su $S$ tale che $\alpha(0) = x$ e $\frac{d\alpha}{dt}(0)=\tau$ allora
\[
(D_\tau X)(x)=\frac{d}{dt}\left(X(\alpha(t))\right)_{\big|_{t=0}} =(\tau\cdot D)X=\sum_{j=1}^3\tau_j\frac{\partial}{\partial x_j}X.
\]
Prendiamo come campo $X$ il versone normale $\vec{n}(x)$ , ricordando la definizione della seconda forma fondamentale \eqref{eq:cp-113} e della curvatura media  \eqref{eq:cp-111-add} otteniamo:
\[
H(x)= \sum_{j=1}^{2}{\left<F(\tau_j),\tau_j\right>} = -\sum_{j=1}^{2}{\left<(D_\tau \vec{n})(x),\tau^j\right>}=-(\Div_{S}\vec{n})(x).
\]
Ora riprendendo la definizione di divergenza di superficie ed operando alcune manipolazioni algebriche, senza entrare troppo nel dettaglio (per tutti i passaggi e le ipotesi da fare \cite[vedi][cp 1]{giga:main}), possimao scrivere:
\[
\begin{aligned}
(\Div_{S}\vec{n})(x) &= \trace\left(\sum_{j=1}^2{(\tau_j\otimes\tau_j)}D\vec{n}(x)\right)=\\
&\trace\big((I-\vec{n}(x)\otimes\vec{n}(x))D\vec{n}(x)\big)=\Div\left(\vec{n}(x)\right).
\end{aligned}
\]
dove $p\otimes p=p^tp$ con $p$ vettore colonna in $\mathbb{R}^3$ e $(I-\vec{n}(x)\otimes\vec{n}(x))$ matrice di proiezione che associa a un punto $\zeta \in T_x\mathbb{R}^3=\mathbb{R}^3$ un punto su $T_xS$ con $x\in S$.
\end{osservazione}
%%%%%%%%%%%%%%%%%%%%%%%%%%%%%%%%%%%%%%%%%%%%%%%%%%%%
%
% Section 1.2
%
%%%%%%%%%%%%%%%%%%%%%%%%%%%%%%%%%%%%%%%%%%%%%%%%%%%%
\section{Moto per curvatura media \emph{Volume Preserving}}

Consideriamo superfici che si evolvono nel tempo. \\
Sia $\mathcal{S}(\tilde{p},t):U\times[0,T)\longrightarrow\mathbb{R}^3$, con $U\subseteq\mathbb{R}^2$, una famiglia di superfici regolari, dove $t$ parametrizza la famiglia e $\tilde{p}$ la superficie. Assumiamo che essa obedisce alla seguente equazione alle derivate parziali:
\begin{equation}
  \label{eq:cp121}
  \frac{\partial\mathcal{S}(\tilde{p},t)}{\partial t} = \alpha(\tilde{p},t)\vec{t}(\tilde{p},t) + \beta(\tilde{p},t)\vec{n}(\tilde{p},t)
\end{equation}
con $\mathcal{S}_0(\tilde{p})$ condizione iniziale. Indichiamo con $\vec{t}$ e $\vec{n}$ ripsettivamente, il versore tangente e il versore normale interno alla superficie; mentre $\alpha$ e $\beta$ sono le componenti tangenziale e normale della velocità con cui si deforma la superfcie. L'equazione scritta sopra risulta essere molto generale, infatti notiamo che per una qualsiasi velocità di deformazione $\vec{V}$ si ha $\alpha=<\vec{V},\vec{t}>$ e $\beta=<\vec{V},\vec{n}>$. Essendo interessati alla sola deformazione geometrica e non a come cambia la parametrizzazione, l'equazione \eqref{eq:cp121} può essere semplificata applicando il seguente lemma:
\begin{lemma}
\label{lemm:cp-121}
Se $\beta$ non dipende dalla parametrizzazione, ma è una caratteristica geometrica intrinseca della superficie, tipo la curvatura, allora l'imagine di $\mathcal{S}(\tilde{p},t)$ che soddisfa \eqref{eq:cp121} è identica all'imagine della famiglia $\mathcal{S}(p,t)$ che soddisfa
\begin{equation}
  \label{eq:cp122}
  \frac{\partial\mathcal{S}(p,t)}{\partial t} =\beta(p,t)\vec{n}(p,t).
\end{equation}
\end{lemma}

Per la dimostrazione del lemma \cite[vedi][cp 2]{gui:sapiro}, soffermiamoci qui solo sulle conseguenze.
In poche parole, se la componente normale della velocità dipende soltato da caratteristiche geometriche della curva, l'evoluzione della famiglia è influenzata solo da quest'ultima componente; mentre la componente tangenziale influisce esclusivamente sulla parametrizzazione.
Consideriamo ora un flusso geometrico di superfici della forma:
\[
\frac{\partial\mathcal{S}}{\partial t} = \beta(H)\vec{n},
\] 
dove la velocità $\beta$ è una funzione della curvatura media. In particolare se $\beta(H)=H$ l'equazione precedente diventa:
\begin{equation}
  \label{eq:cp-123}
  \frac{\partial\mathcal{S}}{\partial t} = \vec{V} = H\vec{n}
\end{equation}
con $\mathcal{S}(p,0)=S_0$ dato iniziale.
\begin{osservazione}
Questo flusso è quello che in letteratura viene anche chiamato \emph{gradient discendent flow} , in quanto la superficie si evolve cercando di minimizzare l'area. 
\end{osservazione}
\begin{osservazione}
Come provato nell'articolo \cite[vedi][]{bi:chow} , questo flusso deforma superfici convesse in un punto, ma superfici non covesse posso, per esempio, dividersi prima di diventare convesse generando delle singolarità. Facciamo alcuni semplici esempi.
\end{osservazione}

\begin{esempio}[The Sphere]
Calcoliamo la soluzione esatta nel caso della sfera che si evolve secondo  \eqref{eq:cp-123}. Sia $\mathcal{S}(p,t) = S^2(R(t))$, cioè una famiglia di sfere in $\mathbb{R}^3$ di raggio $R(t)$ e centro l'origine, e $\mathcal{S}(p,0) = S^2(R_0)$ il nostro dato iniziale. Calcoliamoci la curvatura media, utilizzando l'espressione \eqref{eq:cp-114} e ricordando che la normale interna alla sfera è $\vec{n}(x(t))=-\frac{x(t)}{|x(t)|}$ con $x(t)\in S^2(R(t))$:
\[
H(x(t)) =-\Div(\vec{n}(x(t))=\Div\left(\frac{x(t)}{|x(t)|}\right)=\frac{2}{R(t)}.   
\]
Allora la nostra equazione \eqref{eq:cp-123} diventa:
\[
\overset{\displaystyle.}{R}(t)=-\frac{2}{R(t)}, \text{ con } R(0) = R_0,
\]
risolvendo otteniamo:
\[
R(t)=\sqrt{R_0^2-4t}.
\]
Notiamo che:
\[
R(\hat{t})=0 \Longleftrightarrow \hat{t}=\frac{R_0^2}{4}.
\]
Quindi la sfera si ridurrà ad un punto in un tempo finito $\hat{t}$, che è chiamato tempo di collasso (figura \ref{fig:cp-121}).

\begin{figure}[!htbp]
  \begin{center}
  \tdplotsetmaincoords{60}{40}
  \begin{tikzpicture}[tdplot_main_coords,gray,thick]
   
    \coordinate (O) at (0,0,0);
       
    \tdplotsetcoord{P}{2.0}{90}{45};
    \tdplotsetcoord{P2}{5.5}{90}{45};
    \tdplotsetcoord{P3}{5.5}{90}{40};
    
    \draw [->,blue!52](P) to[out=50,in=130](P2);
    \node[shape=circle,draw=blue!50,fill=blue!12,inner sep=0pt,minimum size=1mm]
    (origin) at (O) {};
    \node [blue,left] at (origin.south) {$O$};
    \node[shape=circle,draw=gray,fill=gray,inner sep=0pt,minimum size=2mm]
    (end) at (P3) {};
    \node [blue,below] at (end.south) {$O$};

    \draw  [tdplot_main_coords,dashed] (O) ellipse (42pt and 20pt);
    \tdplotsetthetaplanecoords{40};
    \draw [thick,tdplot_rotated_coords](1.5,0,0) arc (0:360:1.5);
    
        
  \end{tikzpicture}
  \end{center}
  \caption{Evoluzione della sfera di raggio $R$, per curvatura media}
  \label{fig:cp-121}
\end{figure}
\end{esempio}

\begin{esempio}[The Dumbbell]
Il manubrio(dumbbell) può essere considerato come due spfere di equal raggio, connesse da un cilindro. Assumendo che le due sfere sono molto grandi ed il cilindro lungo e fino; l'evoluzione per curvatura media fa collassare velocemente il cilindro in una linea e le due sfere si evolvono più lentamente sempre in sfere più piccole. Dopo un certo tempo il manubrio si divide in due sfere disconnesse, in questa situazione si è verificato un cambiamento della topologia della superficie e quindi non si può più parlare di vettore normale e di curvatura. A questo punto le due sfere si evolvono separatamente fin a collassare in due punti (ved figura \ref{fig:cp-122}). 

\begin{figure}[!hptb]
\begin{center}
 \tdplotsetmaincoords{60}{40}
 \begin{tikzpicture}[tdplot_main_coords,gray,thick]
   
   \coordinate (O) at (0,0,0);
      
   \tdplotsetcoord{P}{3.5}{90}{5};
   \tdplotsetcoord{P2}{5.5}{90}{-30};
   \tdplotsetcoord{P3}{7.5}{90}{40};
   
   \draw [->,blue!52](P) to[out=190,in=150](P2);
   \node[shape=circle,draw=blue!50,fill=blue!12,inner sep=0pt,minimum size=1mm]
   (origin) at (O) {};
   \node [blue,below] at (origin.south) {$O$};
   \node[shape=circle,draw=blue!50,fill=blue!12,inner sep=0pt,minimum size=1mm]
   (end) at (P3) {};
   \node [blue,below] at (end.south) {$P$};

   \draw  [tdplot_main_coords,dashed] (O) ellipse (42pt and 20pt);
   \tdplotsetthetaplanecoords{40};
   \draw [thick,tdplot_rotated_coords](1.5,0,0) arc (0:360:1.5);

   \draw  [tdplot_main_coords,dashed] (P3) ellipse (42pt and 20pt);
   \tdplotsetthetaplanecoords{40};
   \draw [tdplot_rotated_coords] (P3) circle (1.5);
  
   \draw (1.8,0,0.2) -- (5.4,3.0,0.2);
   \draw (1.9,0,1.2) -- (5.5,2.9,1.2);
        
 \end{tikzpicture}


 \tdplotsetmaincoords{60}{40}
 \begin{tikzpicture}[tdplot_main_coords,gray,thick]
   
   \coordinate (O) at (0,0,0);
      
   \tdplotsetcoord{P}{2.0}{90}{10};
   \tdplotsetcoord{P2}{3.5}{90}{-30};
   \tdplotsetcoord{P3}{7.5}{90}{40};
   
   \draw [->,blue!52](P) to[out=190,in=150](P2);
   \node[shape=circle,draw=blue!50,fill=blue!12,inner sep=0pt,minimum size=1mm]
   (origin) at (O) {};
   \node [blue,below] at (origin.south) {$O$};
   \node[shape=circle,draw=blue!50,fill=blue!12,inner sep=0pt,minimum size=1mm]
   (end) at (P3) {};
   \node [blue,below] at (end.south) {$P$};

   \draw  [tdplot_main_coords,dashed] (O) ellipse (22pt and 5pt);
   \tdplotsetthetaplanecoords{40};
   \draw [tdplot_rotated_coords](O) circle (0.8);

   \draw  [tdplot_main_coords,dashed] (P3) ellipse (22pt and 5pt);
   \tdplotsetthetaplanecoords{40};
   \draw [tdplot_rotated_coords] (P3) circle (0.8);
  
   \draw (1.0,0.0,0.4) -- (6.2,3.0,1.0);
        
 \end{tikzpicture}

 \tdplotsetmaincoords{60}{40}
 \begin{tikzpicture}[tdplot_main_coords,gray,thick]
   
   \coordinate (O) at (0,0,0);
      
   \tdplotsetcoord{P}{2.0}{90}{45};
   \tdplotsetcoord{P2}{4.5}{90}{45};
   \tdplotsetcoord{P3}{3.0}{90}{40};
   \tdplotsetcoord{P4}{6.5}{90}{45};
   \tdplotsetcoord{P5}{6.7}{90}{40};
   \tdplotsetcoord{P6}{8.0}{90}{40};

   \draw [->,blue!52](P2) to[out=50,in=130](P4);
   \node[shape=circle,draw=blue!50,fill=blue!12,inner sep=0pt,minimum size=1mm]
   (origin) at (O) {};
   \node [blue,below] at (origin.south) {$O$};
   \node[shape=circle,draw=blue!50,fill=blue!12,inner sep=0pt,minimum size=1mm]
   (end) at (P3) {};
   \node [blue,below] at (end.south) {$P$};
   \node[shape=circle,draw=gray,fill=gray,inner sep=0pt,minimum size=2mm]
   (end) at (P5) {};
   \node [blue,below] at (end.south) {$0$};
   \node[shape=circle,draw=gray,fill=gray,inner sep=0pt,minimum size=2mm]
   (end) at (P6) {};
   \node [blue,below] at (end.south) {$P$};

   \draw  [tdplot_main_coords,dashed] (O) ellipse (13pt and 5pt);
   \tdplotsetthetaplanecoords{40};
   \draw [tdplot_rotated_coords](O) circle (0.5);

   \draw  [tdplot_main_coords,dashed] (P3) ellipse (13pt and 5pt);
   \tdplotsetthetaplanecoords{40};
   \draw [tdplot_rotated_coords] (P3) circle (0.5);
      
 \end{tikzpicture}
\end{center}
  \caption{Evoluzione del dumbbell  per curvatura media}
  \label{fig:cp-122}
\end{figure}
\end{esempio}
Questi due esempi ci confermano quanto detto sopra, cioè superfici convesse collassano in un punto e superfici non convesse posso generare delle discontinuità, vediamo come aggirare questi due problemi. 

Partiamo dal primo. Quello che vogliamo fare è normalizzare l'equazione \eqref{eq:cp-123} così che il volume vega preservato. Tale processo di normalizzazione, viene raggiunto cambiando la scala temporale, passando da $t$ a $\tau$, in modo tale che la nostra superficie diventi:
\[
\mathcal{\tilde{S}}(\tau):=\psi(t)\mathcal{S}(t),
\]
con $\psi(t)$ rappresentante il fattore di normalizzazione (\emph{time-scaling}) dato da:
\[
\psi^n(t) = \frac{\partial \tau}{\partial t},
\]
con $n$ tale che $\psi^{-n+1}(t)H=\tilde{H}$. Uno dei vantaggi di questa normalizzazione è che le due superfici hanno la stesse proprietà geometriche e quindi il loro comportamento è identico. Il nuovo tempo $\tau$ deve essere scelto in modo da ottenere $V_{\tau}\equiv 0$. Ricordando che la nostra superficie $\mathcal{S}(t)$ si evolve secondo l'equazione \eqref{eq:cp-123}, allora otteniamo la seguente evoluzione per $\mathcal{\tilde{S}}(\tau)$ : 
\[
\begin{aligned}
  &\frac{\partial\mathcal{\tilde{S}}(\tau)}{\partial\tau} =  \frac{\partial\mathcal{\tilde{S}}(t(\tau))}{\partial\tau} = \frac{\partial t}{\partial\tau}\frac{\partial\mathcal{\tilde{S}}(t(\tau))}{\partial t} =\\
  &=\psi^{-n}\left(\psi_t\mathcal{S}+\psi\mathcal{S}_t\right)=\psi^{-n}\psi_t\mathcal{S}+\psi^{-n+1}H\vec{n}=\\
  &=\psi^{-n}\psi_t\mathcal{S}+\tilde{H}\vec{n}=\psi^{-n-1}\psi_t\mathcal{\tilde{S}}+\tilde{H}\vec{n}.
\end{aligned}
\]

Poichè la geometria dell'evolulzione è influenzata solo dalla componente normale della velocità, utilizzando il Lemma \ref{lemm:cp-121} otteniamo:
\[
\frac{\partial\mathcal{\tilde{S}}(\tau)}{\partial\tau}=\psi^{-n-1}\psi_t\left<\mathcal{\tilde{S}}(\tau),\vec{n}\right>\vec{n}+\tilde{H}\vec{n}.
\]
Definiamo la funzione supporto $\rho$ come:
\[
\rho:=-\left<\mathcal{S},\vec{n}\right>,
\]
allora
\[
\frac{\partial\mathcal{\tilde{S}}(\tau)}{\partial\tau}=\left(-\psi^{-n-1}\psi_t\tilde{\rho}+\tilde{H}\right)\vec{n}.
\]
Ricordando che, se una superficie evolve secondo un'equazione del tipo
\[
\frac{\partial \mathcal{S}}{\partial t}=\beta\vec{n},
\]
allora il tasso di cambiamento del volume è dato da
\begin{equation}
\label{eq:cp-124}
\frac{d V}{d t}=-\iint\beta d\mu
\end{equation}
con $d\mu$ elemento infinitesimo di superficie, vedere \emph{Osservazione} \ref{oss:cp-122}.
 Applicando la \eqref{eq:cp-124} giungiamo a:
\[
\tilde{V}_{\tau}=\iint(\psi^{-n-1}\psi_t\tilde{\rho}-\tilde{H})d\tilde{\mu},
\]
se noi richiediamo che $\tilde{V}_{\tau}\equiv 0$, allora:
\[
\psi^{-n-1}\psi_t\iint\tilde{\rho}d\mu = \iint\tilde{H}d\tilde{\mu}.
\]
 Tenendo presente che il volume di una superficie è pari a:
\begin{equation}
  \label{eq:cp-125}
V=\frac{1}{3}\iint-\left<\mathcal{S},\vec{n}\right>d\mu,
\end{equation}
con $\vec{n}$ che indica la normale interna (ved \emph{Osservazione} \ref{oss:cp-121}), e che rimane costante nel tempo (come da noi richiesto) possiamo scrivere:
\[
V(0)=\tilde{V}(\tau)=\frac{1}{3}\iint\tilde{\rho}d\tilde{\mu};
\]
mettendo insieme questa con l'espressione precedente otteniamo:
\[
\psi^{-n-1}\psi_t=\frac{\iint\tilde{H}d\tilde{\mu}}{3V_0},
\]
e finalmente l'equazione \emph{volume preserving} cercata:
\begin{equation}
\label{eq:cp-126}
\frac{\partial\mathcal{\tilde{S}}(\tau)}{\partial\tau}=\left(\tilde{H}-\frac{\tilde{\rho}\iint\tilde{H}d\tilde{\mu}}{3V_0}\right).
\end{equation}
\begin{osservazione}[Calcolo del Volume]
\label{oss:cp-121}
Vediamo da dove deriva l'espressione \eqref{eq:cp-125}. Prendiamo un insieme $\Omega\subset\mathbb{R}^3$ limitato, la cui frontiera è costituita da una superficie regolare e orientabile, ed un campo vettoriale $F$ regolare in esso, allora vale il \emph{Teorema della Divergenza} cioè:
\[
\int_{\Omega}\Div{F(x)}dV =\int_{\partial\Omega}\left<F(x),\vec{n}(x)\right>d\mu
\]
dove $\vec{n}(x)$, in tal caso, indica la normale esterna e $dV$,$d\mu$ rispettivamente elemento di volume ed elemento di superficie. Prendiamo come campo vettoriale $F(x)=\frac{1}{3}x$, quindi il volume di $\Omega$ è pari a:
\[
vol(\Omega)=\int_{\Omega}1dV=\int_{\Omega}\Div{(F(x))}dV=\frac{1}{3}\int_{\partial\Omega}\left<x,\vec{n}\right>d\mu,
\]
essendo l'ultimo integrale fatto su $\partial\Omega$, il vettore $x$ rappresenta un punto della supeficie, quindi equivale a \eqref{eq:cp-125}.
\end{osservazione}

\begin{osservazione}[Tasso di variazione del Volume]
\label{oss:cp-122}
Cerchiamo di dare una spiegazione all'equazione \eqref{eq:cp-124}. Partiamo da una superficie $\Gamma$ regolare immersa orientabile chiusa in $\mathbb{R}^3$
e una funzione regolare $\gamma=\gamma(x,\vec{n},H)$, che dipende dal punto di superficie $x$, dalla normale unitaria $\vec{n}$ e dalla curvatura media $H$.
Consideriamo un dominio $\mathcal{D}$ che contiene la superficie $\Gamma$ ed un campo vettoriale $\vec{V}$ definito su $\mathcal{D}$, il quale definisce la seguenza continua di superfici perturbate $\left\{\Gamma_t\right\}_{t\ge 0}$ con $\Gamma_0:=\Gamma$. Ed ogni punto $x\in \Gamma_0$ è deformato da un equazione differenziale ordinaria definita dal campo $\vec{V}$:
\[
\frac{dx}{dt}=\vec{V}(x(t)),\quad \forall t\in[0,T],\quad x(0)=X\in\Gamma_0=\Gamma.
\]
Ora prendiamo il seguente funzionale $J(\Gamma)$:
\[
J(\Gamma) = \int_{\Gamma}\gamma(x,\vec{n},H)dS,
\]
si definisce \emph{shape derivate} o derivata Euleriana di $J(\Gamma)$ in $\Gamma$ nella direzione del campo vettoriale $\vec{V}$ il limite
\[
dJ(\Gamma;\vec{V}) = \lim_{t\to 0}\frac{1}{t}(J(\Gamma_t)-J(\Gamma)).
\]
Per ottenere la \eqref{eq:cp-124} dobbiamo calcolarci questa shape derivate e per farlo useremo il \emph{Corollario 4.1} dell'articolo \cite[vedi][]{dog:noch} con $\gamma$ dipendente solo dal punto $x$ e dalla normale unitaria $\vec{n}$: 
\begin{equation}
\label{eq:cp-121-add}
\gamma(x,\vec{n})=-\frac{1}{3}<\mathcal{S},\vec{n}> = -\frac{1}{3}\sum_{i=1}^3x_in_i.
\end{equation}
Secondo tale corollario otteniamo:
\[
dJ(\Gamma;\vec{V}) = \int_{\Gamma}(H\gamma+\partial_{n}\gamma+\Div_{\Gamma}[\gamma_y]_{\Gamma})VdS.
\]
dove:
\begin{itemize}
  \item $\partial_n\gamma=<D_x\gamma,\vec{n}>$, sarebbe la derivata partiale rispetto ad $x$ nella direzione della normale;
  \item $\gamma_y=D_n\gamma$, indica il grandiente di $\gamma$ rispetto alla normale;
  \item $[\cdot]_{\Gamma}$, rappresenta la proiezione sul piano tangente;
  \item $\Div_{\Gamma}\vec{W}=\left(\Div\vec{W}-\vec{n}\cdot D\vec{W} \cdot \vec{n}\right)_{|\Gamma}$, denota la divergenza tangensiale di $\vec{W}$, propriamente esteso in un intorno di $\Gamma$;
  \item $\Div_{\Gamma}[\gamma_y]_{\Gamma}=\Div_{\Gamma}[\gamma_y]-H<\gamma_y,\vec{n}>$, con $\Div_{\Gamma}[\gamma_y]$ calcolata nel \emph{Lemma 3.3} di \cite[vedi][]{dog:noch};
  \item $V=<\vec{V},\vec{n}>$, componente normale del campo vettoriale.
\end{itemize}
Calcolandoci queste quantità per $\gamma$ pari a \eqref{eq:cp-121-add} e ricordando che nel nostro caso la componente normale del campo è $\beta$,  otteniamo:
\[
dJ(\Gamma;\vec{V})=\int_{\Gamma}(H\gamma -\frac{1}{3} -1 +\frac{1}{3} -H\gamma)\beta dS=-\int_{\Gamma}\beta dS.
\]
\end{osservazione}
%%%%%%%%%%%%%%%%%%%%%%%%%%%%%%%%%
%
%
% Section 1.3 
%
%
%%%%%%%%%%%%%%%%%%%%%%%%%%%%%%%%%%
\section{Approccio \emph{level-set}}  

Nella sezione precedente abbiamo visto come risolvere il problema del collasso, cioè evitare che in un tempo finito il volume diventa nullo. Tuttavia quando si implementa numericamente tale flusso, ci si imbatte in altri problemi:
\begin{itemize}

  \item Il nostro algoritmo deve essere robusto in quanto deve approssimare un flusso continuo. Una semplice approssimazione Lagrangiana basata sulle particelle che si muovono lungo la superficie richiederebbe un passo temporale impraticabile, poichè le particelle possono trovarsi molto vicine o molto lontane durante la deformazione.

   \item Lo svilupparsi di singolarità, è un altro problema che dobbiamo tener presente. Quindi abbiamo bisogno di uno schema che trovi ,in tal caso, la corretta soluzione debole.
   
   \item Cambiamenti topologici possono verificarsi (fusione o divisione) e tener traccia di questi con tracciamenti di particelle è un compito alquano arduo.

\end{itemize}

Tutti questi problemi hanno portato allo svilupparsi di tecniche \emph{level-set}. Introduciamo alcuni concetti base e riformuliamo la nostra equazione con un \emph{approccio level-set}, per maggiori chiamrimenti o approfondimenti \cite[vedi][]{giga:main}.
Alla base di questo approccio c'è l'idea di vedere la nostra superficie come  insieme di livello di una funzione. Più precisamente,
un insieme $S\subset\mathbb{R}^3$ è chiamato superfice di classe $C^m$ (o ipersuperficie se siamo in $\mathbb{R}^N$ con $N>3$) intorno ad un punto $x_0$ di $S$ se esiste una funzione $u(x)$ di classe $C^m$ definita in un intorno $U$ di $x_0$ tale che
\[
S\cap U=\left\{x\in U; u(x)=0\right\}
\]
e il gradiente di $u$
\[
 Du=\left(\frac{\partial u}{\partial x_1},\dots,\frac{\partial u}{\partial x_3}\right)
\]
non svanisce su $S$. Tale rappresentazione è chiamta \emph{level-set}. Se $u$ può essere presa $C^{\infty}$ in $U$, diremo che la superficie $S$ è regolare intorno ad $x_0$ e se $u$ è anche definita intorno ad ogni punto di $x_0\in S$, allora si parla semplicemetne di superficie regolare. Questa reappresentazione può essere estesa anche ad insiemi dipendenti dal tempo. Difatti diremo che una famiglia $S_t$ è una superficie regolare che si evolve nel tempo se esiste una funzione $u(x.t)$ definita per ogni $x\in S_t$ e $t\in I$ tale che:
\[
S_t=\left\{x\in S_t; u(x,t)=0\right\},
\]
e tale che il gradiente spaziale di $u$ non svanisca su $S_t$. Ora, consideriamo una curva $\alpha(t)$ in $\mathbb{R}^3$ su $S_t$ definita in $(t-\delta,t+\delta)$, con $\delta>0$ tale che $\alpha(t)\in S_t$ e $\alpha(t_0)=x_0$. Dalla definizione di $S_t$ deve essere $u(\alpha(t),t)=0$ per $t$ vicino a $t_0$, quindi differenziando  $u(\alpha(t),t)$ ed evalutandola nel punto $(x_0,t_0)$ otteniamo:
\[
u_t(x_0,t_0)+\left<\frac{d\alpha}{dt}(t_0),Du(x_0,t_0)\right>=0,
\]
ricordando che la normale $\vec{n}=-\frac{Du}{|Du|}$(vedi \emph{Osservazione} \ref{oss:cp-131}), abbiamo:
\[
u_t(x_0,t_0)-|Du(x_0,t_0)|\left<\frac{d\alpha}{dt}(t_0),\vec{n}\right>=0.
\]
Il prodotto scalare che compare nell'espressione precedente rappresenta la velocità normale di $S_t$ nel punto $x_0$ al tempo $t_0$ per cui: 
\[
V=\frac{u_t(x_0,t_0)}{|Du(x_0,t_0)|},
\]
chiaramente  $V$ non dipende dalla scelta della curva $\alpha(t)$. Poichè nel moto per curvatura media la velocità normale è uguale alla curvatura media otteniamo:
\begin{equation}
  \label{eq:cp-131}
  u_t(x,t)=|Du(x,t)|\Div{\left(\frac{D(x,t)}{|D(x,t)|}\right)}
\end{equation}
che rappresenta la formulazione level-set dell'equazione \eqref{eq:cp-123}. 
Quindi, ricapitolando, quando una funzione si muove secondo la \eqref{eq:cp-131}, i suoi insiemi di livello si muovono secondo la \eqref{eq:cp-123}. Inseriamo anche il termine che preserva il volume:
\begin{equation}
  \label{eq:cp-132}
  u_t(x,t)=|Du(x,t)|\Div{\left(\frac{D(x,t)}{|D(x,t)|}\right)}-\frac{\iint\Div{\left(\frac{D(x,t)}{|D(x,t)|}\right)}d\mu}{3V_0}x^tDu
\end{equation}
 Tale formulazione ci consente di superare i problemi messi in risalto ad inizio paragrafo, infatti:
\begin{itemize}
  \item Riusciamo a risolver il problema di creare un metodo robusto e stabile, in quanto la formulazione level-set è scritta in un sistema di coordinate fisso quindi non dobbiamo tener traccia di come evolvono tutte le particelle.

  \item L'emrgere di discontinuità viene risolto utilizzando la teoria viscosa, in quanto è stato dimostrata per alcune velocità (tra cui anche quella presa in esame da noi) l'esistenza e l'unicità di soluzioni viscose per il flusso level-set.

   \item I cambiamenti topologici della superficie non sono più un problema, poichè la topologia di $u$ è fissa. Tali cambiamenti vengono scoperti solo quando viene calcolato il corrispondente insieme di livello. 

\end{itemize}

\begin{osservazione}[vettore unitario in rappresentazione level-set]
\label{oss:cp-131}
Supponiamo che $S$ sia una superfice intorno a $x_0\in S$, in rappresentazione level-set. Se $\vec{n}(x)$ è un vettore normale unitario in $x$ vicino $x_0$ e esso dipende da $x$ in modo almeno continuo, diremo che $\vec{n}$ è un campo unitario normale di orientazione (attorno a $x_0$) di $S$. Vediamo che esiste, almeno in un intorno di $x_0$. Per $x\in S\cap U$ sia $\alpha(t)$ una curva su $S$ tale che $\alpha(0)=x$. Deriviamo $u(\alpha(t)) = 0$ e cancoliamolo in $t=0$
\[
\left<Du(x),\frac{d\alpha}{dt}(0)\right> = 0.
\]
Questo ci dice che il gradiente in $x$ di $u$ è ortogonale al piano tangente a $S$ in $x$. Per la formulazione level-set $u\in C^m$ , $Du\in C^{m-1}$ ($m>0$) quindi almeno continua e $Du$ non svanisce intorno a $x_0$; per cio possiamo definire il vetore normale (campo vettoriale unitario di orientazione) come
\[
\vec{n}(x)=-\frac{Du(x)}{|Du(x)|},
\]
il segno meno è per fissare le idee.
\end{osservazione}
