\chapter{Evoluzione Geometrica Per Curvatura Media}
'WIP'
%%%%%%%%%%%%%%%%%%%%%%%%%%%%%%%%%%%%%%%%%%%%%%%%%%%%
%
% Section 1.1
%
%%%%%%%%%%%%%%%%%%%%%%%%%%%%%%%%%%%%%%%%%%%%%%%%%%%%
\section{Evoluzioni di Superfici in $\mathbb{R}^3$}

Consideriamo un sottoinsieme $S\subseteq\mathbb{R}^3$, diremo che è una superficie regolare se per ogni punto $p\in S$ esiste un intorno $V$ di $p$ in $\mathbb{R}^3$, un aperto $U\subset\mathbb{R}^2$ e un'applicazione differenziabile $\mathcal{S}:U\longrightarrow V\cap S$ tale che:
\begin{enumerate}
  \item $\mathcal{S}$ è un omeomorfismo,
  \item il differenziale $d\mathcal{S}$ ha rango massimo in ogni punto di $U$.
\end{enumerate}
L'applicazione $\mathcal{S}$ si dice parametrizzazione regolare in un intorno del punto $p$. Mentre l'insieme $V\cap S=\mathcal{S}(U)$ si dice porzione di superficie parametrica regolare. La condizione di regolarità, ci esclude superfici che intersecano se stesse.

Siano $u$ e $v$ coordinate su $U\subset\mathbb{R}^2$ e $p=\mathcal{S}(u_0,v_0)$; si definisce spazio tangente ad $S$ nel punto $p$, e si indica con $T_pS$, l'immgine di $\mathbb{R}^2$ tramite l'applicazione $d\mathcal{S}_{(u_0,v_0)}$:
\[
(a,b)\in\mathbb{R}^2\quad d\mathcal{S}_{(u_0,v_0)}(a,b)= 
\frac{d}{dt}\mathcal{S}(u_0+at,v_0+bt)|_{t=0}=
\]
\[
=a\mathcal{S}_u(u_0,v_0)+b\mathcal{S}_v(u_0,v_0),
\]
con $\mathcal{S}_u(u_0,v_0)$ e $\mathcal{S}_v(u_0,v_0)$ colonne della matrice:
\[
d\mathcal{S}=
\begin{bmatrix}
  x_u & x_v \\
  y_u & y_v \\
  z_u & z_v 
\end{bmatrix}
\]
avendo tale matrice rango massimo, per definizione di superficie regolare, i vettori $\mathcal{S}_u$ e $\mathcal{S}_v$ sono dunque una base per $T_pS$. Poichè il loro prodotto vettoriale è non nullo, in quanto linearmente indipendenti, e ortogonale al piano da essi individuato, definiamo il versore normale alla superficie nel punto $p$ come (vedere figura \ref{fig:cp-11}):
\begin{equation}
  \label{eq:cp-10}
  \vec{n} = \frac{\mathcal{S}_u\wedge\mathcal{S}_v}{||\mathcal{S}_u\wedge\mathcal{S}_v||}.
\end{equation}
\begin{figure}[!hp]
  \tdplotsetmaincoords{60}{40}
  \begin{tikzpicture}[tdplot_main_coords]

    \coordinate (O) at (0,0,0);

    \tdplotsetcoord{P}{1.5}{90}{240};
    \tdplotsetcoord{P1}{2.5}{90}{105};
    \tdplotsetcoord{P2}{2.7}{90}{-30};
    \tdplotsetcoord{P3}{3.4}{90}{45}

    
    \draw[thick,->,blue] (0,0,0)node[anchor=east]{$p$} -- 
    (2,0,0) node[anchor=south]{$\mathcal{S}_u$};
    \draw[thick,->,blue] (0,0,0) -- (0,2,0) node[anchor=north]{$\mathcal{S}_v$};
    \draw[thick,->,blue] (0,0,0) -- (0,0,2.5) node[anchor=north east]{$\vec{n}$};
    
    \draw (P) -- (P1) -- (P3) -- (P2) -- (P);
    
    \tdplotsetcoord{W}{6}{90}{281}
    \tdplotsetcoord{W1}{6.5}{90}{91}
    \tdplotsetcoord{W2}{6}{90}{318}
    \tdplotsetcoord{W3}{5.5}{90}{25}
    
    \draw (W) .. controls (-2,0,0) and (-2,1,0) .. (W1);
    \draw[dashed] (W2) .. controls (3,0,0) and (3,0.5,0) .. (W3);
    \draw (W) to[out=60,in=60] (W2);
    \draw (W) to[out=240,in=240](W2);
    \draw[dashed] (W1) to[out=190,in=190] (W3);
    \draw (W1) to[out=10,in=10] (W3);

  \end{tikzpicture}

  \caption{Piano tangente e versore normale alla superficie $S$ nel punto $p$.}
  \label{fig:cp-11}
\end{figure}



