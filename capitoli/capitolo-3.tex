\chapter{Costruzione ed analisi dello schema Semi-Lagrangiano}
In questo capitolo costruiremo ed analizzeremo uno schema Semi-Lagrangiano per l'equazione di curvatura media in $\mathbb{R}^3$.
Ricordiamo la nostra equazione :
\begin{equation}
\label{eq:cp3-01}
\begin{cases}
u_t = \Div\left(\frac{Du}{|Du|}\right)|Du|\quad (x_1,x_2,x_3)\in\mathbb{R}^3,t\in(0,T) \\
u(x,0) = u_0(x) \qquad \quad \mathbb{R}^3\times\{t = 0\}
\end{cases}
\end{equation}
con $Du$ il gradiente di $u$.
%%%%%%%%%%%%%%%%%%%%%%%%%%%%%%%%%%%%%%%%%%%%%%%%%%%%%%%%%%%%%%%%%%%%%%%%%%%%%
% Section 3.1 Costruzione dello schema
%
%%%%%%%%%%%%%%%%%%%%%%%%%%%%%%%%%%%%%%%%%%%%%%%%%%%%%%%%%%%%%%%%%%%%%%%%%%%%%%%%
\section{Costruzione dello schema MCM}
Partiamo dall'equazione \eqref{eq:cp3-01} e operiamo alcune manipolazioni algebriche, al fine di scrivere meglio il termine del flusso. 
\begin{proposizione}
Vale la seguente serie di uguaglianze:
\[
\Div\left(\frac{Du}{|Du|}\right)|Du|=\trace\left(\left(I-\frac{DuDu^t}{|Du|^2}\right)D^2u\right)=\vec{v}_1^tD^2u\vec{v}_1+\vec{v}_2^tD^2u\vec{v}_2.
\]
con $\vec{v}_1$ e $\vec{v}_2$ autovettori della matrice $P=I-\frac{DuDu^t}{|Du|^2}$, definiti da :
\[
\vec{v_1}=
\begin{bmatrix}
\frac{-u_{x_3}}{\sqrt{u_{x_1}^2+u_{x_3}^2}} \\
0 \\
\frac{u_{x_1}}{\sqrt{u_{x_1}^2+u_{x_3}^2}}
\end{bmatrix}
\quad
\vec{v_2}=\frac{1}{|Du|}
\begin{bmatrix}
\frac{-u_{x_1}u_{x_2}}{\sqrt{u_{x_1}^2+u_{x_3}^2}} \\
\sqrt{u_{x_1}^2+u_{x_3}^2} \\
\frac{-u_{x_2}u_{x_3}}{\sqrt{u_{x_1}^2+u_{x_3}^2}}
\end{bmatrix}
\]
\end{proposizione}
\begin{proof}
Iniziamo con la prima uguaglianza, svolgendo esplicitamente $\Div\left(\frac{Du}{|Du|}\right)$ (senza riportare tutti i conti) otteniamo :
\[
|Du|\Div\left(\frac{Du}{|Du|}\right)=|Du|\left(\frac{\Div(Du)}{|Du|} -(Du)^tD^2u\frac{Du}{|Du|^3}\right) =
\]
\[
= \Delta u - \left<D^2u\frac{Du}{|Du|},\frac{Du}{|Du|}\right> .
\]
Ora ricordando che $\Delta u$ rappresenta il laplaciano della funzione $u$ e osservando che :
\[
\left<D^2u\frac{Du}{|Du|},\frac{Du}{|Du|}\right> = \left<\frac{1}{|Du|}(u_{x_1}(D^2u)^1+u_{x_2}(D^2u)^2+u_{x_3}(D^2u)^3),\frac{Du}{|Du|}\right> =
\]
\[
=\frac{1}{|Du|^2}(u_{x_1}<(D^2u)^1,Du>+u_{x_2}<(D^2u)^2,Du>+u_{x_3}<(D^2u)^3,Du> = 
\]
\[
=\trace\left(\frac{DuDu^t}{|Du|^2}D^2u\right),
\]
avendo indicato con $(D^2u)^n$ l'$n$-esima colonna della matrice delle derivate seconde della funzione $u$, giungiamo all'uguaglianza cercata:
\[
\Div\left(\frac{Du}{|Du|}\right)|Du| = \trace\left(\left(I-{\frac{DuDu^t}{|Du|^2}}\right)D^2u\right).
\] 
Ora occupiamoni della seconda uguaglianza. Iniziamo co fare alcune osservazioni sulla matrice $P=I-\frac{DuDu^t}{|Du|^2}$. Questa è una matrice di proiezione, $3\times3$ a valori in $\mathbb{R}$, simmetrica ($P^t=P$) e idempotente ($P^2=P$), ammette due autovettori $\vec{v}_1$,$\vec{v}_2$ e vale la seguente decomposizione
\[
P=\sum_{i=1}^2\vec{v}_i\vec{v}_i^t=\sigma\sigma^t\quad\text{ con } \sigma=[\vec{v_1},\vec{v_2}],
\]
con $\sigma$ matrice $3\times2$ che ha come colonne gli autovettori di $P$. Detto ciò abbiamo: 
\[
\trace\left(\left(I-{\frac{DuDu^t}{|Du|^2}}\right)D^2u\right)=\trace\left(\sigma\sigma^tD^2\right)=\trace\left([\vec{v}_1,\vec{v}_2][\vec{v}_1,\vec{v}_2]^tD^2u\right)=
\]
\[
=\trace\left((\vec{v}_1\vec{v}_1^t+\vec{v}_2\vec{v}_2^t)D^2u\right)=\trace\left(\vec{v}_1\vec{v}_1^tD^2u\right)+\trace\left(\vec{v}_2\vec{v}_2^tD^2u\right)
\]
Ricordando i calcoli fatti nella dimostrazione della prima uguaglianza possiamo scrivere:
\[
\begin{split}
&\trace\left(\vec{v}_1\vec{v}_1^tD^2u\right)+\trace\left(\vec{v}_2\vec{v}_2^tD^2u\right)=\left<D^2u\vec{v}_1,\vec{v}_1\right>+\left<D^2u\vec{v}_2,\vec{v}_2\right>=\\
& =\vec{v}_1^tD^2u\vec{v}_1+\vec{v}_2^tD^2u\vec{v}_2.
\end{split}
\]
\end{proof}

Riformuliamo l'equazione \eqref{eq:cp3-01} nel seguente modo :
\begin{equation}
u_t=\vec{v}_1^tD^2u\vec{v}_1 + \vec{v}_2^tD^2u\vec{v}_2,
\end{equation}
e costruiamoci il metodo.
Approssimiano la derivata temporale con le differenze finite in avanti, mentre per il termine del flusso utilizziamo le derivate direzionali scegliendo come incremento $\sqrt{2\Delta t}$.
Quindi nel caso in cui $Du \ne 0$ e $u_{x_1}^2+u_{x_3}^2\ne 0$ otteniamo :
\[
\begin{split}
&\frac{u(x,t+\Delta t)-u(x,t)}{\Delta t} +O(\Delta t)= \frac{1}{4\Delta t}\bigl[u(x+\sqrt{2\Delta t}(\vec{v}_1+\vec{v}_2),t) +\\
& +u(x+\sqrt{2\Delta t}(-\vec{v}_1+\vec{v}_2),t) + u(x+\sqrt{2\Delta t}(\vec{v}_1-\vec{v}_2),t) \\
& + u(x+\sqrt{2\Delta t}(-\vec{v}_1-\vec{v}_2),t) - 4u(x,t)\bigr] + O(\Delta t)
\end{split}
\]

Osservando tale approssimazione, possiamo notare che i punti  \\
$(x + \sqrt{2\Delta t}(\pm\vec{v}_1 \pm\vec{v}_2))$ non necessariamente saranno i nodi della griglia, quindi è necessario introdurre l'operatore di interpolazione $I[.]$.
Per completare la costruzione consideriamo una griglia spaziale $x_j=(j_1\Delta x,j_2\Delta x,j_3\Delta x)$ con $j=(j_1,j_2,j_3)$ multi-indice in $\mathbb{Z}^3$ ed una griglia temporale $t^n=n\Delta t$ con $n\in\mathbb{N}$.
Indichiamo con $u_j^n$ l'approssimazione di $u$ nel punto della griglia $(x_j,t^n)$ e con $D_j^n$ l'approssimazione del gradiente di $u$ nel suddetto punto. Quindi il nostro schema semi-lagrangiano risulta :
\begin{equation}
\label{eq:cp3-02}
u_j^{n+1} = \frac{1}{4}\sum_{+,-}I[u^n](x_j+\sqrt{2\Delta t}(\pm\vec{v}_1\pm\vec{v}_2)).
\end{equation} 
\begin{osservazione}
Questa costruzione non è applicabile nel caso in cui 
\begin{equation}
\label{eq:cp3-01-add}
Du = 0 \text{ o } u_{x_1}^2+u_{x_3}^2 = 0. 
\end{equation}
Quindi per distinguere i due casi dal punto di vista numerico, introduciamo una soglia $s$ ed una costante $C$. In modo tale che il verificarsi del caso \eqref{eq:cp3-01-add} è rappresentato numericamente dalle disequazioni
\[
|D_j^n|\le C\Delta x^s\text{ e }|(D_j^n)_1^2+(D_j^n)_3^2|\le C\Delta x^s.
\]
\end{osservazione}
*** Manca da trattare il caso discontinuo ****

%%%%%%%%%%%%%%%%%%%%%%%%%%%%%%%%%%%%%%%%%%%%%%%%%%%%%%%%%%%%%%%%%%%%%%%%%%%%%%
% Section 3.2 Consistenza dello schema
%
%%%%%%%%%%%%%%%%%%%%%%%%%%%%%%%%%%%%%%%%%%%%%%%%%%%%%%%%%%%%%%%%%%%%%%%%%%%%%%
\section{Consistenza dello schema}
Occupiamoci ora dell'analisi della consitenza dello schema. Prima di tutto ricordiamo che , per un dato schema numerico $S(x_j,u_j^n,Du_j^n)$ , l'errore di troncamento locale (o LTE) $\tau_j^n$ è l'errore che si genera pretendendo che la soluzione esatta soddisfi lo schema numerico :
\[
\tau_j^n=\frac{u(x_j,t^{n+1})-S(x_j,u(x_j,t^n),Du(x_j,t^n))}{\Delta t}
\]
Si definisce anche errore di troncamento totale la quantità
\[
\tau(\Delta t,\Delta x) := \max_{j,n} |\tau_j^n|.
\]
In generale un metodo si dirà consistente quando il suo LTE è infinitesimo rispetto all'incremento. Mentre si dirà consitente di ordine p in tempo e q in spazio se :
\[
\tau(\Delta t,\Delta x) = O(\Delta t^p +\Delta x^q).
\] 
Nel nostro caso, in quanto l'equazione che approssimiamo è singolare quando il gradiente  $Du$ si annulla oppure quando $u_{x_1}^2+u_{x_3}^2=0$, è necessario introdurre una definizione di consistenza più generale.
\begin{definizione}
Sia $(\Delta x_m,\Delta t_m)$ una seguenza generica di parametri di discretizzazione e sia $(x_{j_m},t^{n_m})\in G\times\left\{0,\dots,\Delta t_mN\right\}$ una seguenza generica di nodi, tale che per $m\to\infty$,
\[
(\Delta x_m,\Delta t_m)\to0 \quad\text{ e } (x_{j_m},t^{n_m})\to(x,t).
\]
E sia inoltre $\phi\in C^{\infty}(\mathbb{R}^3\times(0,t])$ e $\phi^{n_m-1}\equiv(\phi(x_{j_m},t^{n_m-1})_{x_{j_m}}$. Allora lo schema $S$ è detto consistente se
\begin{equation}
\label{eq:cp3-03}
\begin{split}
&\phi_t(x,t)+\underline{F}(D\phi(x,t),D^2\phi(x,t))\le \\
&\le\liminf_{m\to\infty}\frac{\phi(x_{j_m},t^{n_m})-S(x_{j_m},\phi_{j_m}^{n_m-1},D\phi_{j_m}^{n_m-1})}{\Delta t_m}\le \\
&\le\limsup_{m\to\infty}\frac{\phi(x_{j_m},t^{n_m})-S(x_{j_m},\phi_{j_m}^{n_m-1},D\phi_{j_m}^{n_m-1})}{\Delta t_m}\le \\
&\le\phi_t(x,t)+\overline{F}(D\phi(x,t),D^2\phi(x,t)).
\end{split}
\end{equation}
\end{definizione}
\begin{osservazione}
Osserviamo che se $F$ è continua allora il $\limsup$ ed il $\liminf$ nell'espressione \eqref{eq:cp3-03} coincidono, quindi la definizione si riduce a quella classica di consistenza.
\end{osservazione}

Indichiamo con $u(x,t)$ una soluzione regolare di \eqref{eq:cp3-01} e con $u(x_j,t^n)$ il suo valore sui nodi della griglia.
Prima di procedere con la consistenza, facciamo le seguenti ipotesi sull'errore commesso nell'approssimazione del gradiente e nell'interpolazione:
\begin{gather}
\label{eq:cp3-04}
||I[u^n](\cdot)-u(\cdot,t^n)||_{\infty}\le C_1\Delta x^r\quad\forall n\in\mathbb{N} \\
\label{eq:cp3-05}
|D_j[u^n]-Du(x_j,t^n)|\le C_2\Delta x^q\quad\forall n\in\mathbb{N}\quad\forall x_j, 
\end{gather}
con $C_1$ e $C_2$ costanti positive che dipendono così come gli ordini $r$ e $q$ dal metodo scelto per l'interpolazione e per l'approssimazioni del gradiente. Nel nostro caso abbiamo scelto entrambi i metodi di ordine $2$, quindi $r=q=2$.

Consideriamo il caso in cui $Du(x,t)\ne 0$ e $u_{x_1}(x,t)^2+u_{x_3}(x,t)^2\ne 0$, vista la regolarità di $u$, esiste un intorno di $(x,t)$ in cui le due espressioni precedenti si mantengono vere. Quindi l'approssimazione del gradiente verifica le seguenti diseguaglianze :
\[
|D_j^n|\ge C\Delta x^s\text{ e }|(D_j^n)_1^2+((D_j^n)_3^2|\ge C\Delta x^s
\]
per cui possiamo applicare lo schema nella forma \eqref{eq:cp3-02}.
\begin{osservazione}
Abbiamo definito $\sigma=[\vec{v}_1,\vec{v}_2]^t$ quindi ponendo  $\vec{b}=[\pm1,\pm1]^t$ possiamo riscrivere il termine generico della sommatoria in \eqref{eq:cp3-02} come
\[
I[u^n](x_j+\sigma(D_j^n)\sqrt{2\Delta t}\vec{b}).
\]
Ora calcolando lo jacobiano si $\sigma$ per un qualsiasi vettore $p$ in $\mathbb{R}^3$ (senza riportare tutto il conto) si dimostra che:
\[
\left|J_{\sigma}(p)\right|^2=\frac{1}{p_1^2+p_3^2},
\]
dove come norma è stata scelta quella di Forbenius. Quindi applicando $\sigma[\cdot]$ al gradiente e sfruttando il fatto che $(D_j^n)_1^2+(D_j^n)_3^2\ge C\Delta x^s$, possiamo concludere che la funzione $\sigma$ risulta lipschitziana con costante $L_{\sigma}\le\frac{1}{C\Delta x^s}$.
Inoltre, avendo preso la $u$ sufficientemente regolare, anche essa sarà lipschitiziana con costatne $L_u$ e con $|Du|< M$. Assumiamo anche che $|u_{tt}(x,t)|\le C$ in modo tale da limitare l'errore di troncamento per l'approssimazione temporale.
\end{osservazione}

 Analizziamo il prima addendo della sommatoria nell'espressione \eqref{eq:cp3-02} e facciamo alcune manipolazioni algebriche
\begin{equation}
\label{eq:cp3-06}
\begin{split}
& I[u^n](x_j+\sigma(D_j^n)\sqrt{2\Delta t}\vec{b}) = I[u^n](x_j+\sigma(D_j^n)\sqrt{2\Delta t}\vec{b}) - u(x_j+\sigma(D_j^n)\sqrt{2\Delta t}\vec{b},t^n) + \\
& + u(x_j+\sigma(D_j^n)\sqrt{2\Delta t}\vec{b},t^n) - u(x_j+\sigma(Du(x_j,t^n))\sqrt{2\Delta t}\vec{b},t^n)+ \\
& + u(x_j+\sigma(Du(x_j,t^n))\sqrt{2\Delta t}\vec{b},t^n).
\end{split}
\end{equation}
Stimiamo la parte destra di \eqref{eq:cp3-06}. Per la prima differenza applichiamo la stima dell'errore di interpolazione quindi
\begin{equation}
\label{eq:cp3-07}
|I[u^n](x_j+\sigma(D_j^n)\sqrt{2\Delta t}\vec{b}) - u(x_j+\sigma(D_j^n)\sqrt{2\Delta t}\vec{b},t^n)|\le C_1\Delta x^r,
\end{equation}
mentre per la seconda utilizzando la lipschitianità di $u$ , di $\sigma$ e la stima dell'errore di aprrossimazione del gradiente ottendo:
\begin{equation}
\label{eq:cp3-08}
|u(x_j+\sigma(D_j^n)\sqrt{2\Delta t}\vec{b},t^n) - u(x_j+\sigma(Du(x_j,t^n))\sqrt{2\Delta t}\vec{b},t^n)|\le M\sqrt{2\Delta t}\frac{C_2}{C}\Delta x^{q-s}.
\end{equation}
Usando \eqref{eq:cp3-07} e \eqref{eq:cp3-08} in \eqref{eq:cp3-06} otteniamo
\[
\begin{split}
&I[u^n](x_j+\sigma(D_j^n)\sqrt{2\Delta t}\vec{b})=u(x_j+\sigma(Du(x_j,t^n))\sqrt{2\Delta t}\vec{b},t^n)+O(\Delta x^r)+ \\
& +O(\Delta x^{q-s}\sqrt{2\Delta t}).
\end{split}
\]
Sviluppiamo ora $u(x_j+\sigma(Du(x_j,t^n))\sqrt{2\Delta t}\vec{b},t^n)$ con una serie di Taylor del terzo ordine in un intorno del punto $x_j$:
\begin{equation}
\label{eq:cp3-09}
\begin{split}
&u(x_j+\sigma(Du(x_j,t^n))\sqrt{2\Delta t}\vec{b},t^n) = u(x_j,t^n)+ \\
&+\left<Du(x_j,t^n),\sigma(Du(x_j,t^n))\vec{b}\right>\sqrt{2}\Delta t^{\frac{1}{2}} +\\
&+\left<D^2u(x_j,t^n)\sigma(Du(x_j,t^n))\vec{b},\sigma(Du(x_j,t^n))\vec{b}\right>\Delta t + \\
&+T_3(\Delta t^{\frac{3}{2}}\sigma(Du(x_j,t^n))\vec{b},D^3u(x_j,t^n))+O(\Delta t^2), 
\end{split}
\end{equation}
non abbiamo esplicitato i termini del terzo ordine in quanto , mettondo tutto insieme, si eliminano cosi come i termini del primo ordine data la simmetria dei $4$ punti $x_j+\sigma(Du(x_j,t^n))\vec{b}\sqrt{2\Delta t}$ con $b=[\pm 1,\pm 1]^t$. Alla fine otteniamo che:
\begin{equation}
\label{eq:cp3-010}
\begin{split}
& S(x_j,u(x_j,t^n),Du(x_j,t^n)) = \\
& = u(x_j,t^n) + \vec{v}_1^tD^2u(x_j,t^n)\vec{v}_1 + \vec{v}_2^tD^2u(x_j,t^n)\vec{v}_2 + \\
& + O(\Delta x^r) + O(\Delta x^{q-s}\Delta t^{\frac{1}{2}}) + O(\Delta t^2).
\end{split}
\end{equation}
Ora sviluppando  anche il termine $u(x_j,t^{n+1})$ e applicando la definizione generale di consistenza \eqref{eq:cp3-03} otteniamo 
\[
\begin{split}
&\frac{u(x_j,t^{n+1})-S(x_j,u(x_j,t^n),Du(x_j,t^n))}{\Delta t}= u_t(x_j,t^n) + \\
&  -\vec{v}_1^tD^2u(x_j,t^n)\vec{v}_1 -\vec{v}_2^tD^2u(x_j,t^n)\vec{v}_2 + \\
& +O(\frac{\Delta x^r}{\Delta t}) + O(\frac{\Delta x^{q-s}}{\Delta t^{\frac{1}{2}}}) + O(\Delta t)
\end{split}
\]
questa per $(\Delta t,\Delta x)\to O$ e $(x_j,t^n)\to(x,t)$ risulta soddisfatta, nella forma classica, se $\Delta x = o(\Delta t^{\frac{1}{r}})$ e $\Delta x^{q-s}=o(\Delta t^{\frac{1}{2}})$
