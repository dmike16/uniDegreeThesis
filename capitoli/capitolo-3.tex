\chapter{Costruzione ed analisi dello schema Semi-Lagrangiano}
In questo capitolo costruiremo ed analizeremo uno schema semi-lagrangiano per l'equatione di curvatura media di evoluzione di una superficie chiusa in $\mathbb{R}^3$.
Ricordiamo la nostra equazione :
\begin{equation}
\label{eq:cp3-01}
\begin{cases}
u_t = \Div\left(\frac{Du}{|Du|}\right)|Du|\quad (x_1,x_2,x_3)\in\mathbb{R}^3,t\in(0,T) \\
u(x,0) = u_o(x) \qquad \quad \mathbb{R}^3\times\{t = 0\}
\end{cases}
\end{equation}
\section{Costruzione dello schema MCM}
Partiamo dall'equazione \eqref{eq:cp3-01} e operiamo alcune manipolazioni algebriche, al fine di scrivere meglio il termine del flusso. 
\begin{proposizione}
Vale la seguente serie di uguaglianze:
\[
\Div\left(\frac{Du}{|Du|}\right)|Du|=\trace\left(\left(I-\frac{DuDu^t}{|Du|^2}\right)D^2u\right)=\vec{v}_1^tD^2u\vec{v}_1+\vec{v}_2^tD^2u\vec{v}_2.
\]
con $\vec{v}_1$ e $\vec{v}_2$ autovettori della matrice $P=I-\frac{DuDu^t}{|Du|^2}$; rispettivamente :
\[
\vec{v_1}=
\begin{bmatrix}
\frac{-u_{x_3}}{\sqrt{u_{x_1}^2+u_{x_3}^2}} \\
0 \\
\frac{u_{x_1}}{\sqrt{u_{x_1}^2+u_{x_3}^2}}
\end{bmatrix}
\quad
\vec{v_2}=\frac{1}{|Du|}
\begin{bmatrix}
\frac{-u_{x_1}u_{x_2}}{\sqrt{u_{x_1}^2+u_{x_3}^2}} \\
\sqrt{u_{x_1}^2+u_{x_3}^2} \\
\frac{-u_{x_2}u_{x_3}}{\sqrt{u_{x_1}^2+u_{x_3}^2}}
\end{bmatrix}
\]
\end{proposizione}
\begin{proof}
Iniziamo con la prima uguaglianza, svolgendo esplicitamente $\Div\left(\frac{Du}{|Du|}\right)$ (senza riportare tutti i conti) otteniamo :
\[
|Du|\Div\left(\frac{Du}{|Du|}\right)=|Du|\left(\frac{\Div(Du)}{|Du|} -(Du)^tD^2u\frac{Du}{|Du|^3}\right) =
\]
\[
= \Delta u - \left<D^2u\frac{Du}{|Du|},\frac{Du}{|Du|}\right> .
\]
Ora ricordando che $\Delta u$ rappresenta il laplaciano della funzione $u$ e osservando che :
\[
\left<D^2u\frac{Du}{|Du|},\frac{Du}{|Du|}\right> = \left<\frac{1}{|Du|}(u_{x_1}(D^2u)^1,u_{x_2}(D^2u)^2,u_{x_3}(D^2u)^3)^t,\frac{Du}{|Du|}\right> =
\]
\[
=\frac{1}{|Du|^2}(u_{x_1}^2(D^2u)^1+u_{x_2}^2(D^2u)^2+u_{x_3}^2(D^2u)^3 = \trace\left(\frac{DuDu^t}{|Du|^2}D^2u\right),
\]
avendo indicato con $(D^2u)^n$ l'$n$-esima colonna della matrice delle derivate seconde della funzione $u$, giungiamo all'uguaglianza cercata:
\[
\Div\left(\frac{Du}{|Du|}\right)|Du| = \trace\left(\left(I-{\frac{DuDu^t}{|Du|^2}}\right)D^2u\right).
\] 
Ora occupiamoni della seconda uguaglianza. Iniziamo co fare alcune osservazioni sulla matrice $P=I-\frac{DuDu^t}{|Du|^2}$. Questa è una matrice di proiezione, $3x3$ a valori in $\mathbb{R}$, simmetrica ($P^t=P$) e idempotente ($P^2=P$), ammette due autovettori $\vec{v}_1$,$\vec{v}_2$ e vale la seguente decomposizione
\[
P=\sum_{i=1}^2\vec{v}_i\vec{v}_i^t=\sigma\sigma^t\quad\text{ con } \sigma=[\vec{v_1},\vec{v_2}],
\]
con $\sigma$ matrice $3x2$ che ha come colonne gli autovettori di $P$. Detto ciò abbiamo: 
\[
\trace\left(\left(I-{\frac{DuDu^t}{|Du|^2}}\right)D^2u\right)=\trace\left(\sigma\sigma^tD^2\right)=\trace\left([\vec{v}_1,\vec{v}_2][\vec{v}_1,\vec{v}_2]^tD^2u\right)=
\]
\[
=\trace\left((\vec{v}_1\vec{v}_1^t+\vec{v}_2\vec{v}_2^t)D^2u\right)=\trace\left(\vec{v}_1\vec{v}_1^tD^2u\right)+\trace\left(\vec{v}_2\vec{v}_2^tD^2u\right)
\]
Ricordando i calcoli fatti nella dimostrazione della prima uguaglianza possiamo scrivere:
\[
\begin{split}
&\trace\left(\vec{v}_1\vec{v}_1^tD^2u\right)+\trace\left(\vec{v}_2\vec{v}_2^tD^2u\right)=\left<D^2u\vec{v}_1,\vec{v}_1\right>+\left<D^u\vec{v}_2,\vec{v}_2\right>=\\
& =\vec{v}_1^tD^2u\vec{v}_1+\vec{v}_2^tD^2u\vec{v}_2.
\end{split}
\]
\end{proof}

Riformuliamo l'equazione \eqref{eq:cp3-01} nel seguente modo :
\begin{equation}
u_t=\vec{v}_1^tD^2u\vec{v}_1 + \vec{v}_2^tD^2u\vec{v}_2,
\end{equation}
e costruiamoci il metodo.
Approssimiano la derivata temporale con le differenze finite in avanti, mentre per il termine del flusso utilizziamo le derivate direzionali scegliendo come incremento $\sqrt{2\Delta t}$.
Quindi nel caso in cui $Du \ne 0$ e $u_{x_1}^2+u_{x_3}^2\ne 0$ otteniamo :
\[
\begin{split}
&\frac{u(x,t+\Delta t)-u(x,t)}{\Delta t} = \frac{1}{4\Delta t}\bigl[u(x+\sqrt{2\Delta t}(\vec{v}_1+\vec{v}_2),t) +\\
& +u(x+\sqrt{2\Delta t}(-\vec{v}_1+\vec{v}_2),t) + u(x+\sqrt{2\Delta t}(\vec{v}_1-\vec{v}_2),t) \\
& + u(x+\sqrt{2\Delta t}(-\vec{v}_1-\vec{v}_2),t) - 4u(x,t)\bigr] + O(\Delta t)
\end{split}
\]

Osservando tale approssimazione, possiamo notare che i punti  \\
$(x + \sqrt{2\Delta t}(\pm\vec{v}_1 \pm\vec{v}_2))$ non necessariamente saranno i nodi della griglia, quindi è necessario introdurre l'operatore di interpolazione $I[.]$.
Per completare la costruzione consideriamo una griglia spaziale $x_j=(j_1\Delta c,j_2\Delta x)$ con $j=(j_1,j_2)$ multi-indice in $\mathbb{Z}^2$ ed una griglia temporale $t^n=n\Delta t$ con $n\in\mathbb{N}$.
Indichiamo con $u_j^n$ l'approssimazione di $u$ nel punto della griglia $(x_j,t^n)$ e con $D_j^n$ l'approssimazione del gradiente di $u$ nel suddetto punto. Quindi il nostro schema semi-lagrangiano risulta :
\begin{equation}
u_j^{n+1} = \frac{1}{4}\sum_{\pm}I[u^n](x_j+\sqrt{2\Delta t}(\pm\vec{v}_1\pm\vec{v}_2))
\end{equation} 

\section{Consistenza dello schema}
