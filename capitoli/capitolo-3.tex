\chapter{Costruzione ed analisi dello schema Semi-Lagrangiano}
\label{cp:cp2-00}
In questo capitolo costruiremo ed analizzeremo uno schema Semi-Lagrangiano per l'equazione di curvatura media in $\Omega\subseteq\mathbb{R}^3$:
\begin{equation}
\label{eq:cp3-01}
\begin{cases}
u_t = \Div\left(\frac{Du}{|Du|}\right)|Du|\quad (x_1,x_2,x_3)\in\Omega,t\in(0,T) \\
u(x,0) = u_0(x) \qquad \quad\quad\Omega\times\{t = 0\}
\end{cases}
\end{equation}
con $Du$ il gradiente di $u$, e per la sua variante che preserva il
volume:
\begin{equation}
\label{eq:cp3-01-vp}
\begin{cases}
 u_t=|Du|\Div{\left(\frac{Du}{|Du|}\right)}-\frac{\iint\Div{\left(\frac{Du}{|Du|}\right)}d\mu}{3V_0}x^tDu\quad (x_1,x_2,x_3)\in\Omega,t\in(0,T) \\
u(x,0) = u_0(x) \qquad\qquad\qquad\qquad\qquad\qquad\quad\Omega\times\{t = 0\}
\end{cases}
\end{equation}
%%%%%%%%%%%%%%%%%%%%%%%%%%%%%%%%%%%%%%%%%%%%%%%%%%%%%%%%%%%%%%%%%%%%%%%%%%%%%
% Section 3.1 Costruzione dello schema
%
%%%%%%%%%%%%%%%%%%%%%%%%%%%%%%%%%%%%%%%%%%%%%%%%%%%%%%%%%%%%%%%%%%%%%%%%%%%%%%%%
\section{Costruzione dello schema MCM}
\label{sec:cp3-sc3-1}
Partiamo dall'equazione \eqref{eq:cp3-01} e operiamo alcune manipolazioni algebriche, al fine di scrivere diversamente il termine del flusso. 
\begin{proposizione}
\label{prop:cp3-01}
Vale la seguente serie di uguaglianze:
\[
\Div\left(\frac{Du}{|Du|}\right)|Du|=\trace\left(\left(I-\frac{DuDu^t}{|Du|^2}\right)D^2u\right)=\vec{v}_1^tD^2u\vec{v}_1+\vec{v}_2^tD^2u\vec{v}_2.
\]
con $\vec{v}_1$ e $\vec{v}_2$ autovettori della matrice $P=I-\frac{DuDu^t}{|Du|^2}$, definiti da :
\[
\vec{v_1}=
\begin{bmatrix}
\frac{-u_{x_3}}{\sqrt{u_{x_1}^2+u_{x_3}^2}} \\
0 \\
\frac{u_{x_1}}{\sqrt{u_{x_1}^2+u_{x_3}^2}}
\end{bmatrix}
\quad
\vec{v_2}=\frac{1}{|Du|}
\begin{bmatrix}
\frac{-u_{x_1}u_{x_2}}{\sqrt{u_{x_1}^2+u_{x_3}^2}} \\
\sqrt{u_{x_1}^2+u_{x_3}^2} \\
\frac{-u_{x_2}u_{x_3}}{\sqrt{u_{x_1}^2+u_{x_3}^2}}
\end{bmatrix}
\]
\end{proposizione}
\begin{proof}
Iniziamo con la prima uguaglianza, svolgendo esplicitamente $\Div\left(\frac{Du}{|Du|}\right)$ (senza riportare tutti i conti) otteniamo :
\[
|Du|\Div\left(\frac{Du}{|Du|}\right)=|Du|\left(\frac{\Div(Du)}{|Du|} -(Du)^tD^2u\frac{Du}{|Du|^3}\right) =
\]
\[
= \Delta u - \left<D^2u\frac{Du}{|Du|},\frac{Du}{|Du|}\right> .
\]
Ora ricordando che $\Delta u$ rappresenta il laplaciano della funzione $u$ e osservando che :
\[
\left<D^2u\frac{Du}{|Du|},\frac{Du}{|Du|}\right> = \left<\frac{1}{|Du|}(u_{x_1}(D^2u)^1+u_{x_2}(D^2u)^2+u_{x_3}(D^2u)^3),\frac{Du}{|Du|}\right> =
\]
\[
=\frac{1}{|Du|^2}(u_{x_1}<(D^2u)^1,Du>+u_{x_2}<(D^2u)^2,Du>+u_{x_3}<(D^2u)^3,Du> = 
\]
\[
=\trace\left(\frac{DuDu^t}{|Du|^2}D^2u\right),
\]
avendo indicato con $(D^2u)^n$ l'$n$-esima colonna della matrice delle derivate seconde della funzione $u$, giungiamo all'uguaglianza cercata:
\[
\Div\left(\frac{Du}{|Du|}\right)|Du| = \trace\left(\left(I-{\frac{DuDu^t}{|Du|^2}}\right)D^2u\right).
\] 
Ora occupiamoni della seconda uguaglianza. Iniziamo con il fare alcune osservazioni sulla matrice $P=I-\frac{DuDu^t}{|Du|^2}$. Questa è una matrice di proiezione, $3\times3$ a valori in $\mathbb{R}$, simmetrica ($P^t=P$) e idempotente ($P^2=P$), ammette due autovettori $\vec{v}_1$,$\vec{v}_2$ e vale la seguente decomposizione (vedi formula \eqref{eq:cp1-1-05})
\[
P=\sum_{i=1}^2\vec{v}_i\vec{v}_i^t=\sigma\sigma^t\quad\text{ con } \sigma=[\vec{v_1},\vec{v_2}],
\]
con $\sigma$ matrice $3\times2$ che ha come colonne gli autovettori di $P$. Detto ciò abbiamo: 
\[
\trace\left(\left(I-{\frac{DuDu^t}{|Du|^2}}\right)D^2u\right)=\trace\left(\sigma\sigma^tD^2\right)=\trace\left([\vec{v}_1,\vec{v}_2][\vec{v}_1,\vec{v}_2]^tD^2u\right)=
\]
\[
=\trace\left((\vec{v}_1\vec{v}_1^t+\vec{v}_2\vec{v}_2^t)D^2u\right)=\trace\left(\vec{v}_1\vec{v}_1^tD^2u\right)+\trace\left(\vec{v}_2\vec{v}_2^tD^2u\right)
\]
Ricordando i calcoli fatti nella dimostrazione della prima uguaglianza possiamo scrivere:
\[
\begin{split}
&\trace\left(\vec{v}_1\vec{v}_1^tD^2u\right)+\trace\left(\vec{v}_2\vec{v}_2^tD^2u\right)=\left<D^2u\vec{v}_1,\vec{v}_1\right>+\left<D^2u\vec{v}_2,\vec{v}_2\right>=\\
& =\vec{v}_1^tD^2u\vec{v}_1+\vec{v}_2^tD^2u\vec{v}_2.
\end{split}
\]
\end{proof}

Riformuliamo l'equazione \eqref{eq:cp3-01} nel seguente modo :
\begin{equation}
u_t=\vec{v}_1^tD^2u\vec{v}_1 + \vec{v}_2^tD^2u\vec{v}_2,
\end{equation}
e costruiamoci il metodo.
Approssimiano la derivata temporale con le differenze finite in avanti, mentre per il termine del flusso utilizziamo le derivate direzionali scegliendo come incremento $\sqrt{2\Delta t}$.
Quindi nel caso in cui $Du \ne 0$ e $u_{x_1}^2+u_{x_3}^2\ne 0$ otteniamo :
\[
\begin{split}
&\frac{u(x,t+\Delta t)-u(x,t)}{\Delta t} +O(\Delta t)= \frac{1}{4\Delta t}\bigl[u(x+\sqrt{2\Delta t}(\vec{v}_1+\vec{v}_2),t) +\\
& +u(x+\sqrt{2\Delta t}(-\vec{v}_1+\vec{v}_2),t) + u(x+\sqrt{2\Delta t}(\vec{v}_1-\vec{v}_2),t) \\
& + u(x+\sqrt{2\Delta t}(-\vec{v}_1-\vec{v}_2),t) - 4u(x,t)\bigr] + O(\Delta t)
\end{split}
\]
Osservando tale approssimazione, possiamo notare che i punti $(x + \sqrt{2\Delta t}(\pm\vec{v}_1 \pm\vec{v}_2))$ non necessariamente saranno i nodi della griglia, quindi è necessario introdurre un operatore di interpolazione $I[\cdot]$.
Per completare la costruzione consideriamo una griglia spaziale $x_j=(j_1\Delta x,j_2\Delta x,j_3\Delta x)$ con $j=(j_1,j_2,j_3)$ multi-indice in $\mathbb{Z}^3$ ed una griglia temporale $t^n=n\Delta t$ con $n\in\mathbb{N}$.
Indichiamo con $u_j^n$ l'approssimazione di $u$ nel punto della griglia $(x_j,t^n)$ e con $D_j^n$ l'approssimazione del gradiente di $u$ nel suddetto punto. Quindi il nostro schema Semi-Lagrangiano risulta :
\begin{equation}
\label{eq:cp3-02}
u_j^{n+1} = \frac{1}{4}\sum_{+,-}I[u^n](x_j+\sqrt{2\Delta t}(\pm\vec{v}_1\pm\vec{v}_2)).
\end{equation} 
\begin{osservazione}
Questa costruzione non è applicabile nel caso in cui 
\begin{equation}
\label{eq:cp3-01-add}
Du = 0 \text{ o } u_{x_1}^2+u_{x_3}^2 = 0. 
\end{equation}
Quindi per distinguere i due casi dal punto di vista numerico, introduciamo una soglia $s$ ed una costante $C$. In modo tale che il verificarsi del caso \eqref{eq:cp3-01-add} è rappresentato numericamente dalle disequazioni
\begin{equation}
\label{eq:cp3-sc3-1-01}
|D_j^n|\le C\Delta x^s\text{ e }|(D_j^n)_1^2+(D_j^n)_3^2|\le C\Delta x^s.
\end{equation}
\end{osservazione}
Nel caso in cui siamo sotto la soglia, possiamo procedere in due modi diversi.
\begin{enumerate}
  \item Approssimiamo la nostra soluzione nel punto $(x_j,t^{n+1})$, con la media dei primi vicini:
    \begin{equation}
      \label{eq:cp3-03-add}
      u_j^{n+1}=\frac{1}{6}\sum_{i\in\delta(j)}u_i^n,
    \end{equation}
dove $\delta(j)=\left\{(j_1\pm 1,j_2,j_3),(j_1,j_2\pm 1,j_3),(j_1,j_2,j_3\pm 1)\right\}$. Questa rappresentazione, non è altro che un'approssimazione dell'equazione del calore $u_t=\varepsilon\Delta u$ in $\mathbb{R}^3\times(0,T)$. Difatti usando Eulero in avanti per la derivata temporale e differenze finite centrate per quella spaziale abbiamo:
\[
\begin{aligned}
  \frac{u_j^{n+1}-u_j^n}{\Delta t}=\varepsilon&\frac{u_{j_1 + 1,j_2,j_3}^n +
    u_{j_1 - 1,j_2,j_3}^n + u_{j_1,j_2 + 1,j_3}^n + u_{j_1,j_2 - 1,j_3}^n}{\Delta x^2} + \\
    &+ \frac{u_{j_1,j_2,j_3 + 1}^n + u_{j_1,j_2,j_3 - 1}^n - 6u_j^n}{\Delta x^2},
\end{aligned}
\]
scegliendo $\varepsilon=\frac{\Delta x^2}{6\Delta t}$ otteniamo la \eqref{eq:cp3-03-add}.
Tale scelta è giustificata dal fatto che, come vedremo nella sezione
successiva, il moto per curvature media è consistente con
l'approssimazione dell'equazione del calore a patto di adottare nella
definizione di soluzioni viscose il flusso \eqref{eq:cp2-14} definito da
\emph{Crandall} in \cite[][]{crand:lion}.

  \item Utilizziamo l'approssimazione dell'equazione $u_t=\frac{1}{2}\Delta u$:
    \begin{equation}
      \label{eq:cp-04-add}
      u_j^{n+1}=u_j^n +\frac{1}{2}\frac{\Delta t}{\Delta x^2}\left(6u_j^n - \sum_{i\in\delta(j)}u_i^n\right).
    \end{equation}
Tale scelta è giustificata, in quanto il moto per curvatura media
risulta essere consistente con l'equazione $u_t=\frac{1}{2}\Delta u$,
prendendo nella definizione di soluzione viscosa il flusso 
\eqref{eq:cp2-13} come in \cite[][]{fed:drag}. L'unico problema è
che la \eqref{eq:cp-04-add} richiede una $\mathit{CFL}$ parabolica, che non è presente nel nostro schema essendo \emph{semi-lagrangiano}; un modo per aggirare il problema è quello di usare uno schema implicito nell'approssimazione di  $u_t=\frac{1}{2}\Delta u$.
\end{enumerate}
Scegliendo, ad esempio, la prima alternativa il nostro schema $\mathit{MCM}$ completo diventa:
\begin{equation}
  \label{eq:mcm-schema}
  u_j^{n+1}=
  \left\{
  \begin{aligned}
    &   &\scriptstyle|D_j^n|>&\scriptstyle C\Delta x^s\\ 
    &\frac{1}{4}\sum_{+,-}I[u^n](x_j+\sqrt{2\Delta t}(\pm\vec{v}_{j}^{n,1}\pm\vec{v}_{j}^{n,2})),  &\scriptstyle\mathbf{and} \\
    &  &\scriptstyle|(D_j^n)_1^2+(D_j^n)_3^2|&\scriptstyle >C\Delta x^2 \\
    &  &  \\
    &  &\scriptstyle|D_j^n|\leq&\scriptstyle C\Delta x^s \\
     & \frac{1}{6}\sum_{i\in\delta(j)}u_i^n,   &\scriptstyle\mathbf{or} \\
  &  &\scriptstyle|(D_j^n)_1^2+(D_j^n)_3^2| &\scriptstyle\leq C\Delta x^s.
  \end{aligned}   
  \right.
\end{equation}
dove
\begin{equation} 
  \label{eq:cp3-sc3-1-end}
  \vec{v}_{j}^{n,1}=\vec{v}_1(D_j^n), \quad\vec{v}_j^{n,2}=\vec{v}_2(D_j^n).
\end{equation} 

%%%%%%%%%%%%%%%%%%%%%%%%%%%%%%%%%%%%%%%%%%%%%%%%%%%%%%%%%%%%%%%%%%%%%%%%%%%%%%
%              Section 3.2 Consistenza dello schema
%
%%%%%%%%%%%%%%%%%%%%%%%%%%%%%%%%%%%%%%%%%%%%%%%%%%%%%%%%%%%%%%%%%%%%%%%%%%%%%%
\section{Consistenza dello schema MCM}
\label{sec:cp3-sc2}
Occupiamoci ora dell'analisi della consitenza dello schema. Prima di tutto ricordiamo che , per un dato schema numerico $S(x_j,u_j^n,Du_j^n)$ , l'errore di troncamento locale (o LTE) $\tau_j^n$ è l'errore che si genera pretendendo che la soluzione esatta soddisfi lo schema numerico :
\[
\tau_j^n=\frac{u(x_j,t^{n+1})-S(x_j,u(x_j,t^n),Du(x_j,t^n))}{\Delta t}
\]
Si definisce anche errore di troncamento totale la quantità
\[
\tau(\Delta t,\Delta x) := \max_{j,n} |\tau_j^n|.
\]
In generale un metodo si dirà consistente quando il suo LTE è infinitesimo rispetto all'incremento. Mentre si dirà consitente di ordine $p$ in tempo e $q$ in spazio se :
\[
\tau(\Delta t,\Delta x) = O(\Delta t^p +\Delta x^q).
\] 
Nel nostro caso, in quanto l'equazione che approssimiamo è singolare quando il gradiente  $Du$ si annulla oppure quando $u_{x_1}^2+u_{x_3}^2=0$, è necessario introdurre una definizione di consistenza più generale.
\begin{definizione}
Sia $(\Delta x_m,\Delta t_m)$ una seguenza generica di parametri di discretizzazione e sia $(x_{j_m},t^{n_m})\in G\times\left\{0,\dots,\Delta t_mN\right\}$ una seguenza generica di nodi, tale che per $m\to\infty$,
\[
(\Delta x_m,\Delta t_m)\to0 \quad\text{ e } (x_{j_m},t^{n_m})\to(x,t).
\]
E sia inoltre $\phi\in C^{\infty}(\mathbb{R}^3\times(0,t])$ e $\phi^{n_m-1}\equiv(\phi(x_{j_m},t^{n_m-1})_{x_{j_m}}$. Allora lo schema $S$ è detto consistente se
\begin{equation}
\label{eq:cp3-03}
\begin{split}
&\phi_t(x,t)+F_*(D\phi(x,t),D^2\phi(x,t))\le \\
&\le\liminf_{m\to\infty}\frac{\phi(x_{j_m},t^{n_m})-S(x_{j_m},\phi_{j_m}^{n_m-1},D\phi_{j_m}^{n_m-1})}{\Delta t_m}\le \\
&\le\limsup_{m\to\infty}\frac{\phi(x_{j_m},t^{n_m})-S(x_{j_m},\phi_{j_m}^{n_m-1},D\phi_{j_m}^{n_m-1})}{\Delta t_m}\le \\
&\le\phi_t(x,t)+F^*(D\phi(x,t),D^2\phi(x,t)),
\end{split}
\end{equation}
con $F_*$ e $F^*$ rispettivamente inviluppo semicontinuo inferioremnte
e speriormente di $F$ definiti in \eqref{eq:cp2-03-add}.
\end{definizione}
\begin{osservazione}
Osserviamo che se $F$ è continua allora il $\limsup$ ed il $\liminf$ nell'espressione \eqref{eq:cp3-03} coincidono, quindi la definizione si riduce a quella classica di consistenza.
\end{osservazione}

Indichiamo con $u(x,t)$ una soluzione regolare di \eqref{eq:cp3-01}, con $(x_{j_m},t^{n_m})$ una successione di nodi che tende a $(x,t)$ come sopra e con $u(x_{j_m},t^{n_m})$ il suo valore su tali nodi; per semplicità notazionale indicheremo questa seguenza omettendo il pedice $m$.
Prima di procedere con la consistenza, facciamo le seguenti ipotesi sull'errore commesso nell'approssimazione del gradiente e nell'interpolazione:
\begin{gather}
\label{eq:cp3-04}
||I[u^n](\cdot)-u(\cdot,t^n)||_{\infty}\le C_1\Delta x^r\quad\forall n\in\mathbb{N} \\
\label{eq:cp3-05}
|D_j[u^n]-Du(x_j,t^n)|\le C_2\Delta x^q\quad\forall n\in\mathbb{N}\quad\forall x_j, 
\end{gather}
con $C_1$ e $C_2$ costanti positive che dipendono così come gli ordini $r$ e $q$ dal metodo scelto per l'interpolazione e per l'approssimazioni del gradiente. Nel nostro caso abbiamo scelto entrambi i metodi di ordine $2$, quindi $r=q=2$.

%%%%%%%%%%%%%%%%%%%%%%%%%%%%%%%%%%%%%%%%%%%%%%%%%%%%%%%%%%%%%%%%%%%%%%%%%%%%%%
%                 SubSection 3.2.1 Caso DU != 0
%
%%%%%%%%%%%%%%%%%%%%%%%%%%%%%%%%%%%%%%%%%%%%%%%%%%%%%%%%%%%%%%%%%%%%%%%%%%%%%%
\subsection{Caso gradiente diverso da zero}
\label{subsec:cp3-sc2-1}
Consideriamo il caso in cui $Du(x,t)\ne 0$ e $u_{x_1}(x,t)^2+u_{x_3}(x,t)^2\ne 0$, vista la regolarità di $u$, esiste un intorno di $(x,t)$ in cui le due espressioni precedenti si mantengono vere, quindi almeno asintoticamente:
\begin{equation}
  \label{eq:cp3-sc21-01}
|D_j^n|\ge C\Delta x^s\text{ e }|(D_j^n)_1^2+((D_j^n)_3^2|\ge C\Delta x^s
\end{equation}
per cui possiamo applicare lo schema nella forma \eqref{eq:cp3-02}.

Abbiamo definito $\sigma=[\vec{v}_1,\vec{v}_2]$ quindi prendendo $\vec{b}$ in $\Pi=\{[1,1]^t,[-1,1]^t,$ $[1,-1]^t,[-1,-1]^t\}$ possiamo riscrivere il termine generico della sommatoria in \eqref{eq:cp3-02} come
\[
I[u^n](x_j+\sigma(D_j^n)\sqrt{2\Delta t}\vec{b}).
\]
Ora calcolando lo jacobiano di $\sigma$ per un qualsiasi vettore $p$ in $\mathbb{R}^3$ (senza riportare il conto) si dimostra che:
\[
\left|J_{\sigma}(p)\right|^2=\frac{1}{p_1^2+p_3^2},
\]
dove come norma è stata scelta quella di Frobenius. Quindi applicando $\sigma[\cdot]$ al gradiente e sfruttando il fatto che $(D_j^n)_1^2+(D_j^n)_3^2\ge C\Delta x^s$, possiamo concludere che la funzione $\sigma$ risulta lipschitziana con costante $L_{\sigma}\le\frac{1}{C\Delta x^s}$.
Inoltre, avendo preso la $u$ sufficientemente regolare, anche essa sarà lipschitiziana con costatne $L_u$ e con $|Du|< M$. Assumiamo anche che $|u_{tt}(x,t)|\le C$ in modo tale da limitare l'errore di troncamento per l'approssimazione temporale.
 Analizziamo il prima addendo della sommatoria nell'espressione \eqref{eq:cp3-02} e facciamo alcune manipolazioni algebriche
\begin{equation}
\label{eq:cp3-06}
\begin{split}
& I[u^n](x_j+\sigma(D_j^n)\sqrt{2\Delta t}\vec{b}) = I[u^n](x_j+\sigma(D_j^n)\sqrt{2\Delta t}\vec{b}) + \\
& -u(x_j+\sigma(D_j^n)\sqrt{2\Delta t}\vec{b},t^n) + u(x_j+\sigma(D_j^n)\sqrt{2\Delta t}\vec{b},t^n)+ \\
& - u(x_j+\sigma(Du(x_j,t^n))\sqrt{2\Delta t}\vec{b},t^n)+ u(x_j+\sigma(Du(x_j,t^n))\sqrt{2\Delta t}\vec{b},t^n).
\end{split}
\end{equation}
Stimiamo la parte destra di \eqref{eq:cp3-06}. Per la prima differenza applichiamo la stima dell'errore di interpolazione quindi
\begin{equation}
  \label{eq:cp3-07}
  \begin{split}
    |I[u^n](x_j+\sigma(D_j^n)\sqrt{2\Delta t}\vec{b}) &- u(x_j+\sigma(D_j^n)\sqrt{2\Delta t}\vec{b},t^n)|\le \\
    & \le C_1\Delta x^r,
\end{split}
\end{equation}
mentre per la seconda utilizzando la lipschitianità di $u$ , di $\sigma$ e la stima dell'errore di aprrossimazione del gradiente ottendo:
\begin{equation}
  \label{eq:cp3-08}
  \begin{split}
    |u(x_j+\sigma(D_j^n)\sqrt{2\Delta t}\vec{b},t^n) &- u(x_j+\sigma(Du(x_j,t^n))\sqrt{2\Delta t}\vec{b},t^n)|\le \\
    & \le M\sqrt{2\Delta t}\frac{C_2}{C}\Delta x^{q-s}.
  \end{split}
\end{equation}
Usando \eqref{eq:cp3-07} e \eqref{eq:cp3-08} in \eqref{eq:cp3-06} otteniamo
\[
\begin{split}
&I[u^n](x_j+\sigma(D_j^n)\sqrt{2\Delta t}\vec{b})=u(x_j+\sigma(Du(x_j,t^n))\sqrt{2\Delta t}\vec{b},t^n)+O(\Delta x^r)+ \\
& +O(\Delta x^{q-s}\sqrt{2\Delta t}).
\end{split}
\]
Sviluppiamo ora $u(x_j+\sigma(Du(x_j,t^n))\sqrt{2\Delta t}\vec{b},t^n)$ con una serie di Taylor del terzo ordine in un intorno del punto $x_j$:
\begin{equation}
\label{eq:cp3-09}
\begin{split}
&u(x_j+\sigma(Du(x_j,t^n))\sqrt{2\Delta t}\vec{b},t^n) = u(x_j,t^n)+ \\
&+\left<Du(x_j,t^n),\sigma(Du(x_j,t^n))\vec{b}\right>\sqrt{2\Delta t} +\\
&+\left<D^2u(x_j,t^n)\sigma(Du(x_j,t^n))\vec{b},\sigma(Du(x_j,t^n))\vec{b}\right>\Delta t + \\
&+T_3(\Delta t^{\frac{3}{2}}\sigma(Du(x_j,t^n))\vec{b},D^3u(x_j,t^n))+O(\Delta t^2), 
\end{split}
\end{equation}
non abbiamo esplicitato i termini del terzo ordine in quanto , mettendo tutto insieme, si eliminano cosi come i termini del primo ordine data la simmetria dei $4$ punti $x_j+\sigma(Du(x_j,t^n))\vec{b}\sqrt{2\Delta t}$ con $b\in\Pi$. Alla fine otteniamo che:
\begin{equation}
\label{eq:cp3-010}
\begin{split}
& S(x_j,u(x_j,t^n),Du(x_j,t^n)) = \\
& = u(x_j,t^n) + \vec{v}_1^tD^2u(x_j,t^n)\vec{v}_1 + \vec{v}_2^tD^2u(x_j,t^n)\vec{v}_2 + \\
& + O(\Delta x^r) + O(\Delta x^{q-s}\Delta t^{\frac{1}{2}}) + O(\Delta t^2).
\end{split}
\end{equation}
Ora sviluppando  anche il termine $u(x_j,t^{n+1})$ e applicando la definizione generale di consistenza \eqref{eq:cp3-03} otteniamo 
\[
\begin{split}
&\frac{u(x_j,t^{n+1})-S(x_j,u(x_j,t^n),Du(x_j,t^n))}{\Delta t}= u_t(x_j,t^n) + \\
&  -\vec{v}_1^tD^2u(x_j,t^n)\vec{v}_1 -\vec{v}_2^tD^2u(x_j,t^n)\vec{v}_2 + \\
& +O(\frac{\Delta x^r}{\Delta t}) + O(\frac{\Delta x^{q-s}}{\Delta t^{\frac{1}{2}}}) + O(\Delta t)
\end{split}
\]
questa per $(\Delta t,\Delta x)\to 0$ e $(x_j,t^n)\to(x,t)$ risulta
soddisfatta, nella forma classica, se $\Delta x = o(\Delta
t^{\frac{1}{r}})$ e $\Delta x^{q-s}=o(\Delta t^{\frac{1}{2}})$

\begin{osservazione}
Nel caso $r=q=2$ ed $s=1$ si ha $\Delta x=o(\Delta
t^{\frac{1}{2}})$. Ponendo $\Delta t = \Delta x^{\alpha}$, affinchè il
metodo sia consistente, $\alpha$ deve verificare
$1-\frac{\alpha}{2}>0$ e quindi $\alpha < 2$. Per cui lo schema, pur
essendo esplicito, permette passi in tempi lunghi e non richiede di
verificare la condizione di tipo CFL parabolica, tipica degli schemi
espliciti alle differenze finiti per equazioni paraboliche. Tale
condizione è molto restrittiva sulla lunghezza dei passi in tempo,
perchè richiede $\Delta t=O(\Delta x^2)$.
\end{osservazione}
%%%%%%%%%%%%%%%%%%%%%%%%%%%%%%%%%%%%%%%%%%%%%%%%%%%%%%%%%%%%%%%%%%%%%%%%%%%%%%
%                   SubSection 3.2.2 Caso DU = 0
%
%%%%%%%%%%%%%%%%%%%%%%%%%%%%%%%%%%%%%%%%%%%%%%%%%%%%%%%%%%%%%%%%%%%%%%%%%%%%%%
\subsection{Caso gradiente nullo}

Nel caso in cui il gradiente è nullo le successioni convergenti a $(x,t)$ possono avere sottosuccessioni di nodi sia sotto che sopra la soglia, quindi dobbiamo distinguere due casi:
\begin{itemize}
  \item \textsf{Caso} $|Du(x,t)|=0,\mathrm{|D_j^n|\ge C\Delta x^s\land|(D_j^n)_1^2+((D_j^n)_3^2|\ge C\Delta x^s}$. In questo caso ci troviamo sopra la soglia. Quindi possiamo utilizzare lo schema nella forma \eqref{eq:cp3-02} e ripetere i passaggi precedenti eliminado qualsiasi riferimento a $\sigma(Du(x_t,t^n))$, in quanto non possiamo assicurare la convergenza di $\sigma(D_j^n)$ a $\sigma(Du(x_t,t^n))$. In particolare abbiamo:
\[
  \begin{split}
    & I[u^n](x_j+\sigma(D_j^n)\sqrt{2\Delta t}\vec{b}) = I[u^n](x_j+\sigma(D_j^n)\sqrt{2\Delta t}\vec{b}) - \\
    & +u(x_j+\sigma(D_j^n)\sqrt{2\Delta t}\vec{b},t^n) + u(x_j+\sigma(D_j^n)\sqrt{2\Delta t}\vec{b},t^n),
  \end{split}
\]
utilizzando l'errore di interpolazione abbiamo
\[
I[u^n](x_j+\sigma(D_j^n)\sqrt{2\Delta t}\vec{b})=u(x_j+\sigma(D_j^n)\sqrt{2\Delta t}\vec{b},t^n) + O(\Delta x^r).
\]
Ripetendo questi passaggi al variare di $\vec{b}\in\Pi$ giungiamo a:
\[
 \begin{split}
 S(x_j,u(x_j,t^n),Du(x_j,t^n))&=\frac{1}{4}\left(\sum_{\vec{b}\in\Pi}u(x_j+\sigma(D_j^n)\sqrt{2\Delta t}\vec{b},t^n)\right)+\\
  &+ O(\Delta x^r).
 \end{split}
\]
Esprimendo il valore di $u$ nei punti $x_j+\sigma(D_j^n)\sqrt{2\Delta t}\vec{b}$ con una serie di Taylor del terzo ordine e cancellando i termini simmetrici otteniamo
\begin{equation}
  \label{eq:cp3-11}
  \begin{split}
    &S(x_j,u(x_j,t^n),Du(x_j,t^n)) = u(x_j,t^n) + \\
    &\Delta t\left(\vec{v}_1^tD^2u(x_j,t^n)\vec{v}_1 + \vec{v}_2^tD^2u(x_j,t^n)\vec{v}_2\right) + O(\Delta t^2) + O(\Delta x^r).
   \end{split}
\end{equation}
Inserendo quest'ultima espressione nella definizione di consistenza generalizzata risulta che
\[
\begin{split}
  &\frac{u(x_j,t^{n+1})-S(x_j,u(x_j,t^n),Du(x_j,t^n))}{\Delta t}= \frac{u(x_j,t^{n+1})-u(x_j,t^n)}{\Delta t} + \\
  &  -\vec{v}_1^tD^2u(x_j,t^n)\vec{v}_1 -\vec{v}_2^tD^2u(x_j,t^n)\vec{v}_2  
   + O(\Delta t) + O(\frac{\Delta x^r}{\Delta t}).
\end{split}
\]
Quindi per  $(\Delta t,\Delta x)\to 0$ e $(x_j,t^n)\to(x,t)$ sotto la condizione che $\Delta x^r = o(\Delta t)$ la \eqref{eq:cp3-03} risulta soddisfatta, in quanto il termine $\vec{v}_1^tD^2u(x_j,t^n)\vec{v}_1 +\vec{v}_2^tD^2u(x_j,t^n)\vec{v}_2 $ è sempre compreso tra il suo lim inf e lim sup :
\[
\begin{split}
  &F_*(Du(x,t),D^2u(x,t))\le-\vec{v}_1^tD^2u(x_j,t^n)\vec{v}_1
  -\vec{v}_2^tD^2u(x_j,t^n)\vec{v}_2\le \\
  &\le F^*(Du(x,t),D^2u(x,t))
\end{split}
\]
con $F(Du(x,t),D^2u(x,t))=-\vec{v}_1^tD^2u(x_j,t^n)\vec{v}_1 -\vec{v}_2^tD^2u(x_j,t^n)\vec{v}_2$.
  \item \textsf{Caso} $|Du(x,t)|=0,\mathrm{|D_j^n|< C\Delta x^s\lor |(D_j^n)_1^2+((D_j^n)_3^2|< C\Delta x^s}$. Tale disuguaglianze implicano che ci troviamo sotto la soglia. Per cui utilizziamo, ad esempio, lo schema nella forma \eqref{eq:cp3-03-add}, che rappresenta una discretizzazione dell'equazione del calore:
\[
S(x_j,u(x_j,t^n),Du(x_j,t^n))=\frac{1}{6}\sum_{i\in\delta(j)}u(x_i,t^n).
\]
per cui scrivendo la serie si Taylor del terzo ordine per $u$ nei punti $x\pm e_i\Delta x$, con $e_i$ vettori della base canonica di $\mathbb{R}^3$, otteniamo:
\[
S(x_j,u(x_j,t^n),Du(x_j,t^n))=u(x_j,t^n) + \varepsilon\Delta t\Delta u + O(\Delta t^2).
\]
La precedente uguaglianza implica che
\[
\begin{split}
  &\frac{u(x_j,t^{n+1})-S(x_j,u(x_j,t^n),Du(x_j,t^n))}{\Delta t}= \frac{u(x_j,t^{n+1})-u(x_j,t^n)}{\Delta t} + \\
  &  -\varepsilon\Delta u(x_j,t^n) + O(\Delta t),
\end{split}
\]
sotto la condizione che $\Delta x^2=o(\Delta t)$ abbiamo che $\varepsilon\to 0$ per $\Delta t\to 0$, quindi la condizione di consistenza generalizzata che si traduce nella seguente diseguaglianza
\[
\underline{F}(Du(x,t),D^2u(x,t))\le\varepsilon\Delta u(x_j,t^n)\le \overline{F}(Du(x,t),D^2u(x,t)),
\]
con $\underline{F}$ e $\overline{F}$ definite come in (\hyperref[eq:cp2-14]{\ref{eq:cp2-14}\ped{$cl$}}), risulta soddisfatta.  
\end{itemize}
%%%%%%%%%%%%%%%%%%%%%%%%%%%%%%%%%%%%%%%%%%%%%%%%%%%%%%%%%%%%%%%%%%%%%%%%%%%%%%
%                 Section 3.3
%
%%%%%%%%%%%%%%%%%%%%%%%%%%%%%%%%%%%%%%%%%%%%%%%%%%%%%%%%%%%%%%%%%%%%%%%%%%%%%%
\section{Schema MCM \emph{volume preserving}}
\label{sec:cp3-sc3-3}

In questa sezione, costruiremo uno schema semi-lagrangiano per
\eqref{eq:cp3-01-vp}. Utilizzando alcuni risultati ottenuti nelle
sezioni precedenti, ed in partiocalare la
\emph{Proposizione} \ref{prop:cp3-01}, possiamo riscriverci l'equazione \eqref{eq:cp3-01-vp} nel seguente modo
\begin{equation}
\label{eq:cp3-sc3-01}
u_t = \vec{v}_1^tD^2u\vec{v}_1 +\vec{v}_2^tD^2u\vec{v}_2-\mathcal{I}(u,t)x^tDu,
\end{equation}
con $\mathcal{I}(u,t)$ pari a
\begin{equation}
\label{eq:cp3-sc3-02}
\mathcal{I}(u,t) = \frac{\iint\Div{\left(\frac{Du}{|Du|}\right)}d\mu}{3V_0}.
\end{equation}
Il numeratore di \eqref{eq:cp3-sc3-02} rappresenta l'integrale della
curvatura media (la divergenza della normale) sulla superficie
individuata da una curva di livello di $u(x,t)$, mentre il
denominatore il volume iniziale della medesima curva di livello. Questo
termine, purtroppo, non è costante (tranne in alcuni casi vedere
§\ref{subsec:cp3-sc3-3-2}) per cui dobbiamo
approssimarlo.
%%%%%%%%%%%%%%%%%%%%%%%%%%%%%%%%%%%%%%%%%%%%%%%%%%%%%%%%%%%%%%%%%%%%%%%%%%%%%%
%                 SubSection 3.3.1
%
%%%%%%%%%%%%%%%%%%%%%%%%%%%%%%%%%%%%%%%%%%%%%%%%%%%%%%%%%%%%%%%%%%%%%%%%%%%%%%
\subsection{Costruzione dello schema}
\label{subsec:cp3-sc3-3-1} 
Prima di procedere con la costruzione del nostro schema, vediamo come
riscrivere il nostro integrale di  superficie
\cite[vedi][§1.5]{osher:fed}. Ricordando la definizione di superfice
in forma level-set, data in §\ref{sec:cp1-00}, abbiamo che i punti
appartenenti alla nostra superficie sono tutti quegli
$x\in\Omega\subseteq\mathbb{R}^3$ tale che $u(x)=c$ con
$c\in\mathbb{R}$ un qualsiasi livello scelto. Consideriamo superfici
chiuse in modo tale da poter definire chiaramente zone interne e zone
esterne e prendiamo come interfaccia $u(x)=0$, cioè scegliamo il
livello zero, lo possiamo fare senza perdità di generalità (vedi
\emph{Osservazione} \ref{oss:cp3-sc3-01}). In questa configurazione
possiamo facilmente  determinare se un punto $x\in\Omega$ è interno o
esterno alla superfice, guardando  al segno locale di $u(x)$; poniamo
che se $u$ in $x$ ha segno negativo allora $x$ è un punto interno, se
ha segno positivo è esterno e mentre se é uguale a zero sta sulla
suferficie. Detto ciò definiamo i seguenti insiemi (vedi Fig \ref{fig:cp3-sc3-01})
\[
\begin{aligned}
\Omega^{+}&=\left\{x\in\Omega \,|\, u(x)>0\right\} \\
\Omega_{-}&=\left\{x\in\Omega \,|\, u(x)<0\right\} \\
 \partial\Omega&=\left\{x\in\Omega \,|\, u(x)=0\right\}.
\end{aligned}
\]  
\begin{esempio}
Prendiamo ad esempio la funzione $u(x)=x_1^2+x_2^2+x_3^2-1$, che al livello
zero rappresenta la sfera in $\mathbb{R}^3$ di raggio uno, allora
abbiamo
\[
\begin{aligned}
\Omega^{+}&=\left\{x\in\mathbb{R}^3 \,|\, |x|^2>1\right\} \\
\Omega_{-}&=\left\{x\in\mathbb{R}^3 \,|\, |x|^2<1\right\} \\
 \partial\Omega&=\left\{x\in\mathbb{R}^3 \,|\, |x|^2=0\right\}.
\end{aligned}
\] 

\begin{figure}[!htbp]
  \begin{center}
  \tdplotsetmaincoords{60}{40}
  \begin{tikzpicture}[tdplot_main_coords,gray,thick]
   
    \coordinate (O) at (0.66,0.0,0.0);
       
    \tdplotsetcoord{P}{1.3}{90}{45};
    \tdplotsetcoord{P2}{-2.0}{120}{30};
    \tdplotsetcoord{P3}{3.0}{60}{40};
    \tdplotsetcoord{P4}{0.8}{97}{153};
    \tdplotsetcoord{P5}{1.3}{-55}{40};

    
    \draw [->,Mahogany!80!Mulberry](P2) to[out=120,in=180](P5);
    \node[fill=red!15,rotate=50,scale=4.0,rounded corners] at (P4) {};
    \node[shape=circle,draw=gray,fill=gray,inner sep=0pt,minimum size=1mm]
    (origin) at (O) {};
    \node [Mahogany!80!Mulberry,left] at (origin.south) {$O$};
    \node[] (end) at (P3) {};
    \node [Mahogany!80!Mulberry] at (end.south) {$\Omega^+$};
    \node[] (end1) at (P) {};
    \node [Mahogany!80!Mulberry] at (end1.south) {$\Omega^-$};
    \node[] (end2) at (P2) {};
    \node [Mahogany!80!Mulberry] at (end2.north) {$\partial\Omega$};

    \draw  [tdplot_main_coords,dashed] (O) ellipse (42pt and 20pt);
    \tdplotsetthetaplanecoords{40};
    \draw [thick,tdplot_rotated_coords](1.0,0.5,0.5) arc (0:360:1.5);
    
        
  \end{tikzpicture}
  \end{center}
  \caption{Insiemi di punti interni ed esterni alla sfera di livello
    zero della funzione $u(x)=|x|^2-1$}
  \label{fig:cp3-sc3-01}
\end{figure}
\end{esempio}

\begin{osservazione}
  \label{oss:cp3-sc3-01}
Da notare che il discorso fatto fin'ora, è valido anche se prendiamo
un livello $c\ne 0$. In quanto possiamo riformulare il tutto
scegliendo il livello zero per la funzione  $\tilde{u}(x)=u(x)-c$.
\end{osservazione}
Ora ricordiamo che l'\emph{integrale di superficie} di una funzione
$f(x)$ sulla frontiera di un insieme in $\mathbb{R}^3$ può essere definito come \cite[vedi][§1.5]{osher:fed}
\begin{equation}
  \label{eq:cp3-sc3-03}
  \int_{\Omega}f(x)\hat{\delta}(x)dx,
\end{equation}
con $\hat{\delta}(x)$  \emph{delta di Dirac}, che taglia
fuori tutto dal calcolo dell'integrale tranne la frontiera di
$\Omega$. La delta di Dirac $\hat{\delta}(x)$, può essere definita
(\cite[vedi][§1.5]{osher:fed}) come la
derivata direzionale (nel senso delle distribuzioni) nella direzione normale della funzione di \emph{Heaviside}:
\begin{equation}
  \label{eq:cp3-sc3-04}
  \hat{\delta}(x)=DH(u(x))\cdot\vec{n}
\end{equation}
ed è una funzione nella variabile multidimensionale $x$, con $H$
funzione unidimensionale della forma 
\begin{equation}
\label{eq:cp3-sc3-1-add-01}
H(u)=
\begin{cases}
  0 &\text{ se $u\leq 0$} \\
  1 &\text{ se $u>0$}
\end{cases}
\end{equation}
Questa distribuzione è diversa da zero solo sull'interfaccia $\partial\Omega$
dove $u=0$. L'equazione \eqref{eq:cp3-sc3-04} può essere riscritta come
\[
\hat{\delta}(x) = H'(u(x))Du(x)\cdot\frac{Du(x)}{|Du(x)|}=H'(u(x))|Du(x)|
\]
dove abbiamo sfruttato la definizione della normale,
$\vec{n}(x)=Du/|Du|$. In una dimensione spaziale, vale la seguente uguaglianza
\[
\delta(u)=H'(u),
\]
con $\delta(u)$ delta di Dirac unidimensionale e $H'(u)$ derivata nel
senso delle distribuzioni della funzione di Heaviside.
Quindi la \eqref{eq:cp3-sc3-04} diventa
\begin{equation}
  \label{eq:cp3-sc3-05}
  \hat{\delta}(x)=\delta(u(x))|Du(x)|,
\end{equation}
inserendo quest'ultima in \eqref{eq:cp3-sc3-03} possiamo riscriverci l'integrale di superficie come 
\begin{equation}
  \label{eq:cp3-sc3-06}
  \int_{\Omega}f(x)\delta(u(x))|Du(x)|dx.
\end{equation}
Applicando questo risultato al numeratore di \eqref{eq:cp3-sc3-02}
otteniamo
\begin{equation}
  \label{eq:cp3-sc3-07}
\mathcal{I}(u,t) = \frac{\int_{\Omega}\Div{\left(\frac{Du}{|Du|}\right)}\delta(u(x))|Du(x,t)|dx}{3V_0}.
\end{equation}
Iniziamo col notare che il termine che compare all'interno
dell'integrale insieme alla delta, non è altro che il flusso, con il
segno meno davanti, dell'equazione classica del moto per curvatura
media \eqref{eq:cp3-01}; quindi possiamo utilizzare l'approssimazione
spaziale dello schema MCM presentata in §\ref{sec:cp3-sc3-1}. Mentre
per approssimare $\delta(u)$ utilizziamo un'approssimazione
``smeared-out'' della funzione $\delta(u)$, cioè sostituiamo
quest'ultima con
\begin{equation}
  \label{eq:cp3-sc3-08}
  \delta(u)=
  \begin{cases}
    0 &\text{ se $u<-\varepsilon$} \\
    \frac{1}{2\varepsilon}+\frac{1}{2\varepsilon}\cos\left(\frac{\pi
      u}{\varepsilon}\right) &\text{ se $-\varepsilon\leq u \leq\varepsilon$} \\
    0 &\text{ se $\varepsilon<u$}
  \end{cases}
\end{equation}
con $\varepsilon=1.5\Delta x$ (come suggerito in
\cite[][§1.5]{osher:fed}). Questa non è niente altro che la derivata
della funzione ``spalmata'' di Heaviside:
\[
H(u)=
\begin{cases}
 0 &\text{ se $u<-\varepsilon$} \\
    \frac{1}{2}+\frac{u}{2\varepsilon} +\frac{1}{2\pi}\sin\left(\frac{\pi
      u}{\varepsilon}\right) &\text{ se $-\varepsilon\leq u \leq\varepsilon$} \\
 1 &\text{ se $\varepsilon<u$} 
\end{cases}
\]
infine come metodo di intergrazione usiamo la formula del punto medio
in $\mathbb{R}^3$. 
Consideriamo una griglia spaziale $x_j=(j_1\Delta x,j_2\Delta
x,j_3\Delta x)$ con $j=(j_1,j_2,$ $j_3)$ multi-indice in $\mathbb{Z}^3$ ed una griglia temporale $t^n=n\Delta t$ con $n\in\mathbb{N}$.
Indichiamo con $u_j^n$ l'approssimazione di $u$ nel punto della
griglia $(x_j,t^n)$ e con $D_j^n$ l'approssimazione del gradiente di
$u$ nel suddetto punto. L'approssimazione $\mathcal{I}^n$ al
tempo $t^n$ di  \eqref{eq:cp3-sc3-02}, ottenuta con il metodo precedentemente
descritto è pari a 
\begin{equation}
\label{eq:cp3-sc3-08-add}
\mathcal{I}^n=\frac{1}{3V_0}\Delta x^3\sum_{j=1}^{n^3}\hat{u}_j^n\delta(u_j^n),
\end{equation}
con $\delta$ data da \eqref{eq:cp3-sc3-08} e $\hat{u}_j^n$ è paria a 
\[
\hat{u}_j^n=
\begin{cases}
\frac{\sum_{+,-}I[u^n]\left(x_j+\sqrt{2\Delta
    t}(\pm\vec{v}_j^{n,1}\pm\vec{v}_j^{n,2})\right)-4u_j^n}{4\Delta t}
&\text{ se vale la \eqref{eq:cp3-sc21-01}} \\
\frac{\sum_{i\in\delta{j}}u_i^n-6u_j^n}{6\Delta t}&\text{ se vale la \eqref{eq:cp3-sc3-1-01}}
\end{cases}
\]
con $\delta(j)$ insieme degli indici dei  primi vicini a $x_j$.
\begin{osservazione}
Come osservato in \cite[][§1.5]{osher:fed}, la scelta del metodo di
integrazione non influisce sull'errore commesso nel calcolo
dell'integrale \eqref{eq:cp3-sc3-06} poichè l'uso della
\eqref{eq:cp3-sc3-08} per approssimare la delta, fa si che il  metodo
abbia un accuratezza del primo ordine.
\end{osservazione}
Infine utilizzando le differenze finite in avanti per la derivata
temporale e le derivate direzionali con incremento $\sqrt{2\Delta t}$
per il flusso, come fatto per MCM in §\ref{sec:cp3-sc3-1}, nel caso in
cui siamo sopra la soglia, cioè la \eqref{eq:cp3-sc21-01} è soddisfatta abbiamo
\begin{equation}
  \label{eq:cp3-sc3-09}
  u_j^{n+1}=\frac{1}{4}\sum_{+,-}I[u^n]\left(x_j(1-\mathcal{I}^n\Delta
  t)+ \sqrt{2\Delta t}(\pm\vec{v}_j^{n,1}\pm\vec{v}_j^{n,2})\right)
\end{equation}
con $\vec{v}_j^{n,1}$ e $\vec{v}_j^{n,2}$ come in
\eqref{eq:cp3-sc3-1-end}. Mentre nel caso in cui siamo sotto la
soglia, ovvero vale la \eqref{eq:cp3-sc3-1-01}, utilizziamo lo stesso
metodo usato nello schema MCM, cioè \eqref{eq:cp3-03-add}.
%%%%%%%%%%%%%%%%%%%%%%%%%%%%%%%%%%%%%%%%%%%%%%%%%%%%%%%%%%%%%%%%%%%%%%%%%%%%%%
%                 SubSection 3.3.2
%
%%%%%%%%%%%%%%%%%%%%%%%%%%%%%%%%%%%%%%%%%%%%%%%%%%%%%%%%%%%%%%%%%%%%%%%%%%%%%%
\subsection{Consistenza dello schema in un caso semplificato}
  \label{subsec:cp3-sc3-3-2}

Analizziamo lo schema \emph{volume preserving}, descritto in
precedenza, in un caso semplice, cioè quando il termine
\eqref{eq:cp3-sc3-02} è costante, quindi non abbiamo bisogno
della sua approssimazione. Questa situazione si verifica, ad esempio, quando gli
insiemi di levello di $u(x,0)=u_0(x)$ rappresentano una sfera. Difatti
in tale caso possiamo calcolarci esplicitamente il termine di
trasporto che compare in \eqref{eq:cp3-01-vp}. Ricordando che
l'equazione level-set della sfera è
\begin{equation}
\label{eq:cp3-sc3-10}
u_0(x)=|x|^2-r_0^2\text{, con $x\in\mathbb{R}^3$},
\end{equation}  
prendendo il livello zero abbiamo
\begin{align}
\label{eq:cp3-sc3-11}
|Du_0(x)|=2r_0\implies&\Div\left(\frac{Du_0(x)}{|Du_0(x)|}\right)|Du_0(x)|=\frac{2}{r_0}2r_0=4 \\
\label{eq:cp3-sc3-12}
&<Du_0(x),x>=2r_0^2.
\end{align}
Ora osserviamo che la sfera quando si evolve per curvatura media
mantiene la sua forma, cioè per ogni tempo $t>0$ $u(x,t)$ rappresenta
sempre una sfera (in tal caso si parla si soluzione
\emph{self-similar}) e con l'aggiunta del termine di trasporto in
\eqref{eq:cp3-01-vp} riusciamo a conservare il volume iniziale, di un livello
fissato. Per cui, poichè il volume della sfera dipende dal raggio,
 $u(x,t)=u_0(x)\,\forall\,t>0$. Quindi
\[
u_t(x,t)=0\implies\Div\left(\frac{Du)}{|Du)|}\right)|Du)|=\mathcal{I}<Du,x>\,
\forall t\geq 0,
\]
utilizzando \eqref{eq:cp3-sc3-11} e \eqref{eq:cp3-sc3-12} abbiamo
\[
4=\mathcal{I}2r_0^2\implies\mathcal{I}=\frac{2}{r_0^2},
\]
scrivendo il raggio iniziale in dipendenza dal volume iniziale $V_0$
otteniamo
\begin{equation}
\label{eq:cp3-sc3-13}
\mathcal{I}=2\left(\frac{4\pi}{3V_0}\right)^{\frac{2}{3}}.
\end{equation}
Riscriviamo il nostro schema, nel caso in cui siamo sopra la soglia,
in questo particolare scenario
\begin{equation}
\label{eq:cp3-sc3-14}
u_j^{n+1}=\frac{1}{4}\sum_{+,-}I[u^n]\left(x_j(1-\mathcal{I}\Delta
  t)+ \sqrt{2\Delta t}(\pm\vec{v}_j^{n,1}\pm\vec{v}_j^{n,2})\right)
\end{equation}
con $\mathcal{I}$ definito in \eqref{eq:cp3-sc3-13}. Andiamo a
studiare la consistenza di questo schema applicando la definizione
generale di consistenza \eqref{eq:cp3-03}. Consideriamo una
successione di di nodi $(x_{j_m},t^{n_m})$ che tende a $(x,t)$ come in
§\ref{subsec:cp3-sc2-1} e per semplicità notazionale omettiamo il pedice
$m$. In questo caso semplice, la nostra $u(x,t)$ rappresenta una sfera
quindi è una soluzione regolare di \eqref{eq:cp3-01-vp},
lipschitiziana con costante $L_u$, con $|Du|<M$ e vale anche
l'assunzione $|u_{tt}(x,t)|\leq C$ fatta in §\ref{subsec:cp3-sc2-1} in modo
da limitare l'errore di troncamento per l'approssimazione
temporale. Analizziamo il caso in cui $Du(x,t)\ne 0$ e
$u_{x_1}(x,t)^2+u_{x_3}(x,t)^2\ne 0$. Vista la regolarità di $u$,
esiste un intorno di $(x,t)$ in cui le due espressioni precedenti si
mantengono vere, quindi almeno asintoticamente vale
\eqref{eq:cp3-sc21-01}.  Iniziamo col riscriverci il termine della
sommatoria in \eqref{eq:cp3-sc3-14} come
\[
I[u^n](x_j(1-\mathcal{I}\Delta t)+\sigma(D_j^n)\sqrt{2\Delta t}\vec{b}),
\]
utilizzando le stesse notazioni di §\ref{subsec:cp3-sc2-1}. Operiamo,
su tale termine, alcune manipolazioni algebriche: aggiungiamo e togliamo
il il valore di $u$ nel punto in cui viene interpolato e aggiungiamo e
togliamo il il valore di $u$ calcolanta in
$x_j(1-\mathcal{I})+\sqrt{2\Delta t}\sigma(Du(x_j,t^n))\vec{b}$, dove
abbiamo sistituito il grandiente approssimato con quello effettivo
calcolato nel punto $(x_j,t^n)$. Ora come in §\ref{subsec:cp3-sc2-1},
sfruttando la stima dell'errore di intepolazione \eqref{eq:cp3-04}, la
stima dell'errore di approssimazione del gradiente \eqref{eq:cp3-05} e
la lipshtizianità di $\sigma(Du)$ e di $u$ otteniamo
\[
\begin{split}
I[u^n]&(x_j(1-\mathcal{I}\Delta t)+\sigma(D_j^n)\sqrt{2\Delta
    t}\vec{b})=\\
=&u(x_j(1-\mathcal{I}\Delta t) +\sigma(Du(x_j,t^n))\sqrt{2\Delta t}\vec{b},t^n) +\\
&+O(\Delta x^r)+O(\Delta x^{q-s}\sqrt{2\Delta t}).
\end{split}
\]
Sviluppiamo il termine $u(x_j(1-\mathcal{I}\Delta t)
+\sigma(Du(x_j,t^n))\sqrt{2\Delta t}\vec{b},t^n)$ con un serie di
Taylor del terzo odine in un intorno di $x_j$
\[
\begin{split}
&u(x_j(1-\mathcal{I}\Delta t)+\sigma(Du(x_j,t^n))\sqrt{2\Delta t}\vec{b},t^n) = u(x_j,t^n)+ \\
&+\left<Du(x_j,t^n),\sigma(Du(x_j,t^n))\vec{b}\right>\sqrt{2\Delta t} +\\
&+\left(\left<D^2u(x_j,t^n)\sigma(Du(x_j,t^n))\vec{b},\sigma(Du(x_j,t^n))\vec{b}\right>-C<Du(x_j,t^n),x_j>\right)\Delta t + \\
&+T_3(\Delta t^{\frac{3}{2}}\sigma(Du(x_j,t^n))\vec{b},D^3u(x_j,t^n))+O(\Delta t^2), 
\end{split}
\]
I termini del primo e del terzo ordine
si eliminato quando sommiamo al variare di $\vec{b}\in\Pi$. In
conclusione abbiamo
\[
\begin{split}
& S(x_j,u(x_j,t^n),Du(x_j,t^n)) = \\
& = u(x_j,t^n) + \vec{v}_1^tD^2u(x_j,t^n)\vec{v}_1 + \vec{v}_2^tD^2u(x_j,t^n)\vec{v}_2 -\mathcal{I}<Du(x_j,t^n),x_j>+ \\
& + O(\Delta x^r) + O(\Delta x^{q-s}\Delta t^{\frac{1}{2}}) + O(\Delta t^2),
\end{split}
\]
e sviluppando anche $u(x_j,t^{n+1})$ ed applicando la definizione
\eqref{eq:cp3-03}, abbiamo che essa risulta soddisfatta nella forma
classica se $\Delta x = o(\Delta t^{\frac{1}{r}})$ e $\Delta x^{q-s}=o(\Delta t^{\frac{1}{2}})$.
