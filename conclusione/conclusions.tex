\chapter*{Conclusioni}
\phantomsection
\addcontentsline{toc}{chapter}{Conclusioni}

Il modello MCM \emph{volume preserving}, che abbiamo presentato,
risulta essere uno schema molto robusto anche per passi temporali
``lunghi'' (non richiede la CFL parabolica), si comporta molto bene
per il \emph{denoising} di superfici geometriche, consentendo di
compiere più iterazioni di filtraggio per eliminare il rumore senza
correre il rischio del collasso in un punto. Tuttavia eredità anche
alcune caratteristiche del classico MCM, come lo smussamento dei bordi
dell'immagine e l'emergere di singolarità nel caso di superfici non
convesse; limite quest'ultimo intrenseco nel flusso per curvatura media,
unica soluzione sarebbe quella di sostituire il flusso con altri più
complicati come suggerito in \cite{gui:sapiro}. Per quanto riguarda il
caso di immagini reali, non abbiamo ottenuto risultati molto soddisfacenti
anche se abbiamo testato il metodo su una sola immagine quindi
si dovrebbe ampliare la casistica per vedere i vantaggi rispetto al
cassico MCM. Per completare il lavoro si dovrebbe aggiungere  un conto
sulla consistenza dello schema generale e sulla convergenza. Mentre
per futuri miglioramenti si potrebbe provare ad utilizzare tecniche
piu raffinate nell'approssimazione del termine di conservazione del
volume. 
