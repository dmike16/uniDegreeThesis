\section{Modello}
\begin{frame}{Moto per curvatura media}
     \begin{block}{}
       \begin{itemize}
       \item $S$ superfice regolare in $\mathbb{R}^3$
       \item $\mathcal{S}(p,t)$ parametrizzazione di $S$
       \item $\mathcal{N}(x,t)$ vettore unitario normale regolare
       \item $\mathcal{H}=-\Div(\mathcal{N})\mathcal{N}$ vettore curvatura media
       \end{itemize}
     \end{block}
     \begin{block}{}
       $(x_1,x_2,x_3)\in S_t$ evolve seconde il sitema parabolico
       \[
       \left\{
       \begin{aligned}
         &\mathcal{S}_t(p,t)=-\Div(\mathcal{N})\mathcal{N}\text{ $t>0$} \\
         &\mathcal{S}(0)=S
       \end{aligned}
       \right.
       \]
     \end{block}
\end{frame}
\subsection*{Limiti}
\begin{frame}{Collasso in un punto}
  \begin{columns}[c]
    \begin{column}{6cm}
      \begin{block}{Superfici convesse}
       Superfici convesse  evolvono in un
       punto in un tempo finito.
       \end{block}
      \begin{exampleblock}{Evoluzione della Sfera}
        Una famiglia di sfere $\mathcal{S}(p,t)=S^2(R(t))$ con
        $R(0)=R_0$, si evolve secondo
        \[
        \begin{aligned}
          \overset{\displaystyle.}{R}(t) &= -\frac{2}{R(t)},\,
          R(0)=R_0,\Rightarrow\\
          R(t)&=\sqrt{R_0^2-4t}\Rightarrow, \\
          R(0)&=0 \Longleftrightarrow t=\frac{R_0^2}{4}<\infty 
        \end{aligned}
        \]
      \end{exampleblock}
    \end{column}
    \begin{column}[c]{4cm}
       \begin{center}
     \only<2->{\animategraphics[autoplay,loop,width=1.0\textwidth,height=0.45\textheight]{0.8}{smooth/mcm/sphere/sphere50-}{0}{4}}
     \end{center}
    \end{column}
    \end{columns}
\end{frame}
\begin{frame}{Possibili cambiamenti topologici e singolarità}
  \begin{columns}[c]
    \begin{column}{5cm}
      \begin{block}{Superfici non convesse}
       Superfici non convesse possono subire cambiamenti topologici
       generando delle singolartà. 
       \end{block}
      \begin{exampleblock}{Evoluzione del Manubrio}
        Il manubrio può essere considerato come due sfere di equal
        raggio connesse da un cilindro. A causa del collasso del
        cilindro in una linea, il manubrio si spezza in due sfere
        disconnesse.
      \end{exampleblock}
    \end{column}
    \begin{column}[c]{5cm}
       \begin{center}
     \only<2->{\animategraphics[autoplay,loop,width=1.0\textwidth,height=0.45\textheight]{0.8}{smooth/mcm/dumbbell/dumbb100-}{0}{5}}
     \end{center}
    \end{column}
    \end{columns}
\end{frame}

\subsection*{Versione PVMCM}
\begin{frame}{Flusso \emph{volume preserving}}
  \begin{columns}[T]
    \begin{column}{6cm}
      \begin{block}{Processo di nomalizzazione}
        Cambiamo la scala temporale
        \begin{itemize}
        \item $\mathcal{\tilde{S}(\tau)}=\psi(t)\mathcal{S}(t)$ 
        \item $\psi^2(t)=\frac{\partial\tau}{\partial t}$
        \end{itemize}
         in modo tale che \alert{$Volume_{\tau}\equiv 0$}, ottenendo   
         \begin{itemize}
         \item $\mathcal{\tilde{S}}_t=\left(\mathcal{\tilde{H}}-\frac{\rho\iint\mathcal{\tilde{H}}d\mu}{3V_0}\right)\mathcal{N}$
         \item $\rho =-<\mathcal{\tilde{S}},\mathcal{N}>$
         \end{itemize}
      \end{block}
    \end{column}
    \begin{column}[T]{4cm}
      %       \begin{center}
      %     \only<2->{\animategraphics[autoplay,loop,width=1.0\textwidth,height=0.45\textheight]{0.8}{smooth/vpmcm/sphere/sphere}{0}{4}}
      %     \end{center}
      \begin{exampleblock}{Evoluzione della sfera}
        \begin{itemize}
        \item $\tau(t)=-T\ln(T-t)$ con $T=\frac{R_0^2}{4}$ tempo di
          collasso
        \item $\tilde{V}=\psi^3(t)V$
        \item $\tilde{V}=\left(\frac{T}{T-t}\right)^{\frac{3}{2}}V$
        \item $V=\frac{4}{3}\pi(R_0^2-4t)^{\frac{3}{2}}$
        \item $\tilde{V}=\frac{4}{3}\pi R_0^3=V_0$
        \end{itemize}
      \end{exampleblock}
    \end{column}
  \end{columns}
\end{frame}
