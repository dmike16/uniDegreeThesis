\section{Formulazione level-set}
\begin{frame}{MCM \emph{volume preserving} in forma level-set}
  \begin{block}{Superfici come insiemi di livello}
    La superficie regolare $S$ è rappresentata come il livello $0$ di
    una funzione ausiliaria $u(x)\in C^{\infty}(\Omega)$
    \begin{itemize}
      \item $S=\{x\in\Omega | u(x)=0\}$ e $Du(x)\neq 0$ in $S$
      \item $\mathcal{N}=-\frac{Du(x)}{|Du(x)|}$ vettore normale
        regolare
    \end{itemize}
  \end{block}
  \begin{block}{Equazione in forma level-set}
    La famiglia di superfici regolari $S(t)$
    \begin{itemize}
      \item $S(t)=\{x\in\Omega | u(x,t)=0\}$ con $u(x,t)$ definita in
        $\Omega\times I$ e tale che $Du(x,t)\neq 0$ in $S(t)$
      \item $x_0\in S(t_0)$, $x(t)$ curva regolare in
        $I_{\delta}(t_0)$ tale che $x(t)\in S(t),x(t_0)=x_0$
      \item $u(x(t),t)=0$ derivando $u_t(x_0,t_0)=<\frac{dx}{dt}(t_0),\mathcal{N}>|Du(x_0,t_0)|$
    \end{itemize}
  \end{block}
\end{frame}

\begin{frame}{Flusso VPMCM}
    \[
      (\text{VPMCM})\,u_t=\Div\left(\frac{Du(x,t)}{|Du(x,t)|}\right)|Du(x,t)|-\frac{\iint\Div\left(\frac{Du(x,t)}{|Du(x,t)|}\right)d\mu}{3V_0}x^tDu(x,t). 
      \]
  %\begin{block}{Parte MCM e termine di traporto} 
    \begin{itemize}
    \item parte MCM 
      \[
      |Du(x,t)|\Div\left(\frac{Du(x,t)}{|Du(x,t)|}\right)=tr(P(Du)D^2u)
      \]
      \item $P(b)=I-\frac{b\otimes b}{|b|^2}\,b\in\mathbb{R}^3$
        matrice di proiezione sul piano ortogonale a $b$, semidefinita
        positiva con rango $2$ singolare in $Du=0$
      \item parte di trasporto, non locale. 
        \[
        \mathcal{I}(\mathcal{H}(u),t)=\frac{\iint\Div\left(\frac{Du(x,t)}{|Du(x,t)|}\right)d\mu}{3V_0}
        \]
     \end{itemize}
  %\end{block}
\end{frame}

\begin{frame}{Vantaggi forma level-set}
 \begin{block}{Vantaggi}
   \begin{itemize}
     \item Forma level-set scritta in un sistema di coordinate fisso
     \item I cambiamenti topologici non sono un problema. Topologia di
       $u$ è fissa.
     \item \hyperlink{jumptoend}{Soluzioni che presentano singolarità nel
       gradiente sono ben 
       poste nella classe delle soluzioni di viscosità}
       \[
       \begin{aligned}
         u_t+&F(x,u,Du,D^2u)=0 \\
         &F:\Omega\times\mathbb{R}\times\mathbb{R}^n\times
           S(n)\to\mathbb{R}\\
       \end{aligned}
           \]
   \end{itemize}
 \end{block}
\end{frame}

%\begin{frame}{Soluzioni di viscosità}
%  \begin{definizione}
%    Sia $\Omega$ aperto di $\mathbb{R}^n$ e $T>0$. Sia $F$ continua in
%    $W=\overline{\Omega}\times
%    [0,T]\times\mathbb{R}\times\mathbb{R}^n\times S(n)$ a valori in
%    $\mathbb{R}$. Sia $\mathcal{O}$ aperto in $\Omega\times(0,T)$.
%    \begin{itemize}
%      \item Una funzione $u:\mathcal{O}\to\mathbb{R}$ semicontinua
%        superiormente è una \alert{sotto
%          soluzione viscosa} di 
%        \[
%        u_t+F(x,t,u,Du,D^2u)=0
%        \]
%        in $\mathcal{O}$ se per ogni $\phi\in C^2(\mathcal{O})$ e
%        $\hat{z}=(\hat{x},\hat{t})\in\mathcal{O}$ tale che $(u-\phi)$ ha
%        massimo in $\hat{z}$ allora
%        \[
%        \phi_t(\hat{z})+F(\hat{z},\phi(\hat{z}),D\phi(\hat{z}),D^2\phi(\hat{z}))\leq 0
%        \]
%        \item $u$ semicontinua inferiomente e una \alert{sopra
%          soluzione viscosa}  in $\mathcal{O}$ se per ogni coppia
%          $\phi$ e $\hat{z}$ tale che $(u-\phi)$ ha min in $\hat{z}$
%          vale che
%          \[
%          \phi_t(\hat{z})+F(\hat{z},\phi(\hat{z}),D\phi(\hat{z}),D^2\phi(\hat{z}))\geq 0
 %         \]
 %       \end{itemize}
 % \end{definizione}
%\end{frame}

%\begin{frame}{Proprietà di confronto}
%  \begin{osservazione}
%    Nel caso $F$ non sia continua, la definizione di solulzione
%    viscosa continua a valere per l'inviluppo semicontinuo di
%    $F$. Ricordiamo che data $h$ definita in $L\subset\mathbb{R}^n$
%    \begin{itemize}
%    \item  invluppo semicontinuo inferiore $h_*(x)=\lim_{r\to
%      0}\inf\{h(y); y\in B_r(x)\cap L\}$ con $x\in\overline{L}$
%    \item inviluppo semicontinuo superiore $h^*(x)=\lim_{r\to
%      0}\sup\{h(y);y\in B_r(x)\cap L\}$ con $x\in\overline{L}$
%    \end{itemize}
%  \end{osservazione}
%  \begin{block}{Propietà di confronto}
%    Se $u$ è una sotto soluzione in $Q=\Omega\times[0,T)$ e
%      $v$ una sopra  soluzione sempre in $Q$,  allora $u\leq
%      v$ in $\mathcal{O}$ se $u\leq v$ sulla frontiera prabolica 
%    \[
%    \partial_pQ=\Omega\times\{0\}\cup\partial\Omega\times[0,T).
%    \]
%  \end{block}
%\end{frame}

\begin{frame}{Esistenza e confronto in MCM e VPMCM}
%\hypertarget{jumptoend}{}
  \begin{block}{Problema risolto}
    Nel caso MCM il flusso $F(p,X)=-tr(P(p)X)$ verifica
    \begin{enumerate}
    \item \alert{propria}: $F(x,r,p,X)\leq F(x,s,p,X)$ per $r\leq
      s$ e $\forall\,p\in\mathbb{R}^n$,$X\in S(n)$
    \item \alert{ellittica degenere}: $F(x,r,p,X)\leq
      F(x,r,p,Y)$ per $X\geq Y$
    \item \alert{geometrica}: $F(x,\lambda p,\lambda X+\sigma
      p\otimes p)=\lambda F(x,p,X)$, $\lambda >0,\sigma\in\mathbb{R}$
    %  \item $-\infty<F_*(x,0,O)=F^*(x,0,O)<\infty$
    \end{enumerate}
    sotto queste proprietà valgono risultati di esistenza e confronto.
  \end{block}
  \begin{block}{Non abbiamo risultati definitivi}
    Nel caso VPMCM $F(x,u,p,X)=-tr(P(p)X)+b(u)x^tp$. In generale
    $b(u)$ non è costante quindi per esempio non è
    \alert{propria} è neanche \alert{geometrica}, sopravvive solo la
    proprietà di \alert{degenere ellitticità}; per cui  non possiamo
    utilizzare i risultati precedenti. Non abbiamo trovato risultati
    definitivi su questi tipi di equazioni.
  \end{block}
Referenza: Y. Chen, Y. Giga e S.Goto (1991).
\end{frame}
