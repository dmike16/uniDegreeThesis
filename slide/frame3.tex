\section{Formulazione level-set}
\begin{frame}{MCM \emph{volume preserving} in forma level-set}
  \begin{block}{Superfici come insiemi di livello}
    La superfice regolare $S$ è rappresentata come il livello $0$ di
    una funzione ausiliaria $u(x)\in C^{\infty}(\Omega)$
    \begin{itemize}
      \item $S=\{x\in\Omega | u(x)=0\}$ e $Du(x)\neq 0$ in $S$
      \item $\mathcal{N}=-\frac{Du(x)}{|Du(x)|}$ vettore normale
        regolare
    \end{itemize}
  \end{block}
  \begin{block}{Equazione in forma level-set}
    La famiglia di curve che si evolve $S_t=\{x\in\Omega | u(x,t)=0\}$
    è ottenuta dalla soluzione di
    \[
      (\text{VPMCM})\,u_t=|Du(x,t)|\Div\left(\frac{Du(x,t)}{|Du(x,t)|}\right)-\frac{\iint\Div\left(\frac{Du(x,t)}{|Du(x,t)|}\right)d\mu}{3V_0}x^tDu(x,t). 
      \]
      ricordando che la \alert{velocità normale$=\frac{u_t}{|Du|}$}.
  \end{block}
\end{frame}

\begin{frame}{Flusso VPMCM}
  Il flusso VPMCM è costituito da una parte MCM più un termine di trasporto.
  \begin{block}{Parte MCM} 
    \begin{itemize}
    \item
      $|Du(x,t)|\Div\left(\frac{Du(x,t)}{|Du(x,t)|}\right)=tr\left(P(Du)D^2u\right)$
    \item $P(b)=I-\frac{b}{|b|}\otimes\frac{b}{|b|}$ con
      $b\in\mathbb{R}^3$, con $P(b)^2=P(b)$
    \item $P(b)$ proietta ogni $x\in\mathbb{R}^3$ nel piano $\perp b$
    \end{itemize}
  \end{block}
  \begin{block}{Termine di traporto}
     Coefficente della parte di trasporto
    \[
    \mathcal{I}(\mathcal{H}(u),t)=\frac{\iint\Div\left(\frac{Du(x,t)}{|Du(x,t)|}\right)d\mu}{3V_0}
    \] 
    il quale è non locale. Abbiamo bisogno di un metodo per
    approssimarlo.
  \end{block}
\end{frame}


\begin{frame}{Flusso VPMCM nel caso della sfera}
  \begin{columns}[T]
    \begin{column}{6.5cm}
      \begin{exampleblock}{Caso semplificato, evoluzione della sfera}
        $S(R(t))$ famiglia di sfera in forma level-set
        \begin{itemize}
        \item $S_0=\{x\in\mathbb{R}^3| |x|^2-R_0^2=0\}$
        \item $V(R(t))=V_0\Rightarrow R(t)=R_0$
        \item $\mathcal{I}(\mathcal{H}(u),t)=2\frac{4\pi}{3V_0}R_0=2\left(\frac{4\pi}{3V_0}\right)^\frac{2}{3}$.
        \end{itemize}
      \end{exampleblock}
    \end{column}
    \begin{column}[T]{3.5cm}
      \begin{center}
        \only<2->{\animategraphics[autoplay,loop,width=1.0\textwidth,height=0.45\textheight]{0.8}{smooth/vpmcm/sphere/sphere}{0}{4}}
      \end{center}
  \end{column}
\end{columns}
\end{frame}

\begin{frame}{Problema ben posto nella teoria viscosa}
     \begin{block}{Classe dell'equazione}
       La nostra equazione VPMCM \alert{$u_t-F(x,u,Du,D^2u)=0$}
       rappresenta un PDE parabolico non lineare singolare nei punti in cui
       il grandiente di $u$ si annulla ed è ben posto nella teoria
       delle solulzioni viscose. $F$ del tipo
       \begin{itemize}
         \item $F:\Omega\times\mathbb{R}\times\mathbb{R}^n\times
           S(n)\to\mathbb{R}$, $S(n)$ matrici simmetriche e
           $\Omega\subset\mathbb{R}^n$
         \item $u_t=F(x,r,p,X)$
       \end{itemize}
     \end{block}
\end{frame}

\begin{frame}{Soluzioni di viscosità}
  \begin{definizione}
    Sia $\Omega$ aperto di $\mathbb{R}^n$ e $T>0$. Sia $F$ continua in
    $W=\overline{\Omega}\times
    [0,T]\times\mathbb{R}\times\mathbb{R}^n\times S(n)$ a valori in
    $\mathbb{R}$. Sia $\mathcal{O}$ aperto in $\Omega\times(0,T)$.
    \begin{itemize}
      \item Una funzione $u:\mathcal{O}\to\mathbb{R}$ semicontinua
        superiormente è una \alert{sotto
          soluzione viscosa} di 
        \[
        u_t-F(x,t,u,Du,D^2u)=0
        \]
        in $\mathcal{O}$ se per ogni $\phi\in C^2(\mathcal{O})$ e
        $\hat{z}=(\hat{x},\hat{t})\in\mathcal{O}$ tale che $(u-\phi)$ ha
        massimo in $\hat{z}$ allora
        \[
        \phi_t(\hat{z})+F(\hat{z},u(\hat{z}),Du(\hat{z}),D^2u(\hat{z}))\leq 0
        \]
        \item $u$ semicontinua inferiomente e una \alert{sopra
          soluzione viscosa}  in $\mathcal{O}$ se per ogni coppia
          $phi$ e $\hat{z}$ tale che $(u-\phi)$ ha min in $\hat{z}$
          vale che
          \[
          \phi_t(\hat{z})+F(\hat{z},u(\hat{z}),Du(\hat{z}),D^2u(\hat{z}))\geq 0
          \]
        \end{itemize}
  \end{definizione}
\end{frame}

\begin{frame}{Proprietà di confronto}
  \begin{osservazione}
    Nel caso $F$ non sia continua, la definizione di solulzione
    viscosa continua a valere per l'inviluppo semicontinuo di
    $F$. Ricordiamo che data $h$ definita in $L\subset\mathbb{R}^n$
    \begin{itemize}
    \item  invluppo semicontinuo inferiore $h_*(x)=\lim_{r\to
      0}\inf\{h(y); y\in B_r(x)\cap L\}$ con $x\in\overline{L}$
    \item inviluppo semicontinuo superiore $h^*(x)=\lim_{r\to
      0}\sup\{h(y);y\in B_r(x)\cap L\}$ con $x\in\overline{L}$
    \end{itemize}
  \end{osservazione}
  \begin{block}{Propietà di confronto}
    Se $u$ è una sotto soluzione in $Q=\Omega\times[0,T)$ e
      $v$ una sopra  soluzione sempre in $Q$,  allora $u\leq
      v$ in $\mathcal{O}$ se $u\leq v$ sulla frontiera prabolica 
    \[
    \partial_pQ=\Omega\times\{0\}\cup\partial\Omega\times[0,T).
    \]
  \end{block}
\end{frame}

\begin{frame}{Esistenza e confronto in MCM e VPMCM}
  \begin{block}{Problema risolto}
    Nel caso MCM il flusso $F(p,X)=-tr(P(p)X)$ verifica
    \begin{enumerate}
    \item \alert{propria}: $F(x,r,p,X)\leq F(x,s,p,X)$ per $r\leq
      s$ e $\forall\,p\in\mathbb{R}^n$,$X\in S(n)$
    \item \alert{ellittica degenere}: $F(x,r,p,X)\leq
      F(x,r,p,Y)$ per $X\geq Y$
    \item \alert{geometrica}: $F(x,\lambda p,\lambda X+\sigma
      p\otimes p)=\lambda F(x,p,X)$, $\lambda >0,\sigma\in\mathbb{R}$
      \item $-\infty<F_*(x,0,O)=F^*(x,0,O)<\infty$
    \end{enumerate}
    sotto queste propietà valgono risultati di esistenza e confronto.
  \end{block}
  \begin{block}{Non abbiamo risultati definitivi}
    Nel caso VPMCM $F(x,u,p,X)=-tr(P(p)X)+b(u)x^tp$. In generale
    $b(u)$ non è costante quindi per esempio non è
    \alert{propria} è neanche \alert{giometrica}, sopravvive solo la
    proprietà di \alert{degenere ellitticità}; per cui  non possiamo
    utilizzare i risultati precedenti. Risultati definitivi su
    questi tipi di equaizioni ancora non ci sono.
  \end{block}
\end{frame}
