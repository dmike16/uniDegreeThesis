% Talk of Volume Preserving Mean Curvature Motion
\documentclass[10pt]{beamer}
\usepackage{settings-my-talk}
\includeonlyframes{ slide/frame1,%
                    slide/frame2,%
                    slide/frame3,%
                    slide/frame4,%
                   }

%
%
% General Informations
\title{Uno schema Semi-Lagrangiano per il moto per curvatura media che
  preserva il volume}
\author{Michele Cipolla \\
\texttt{cipmiky@gmail.com}}
\institute[Dip. Matematica]{Univeristà la Sapienza di Roma}
\date{23 Luglio 2014}
%\logo{\includegraphics[width=0.05\textwidth]{logoSapienzaHalf}}
%
%
% Customize Theme
%\definecolor{sapC}{rgb}{0.53,0.15,0.15}
%\usetheme{Antibes}
%\usetheme{Berlin}
%\usetheme{Berkeley}
\usetheme{CambridgeUS}
%\setbeamercolor*{structure}{fg=sapC}
%\setbeamerfont{size=\Large, series=\bfseries,
%  family=\rmfamily}
\setbeamertemplate{headline}
{%
  \begin{beamercolorbox}{section in head/foot}
    \vskip2pt \insertnavigation{\paperwidth}\vskip2pt
   \end{beamercolorbox}
}
\setbeamertemplate{footline}
{%
  \begin{beamercolorbox}{section in head/foot}
    \vskip2pt \insertpagenumber\vskip2pt
   \end{beamercolorbox}
}
%
%
%
\begin{document}
%
%
% Title page
\begin{frame}
\titlepage
\end{frame}
%
%
% Table of contents
\section*{Outline}
\begin{frame}{Outline}
\tableofcontents
\end{frame}
%
%
% Central Frame
\section{Introduzione}
\begin{frame}
  \frametitle{Che cosa sono i pazzi}
  \begin{definizione}
    Un \alert{pazzo} è uno matto pe davero.
  \end{definizione}
\end{frame}

\section{Modello}
\begin{frame}{Moto per curvatura media}
       \begin{itemize}
       \item $\mathcal{S}(t)$ famiglia di superfici regolari chiuse
       \item $\mathcal{N}(x,t)$ vettore unitario normale regolare con
         $x=(x_1,x_2,x_3)\in S(t)$
       \item $\mathcal{H}=-\Div(\mathcal{N})\mathcal{N}$ vettore curvatura media
       \end{itemize}
       evolve secondo il sistema parabolico
       \[
       \left\{
       \begin{aligned}
         &\frac{\partial\mathcal{S}}{\partial t}=-\Div(\mathcal{N})\mathcal{N}\text{ $t>0$} \\
         &\mathcal{S}(0)=S
       \end{aligned}
       \right.
       \]
Referenza: G. Sapiro (2003).
\end{frame}

\begin{frame}{Collasso in un punto}
  \begin{columns}[c]
    \begin{column}{6cm}
      \begin{block}{Superfici convesse}
       Superfici convesse  evolvono in un
       punto in un tempo finito.
       \end{block}
      \begin{exampleblock}{Evoluzione della Sfera}
        Una famiglia di sfere $\mathcal{S}(p,t)=S^2(R(t))$ con
        $R(0)=R_0$, si evolve secondo
        \[
        \begin{aligned}
          \overset{\displaystyle.}{R}(t) &= -\frac{2}{R(t)},\,
          R(0)=R_0,\Rightarrow\\
          R(t)&=\sqrt{R_0^2-4t}\Rightarrow, \\
          R(t)&=0 \Longleftrightarrow t=\frac{R_0^2}{4}<\infty 
        \end{aligned}
        \]
      \end{exampleblock}
    \end{column}
    \begin{column}[c]{4cm}
       \begin{center}
     \only<2->{\animategraphics[autoplay,loop,width=1.0\textwidth,height=0.45\textheight]{0.8}{smooth/mcm/sphere/sphere50-}{0}{4}}
     \end{center}
    \end{column}
    \end{columns}
\end{frame}

\begin{frame}{Possibili cambiamenti topologici e singolarità}
  \begin{columns}[c]
    \begin{column}{5cm}
      \begin{block}{Superfici non convesse}
       Superfici non convesse possono subire cambiamenti topologici
       generando delle singolartà. 
       \end{block}
      \begin{exampleblock}{Evoluzione del Manubrio}
        Il manubrio può essere considerato come due sfere di equal
        raggio connesse da un cilindro. A causa del collasso del
        cilindro in una linea, il manubrio si spezza in due sfere
        disconnesse.
      \end{exampleblock}
    \end{column}
    \begin{column}[c]{5cm}
       \begin{center}
     \only<2->{\animategraphics[autoplay,loop,width=1.0\textwidth,height=0.45\textheight]{0.8}{smooth/mcm/dumbbell/dumbb100-}{0}{5}}
     \end{center}
    \end{column}
    \end{columns}
\end{frame}


\begin{frame}{Flusso \emph{volume preserving}}
  \begin{columns}[T]
    \begin{column}{6cm}
      \begin{block}{Processo di nomalizzazione}
        Cambiamo la scala temporale
        \begin{itemize}
        \item Da $t$ a $\tau$
        \item $\mathcal{\tilde{S}(\tau)}=\psi(t)\mathcal{S}(t)$ 
        \item $\psi^2(t)=\frac{\partial\tau}{\partial t}$
        \end{itemize}
         in modo tale che \alert{$Volume_{\tau}\equiv 0$}, ottenendo   
         \begin{itemize}
         \item $\mathcal{\tilde{S}}_t=\left(\mathcal{\tilde{H}}-\frac{\rho\iint\mathcal{\tilde{H}}d\mu}{3V_0}\right)\mathcal{N}$
         \item $\rho =-<\mathcal{\tilde{S}},\mathcal{N}>$
         \end{itemize}
      \end{block}
    \end{column}
    \begin{column}[T]{4cm}
      \begin{exampleblock}{Evoluzione della sfera}
        \begin{itemize}
        \item $\tau(t)=\int_0^t\frac{T}{T-\tilde{t}}d\tilde{t}$ con
          $T=\frac{R_0^2}{4}$ tempo di 
          collasso
        \item $\tilde{V}=\psi^3(t)V$
        \item $\tilde{V}=\left(\frac{T}{T-t}\right)^{\frac{3}{2}}V$
        \item $V=\frac{4}{3}\pi(R_0^2-4t)^{\frac{3}{2}}$
        \item $\tilde{V}=\frac{4}{3}\pi R_0^3=V_0$
        \end{itemize}
      \end{exampleblock}
    \end{column}
  \end{columns}
Referenza: G. Sapiro (2003).
\end{frame}

\section{Problemi aperti}
\begin{frame}
  \frametitle{Perche non li chidono dentro na cosa}
  \begin{block}{Problemi risolti}
    Perche ono li bruciano?
  \end{block}
  \begin{block}{hih}
    Daje...o no?
  \end{block}
\end{frame}

\section{Schema Numerico Semi-Lagrangiano}
\begin{frame}{Strumenti iniziali}
  \begin{itemize}
    \item $\Omega$ dominio limitato in $\mathbb{R}^3$
    \item $\partial\Omega=\{x\in\Omega; u(x)=0\}$ superfice chiusa in form
      level-set, $\partial\Omega$ rappresenta l'\alert{interfaccia} al
      livello zero. 
    \item $\Omega^{-}=\{x\in\Omega; u(x)<0\}$
    \item $\Omega_{+}=\{x\in\Omega; u(x)>0\}$
  \end{itemize}

  \begin{center}
  \tdplotsetmaincoords{60}{40}
  \begin{tikzpicture}[tdplot_main_coords,gray,thick]
    \definecolor{sapC}{rgb}{0.54,0.14,0.14}
    \coordinate (O) at (0.66,0.0,0.0);
       
    \tdplotsetcoord{P}{1.3}{90}{45};
    \tdplotsetcoord{P2}{-2.0}{120}{30};
    \tdplotsetcoord{P3}{3.0}{60}{40};
    \tdplotsetcoord{P4}{0.8}{97}{153};
    \tdplotsetcoord{P5}{1.3}{-55}{40};

    
    \draw [->,sapC](P2) to[out=120,in=180](P5);
    \node[fill=red!15,rotate=50,scale=4.0,rounded corners] at (P4) {};
    \node[shape=circle,draw=gray,fill=gray,inner sep=0pt,minimum size=1mm]
    (origin) at (O) {};
    \node [left,sapC] at (origin.south) {$O$};
    \node[] (end) at (P3) {};
    \node [sapC] at (end.south) {$\Omega^+$};
    \node[] (end1) at (P) {};
    \node [sapC] at (end1.south) {$\Omega^-$};
    \node[] (end2) at (P2) {};
    \node [sapC] at (end2.north) {$\partial\Omega$};

    \draw  [tdplot_main_coords,dashed] (O) ellipse (42pt and 20pt);
    \tdplotsetthetaplanecoords{40};
    \draw [thick,tdplot_rotated_coords](1.0,0.5,0.5) arc (0:360:1.5);
    
        
  \end{tikzpicture}
 \end{center}
\end{frame}

\begin{frame}{Riscriviamo il termine MCM}
  Ricordiamo il flusso MCM essere $tr(P(Du)D^2u)$ con
  $P(Du)=I-\frac{Du\otimes Du}{|Du|^2}$
  \begin{enumerate}
    \item $tr(P(Du)D^2u)=v_1^tD^2uv_1+v_2^tD^2uv_2$
    \item $P(Du)$ matrice di proiezione sul piano tangente alla
      superfice, di rango $2$
    \item  $v_1$ e $v_1$ autovettori ortonormali di $P$ 
      \[
      v_1=
      \begin{bmatrix}
        \frac{-u_{x_3}}{\sqrt{u_{x_1}^2+u_{x_3}^2}} \\
        0 \\
        \frac{u_{x_1}}{\sqrt{u_{x_1}^2+u_{x_3}^2}}
      \end{bmatrix}
      \quad
      v_2=\frac{1}{|Du|}
      \begin{bmatrix}
        \frac{-u_{x_1}u_{x_2}}{\sqrt{u_{x_1}^2+u_{x_3}^2}} \\
        \sqrt{u_{x_1}^2+u_{x_3}^2} \\
        \frac{-u_{x_2}u_{x_3}}{\sqrt{u_{x_1}^2+u_{x_3}^2}}
      \end{bmatrix}
      \]
    \item per $v_1$ e $v_2$ si verifica che 
    \[
    P(Du)=\sum_{i=1}^2v_i\otimes v_i=\sigma\sigma^t\quad\text{con }
    \sigma=[v_1,v_2] 
    \]
  \end{enumerate}
\end{frame}

\begin{frame}{Riscriviamo l'integrale nel termine del trasporto}
 L'integrale diventa
 $\iint_{\partial\Omega}\Div\left(\frac{Du}{|Du|}\right)d\mu=\int_{\Omega}\Div\left(\frac{Du}{|Du|}\right)|Du|\delta(u)dx$.
 \begin{itemize}
   \item L'integrale di superfice di una funzione $f(x)$ si può
     definire anche come 
     \[
     \int_{\Omega}f(x)\hat{\delta}(x)dx
     \]   
   \item $\hat{\delta}(x)=DH(u(x))\cdot\mathcal{V}$ delta
     multidimensionale con $\mathcal{V}$ normale esterna ed 
     \[
     H(u(x))=
     \begin{cases}
       0 &\text{ se }u\leq 0 \\
       1 &\text{ se }u > 0
     \end{cases}
     \]
     \item $\hat{\delta}(x)=H'(u(x))Du\cdot\frac{Du}{|Du|}=\delta(u)|Du|$
 \end{itemize}
 \end{frame}

%
%
% Reference 
\begin{frame}{Referenze}
\begin{thebibliography}{Goldbach, 1742}
\bibitem{Goldbach1742}[Goldbach, 1742]
  Christian Goldbach
  \newblock A open problem
  \newblock \emph{Letter to Leonhard Eulero}, 1742.

\bibitem{Goldbach1742}[Goldbach, 1742]
  Christian Goldbach
  \newblock A open problem
  \newblock \emph{Letter to Leonhard Eulero}, 1742.
\end{thebibliography}
\end{frame}
%
%
%
\end{document}


